\documentclass[a4paper,10pt]{article}
\usepackage{eumat}

\begin{document}
\begin{eulernotebook}
\begin{eulercomment}
Nama  : Fiorenza Rafa Talitha Sevandra\\
NIM   : 23030630051\\
Kelas : Matematika E 2023

\end{eulercomment}
\eulersubheading{}
\eulersubheading{1. Pengenalan software aplikasi matematika}
\begin{eulercomment}
a. Hal hal yang dipelajari beserta contohnya\\
\end{eulercomment}
\begin{eulerttcomment}
   - Daftar software matematika dan kegunaanya
   - Beberapa software matematika
\end{eulerttcomment}
\begin{eulercomment}
MATLAB\\
MATLAB (singkatan dari "matrix labolatory") adalah aplikasi untuk
komputasi numerik multiparadigma dan bahasa pemrograman komersial yang
dikembangkan oleh MathWorks. MATLAB memungkinkan manipulasi matriks,
menggambar grafik fungsi dan data, implementasi algoritma, pembuatan
antarmuka pengguna, dan antarmuka dengan program yang ditulis dalam
bahasa lain.

SPSS\\
SPSS digunakan oleh berbagai universitas, institusi, dan perusahaan
untuk melakukan analisis data. Berikut beberapa contoh penggunaan
SPSS, yaitu:\\
- Melakukan riset pemasaran (market research).\\
- Analisis data survey atau kuesioner.\\
- Populer digunakan untuk penelitian akademik mahasiswa.\\
- Populer digunakan oleh keperluan pemerintahan seperti lembaga BPS.\\
- Data mining.\\
- Membantu untuk pengambilan keputusan suatu perusahaan.\\
- Penelitian kesehatan masyarakat.\\
- Mendokumentasikan data.\\
- Representasi data statistik.\\
- Memprediksi suatu kejadian time series

Maple\\
Maple adalah sistem perangkat lunak matematika berbasis komputer,
yaitu komputer sistem aljabar dari Waterloo Maple Sofware (WMS) (Tung,
2003:3). Program yang dikembangkan mencakup tentang penyelesaian
matematika untuk mendukung berbagai topik operasi matematika yang
meliputi analisis numerik, aljabar simbolik, kalkulus, persamaan
differensial, aljabar linier dan grafik untuk melukiskan suatu
peristiwa yang sulit teramati atau bersifat abstrak. Maple bersifat
simbolik dan mampu memanipulasi solusi aljabar dengan tampilan
berbagai mode plot dan berbagai grafik dua dimensi, tiga dimensi, dan
animasi.

Graphmatica\\
Graphmatica merupakan aplikasi untuk menggambar grafik persamaan yang
handal, mudah digunakan, dengan fitur numerik dan kalkulus

Math Script Code\\
Math ScriptConsole (MSC) memungkinkan untuk melakukan penghitungan
matematika menggunakan skrip dan perintah dalam antarmuka
multi-jendela berbasis console, dapat menjalankan satu perintah secara
interaktif atau menjalankan sekumpulan perintah dari file skrip. MSC
adalah pengganti ideal untuk kalkulator windows yang terlalu sederhana
untuk ilmuwan dan insinyur. Math ScriptConsole berada di bawah
kekuatan dan fleksibilitas mesin Microsoft VBScript. Ini mencakup
fungsi tambahan dan perpustakaan yang dirancang khusus untuk ilmuwan
dan insinyur. Terlebih lagi, sangat mudah untuk menambahkan fungsi
sendiri dan membuat perpustakaan sendiri.

Mathcad Prime\\
Mathcad adalah perangkat lunak komputer terutama ditujukan untuk
verifikasi, validasi, dokumentasi dan penggunaan ulang perhitungan
teknik. Pertama diperkenalkan pada tahun 1986 pada MS-DOS , itu adalah
orang pertama yang memperkenalkan mengedit langsung dari notasi
matematika mengeset, dikombinasikan dengan perhitungan otomatis nya.

GeoGebra\\
GeoGebra adalah software matematika dinamis untuk semua jenjang
pendidikan yang menyatukan geometri, aljabar, spreadsheet, grafik,
statistik, dan kalkulus dalam satu paket yang mudah digunakan.
GeoGebra adalah software dengan komunitas jutaan pengguna yang
berkembang pesat yang tersebar di hampir setiap negara. GeoGebra telah
menjadi penyedia terkemuka perangkat lunak matematika dinamis, yang
mendukung pendidikan sains, teknologi, teknik, dan matematika (STEM)
serta inovasi dalam pengajaran dan pembelajaran di seluruh dunia.

LaTeX\\
LaTeX adalah software seperti Word yang digunakan untuk membuat
dokumen seperti buku, laporan, tesis dan makalah. Perbedaan utama
antara LaTeX dan Word adalah Microsoft Word menyediakan teks yang
sudah diformat khusus yang bisa dipilih dari menu-menunya sehingga apa
yang dilihat di layar monitor akan sangat menyerupai hasilnya ketika
dicetak. Pada LaTeX teks masih berbentuk plaintext, yaitu teks yang
belum diformat. Proses formatting teks dilakukan dengan menggunakan
bahasa markup.

Euler Maths Toolbox\\
Euler kemungkinan untuk melakukan komputasi numerik, simbolik atau
integer, serta statistik dan optimasi. Ekspresi simbolik dan fungsi
yang terintegrasi dengan sintaks Euler dan memungkinkan penggunaan
bilangan real dan kompleks, aritmatika interval string dan vektor
string, solusi untuk sistem linear dan dukungan untuk matriks. Euler
Math Toolbox adalah alat yang sangat serbaguna dan membantu yang
sangat berguna untuk menghitung persamaan yang kompleks.

Microsoft Mathematics\\
Microsoft Mathematics adalah aplikasi yang dibuat untuk membantu
pelajar dan mahasiswa dalam belajar matematika. Fitur yang terdapat
pada Microsoft Mathematics yaitu kalkulator sains, visualisasi grafik
secara 2D dan 3D, komputasi operasi aljabar serta komputasi simbolik
untuk fungsi-fungsi dan operasi matematika yang elementer. Microsoft
Math adalah program edukasi, dibuat untuk sistem operasi Microsoft
Windows, yang membantu pengguna untuk menyelesaikan permasalahan
matematika and sains. Dibangun dan diprakarsai oleh Microsoft,
Microsoft Math secara pokok ditargetkan untuk pelajar sebagai alat
bantu belajar. Add-on yang menawarkan fitur dari Microsoft Math
tersedia untuk Microsoft Word 2007. Add-on ini juga diprakarsai oleh
Microsoft dan tersedia gratis untuk pengguna Microsoft Word yang
berhasil memvalidasi Microsoft Word mereka melalui Genuine Microsoft
Program.

R\\
Fungsi R diantaranya digunakan untuk riset dan akademis, karena
software R sangat cocok untuk riset, baik secara statistik, ekonomi,
komputasi numerik, dan pemrograman komputer. Karena didukung oleh
banyak tenaga ahli di bidangnya masing-masing, R layak digunakan
sebagai perangkat lunak yang dijadikan acuan bagi berbagai kalangan,
diantaranya dikalangan akademik (dosen, mahasiswa).

GNU Octave\\
GNU Octave adalah bahasa tingkat tinggi, terutama ditujukan untuk
komputasi numerik. Software ini menyediakan antarmuka baris perintah
yang nyaman untuk memecahkan masalah linier dan nonlinier secara
numerik, dan untuk melakukan eksperimen numerik lainnya menggunakan
bahasa yang sebagian besar kompatibel dengan Matlab. Octave juga dapat
digunakan sebagai bahasa berorientasi batch.

SageMath\\
SageMath adalah sebuah aplikasi matematika yang dapat digunakan untuk
komputasi tingkat dasar sampai tingkat lanjut, matematika terapan
maupun matematika teori. Fitur-fiktur yang terdapat di dalam SageMath
meliputi beberapa aspek matematika seperti aljabar, kombinatorik,
teori grafik, teori bilangan, kalkulus, analisis numerik dan
statistik. Tujuan dari pengembangan SageMath yaitu agar dapat
menggunakan aplikasi-aplikasi matematika tersebut dalam sebuah lembar
kerja secara langsung tanpa harus berpindah-pindah aplikasi. Komputasi
numerik dapat dilakukan dengan fungsi-fungsi yang disediakan oleh
SageMath atau dengan melalui aplikasi untuk komputasi numerik yang
diintegrasikan dalam SageMath. SageMath dapat digunakan secara online
maupun dengan cara diinstal pada komputer.

Maxima\\
Maxima merupakan salah satu software open source yang mempunyai
kemampuan untuk pembelajaran matematika topik aljabar, kalkulus,
aritmetika, dan grafik. Maxima dikembangkan oleh MACSYMA system,
dimana Maxima merupakan salah satu Computer Algebra System (CAS) yang
mengkombinasikan kemampuan grafis, simbol, dan numerik. Maxima dapat
digunakan untuk menyelesaikan pekerjaan-pekerjaan yang berkaitan
dengan turunan, integral, persamaan linier, persamaan polynomial,
fungsi Laurent, deret Taylor, grafik 2D dan 3D, dan beberapa pekerjaan
lainnya

GAP\\
Software GAP merupakan suatu software yang memuat fungsi, operasi, dan
struktur aljabar. Penggunaan software dalam bidang matematika
dapatmembuat mahasiswa lebih memahami suatu materi. GAP sebagai
pendukung sistem pembelajaran aljabar abstrak yang dilakukan secara
tradisional, yaitu cara pengajaran yang fokus utamanya adalah
meggunakan konsep - konsep dasar aljabar abstrak, belajar tentang
bagaimana pembuktian teorema?teorema aljabar abstrak dan memahami
struktur aljabar abstrak.


b. Hal hal yang dilakukan dalam mempelajari materi\\
- Mencari informasi tentang software software matematika.\\
- Mempelajari software matematika yang sudah dicari.\\
- Membagi software yang ditemukan berdasarkan kategori (komersil,
sharaware, freeware, dan open source).\\
- Membagi software berdasarkan kegunaan (CAS, analisis numerik,
geometri, statistika, pendidikan)\\
- Mencari informasi sejarah singkat software.

c. Kendala kendala dan usaha untuk mengatasi kendala tersebut\\
- Kesulitan dalam membagi software berdasarkan kategori dan kegunaan,
solusinya mencari informasi melalui youtube dan melihat cara kerja
software tersebut.\\
- Kesulitan menentukan software yang cocok untuk menyelesaikan masalah
matematika, solusinya mempelajari modul yang telah dibagikan oleh
dosen.\\
\end{eulercomment}
\eulersubheading{}
\begin{eulerprompt}
> 
\end{eulerprompt}
\eulersubheading{2. Pengenalan software Euler Maths Toolbox (EMT)}
\begin{eulercomment}
a. Hal hal yang dipelajari beserta contohnya\\
\end{eulercomment}
\begin{eulerttcomment}
   - Mempelajari software EMT
   - Memasang software EMT
   - Mempelajari Panduaan singkat dan pengantar penggunaan EMT
\end{eulerttcomment}
\begin{eulercomment}
Panduan ini ditulis dengan Euler dalam bentuk notebook Euler, yang
berisi teks (deskriptif), baris-baris perintah, tampilan hasil
perintah (numerik, ekspresi matematika, atau gambar/plot), dan gambar
yang disisipkan dari file gambar.

Untuk menambah jendela EMT, Anda dapat menekan [F11]. EMT akan
menampilkan jendela grafik di layar desktop Anda. Tekan [F11] lagi
untuk kembali ke tata letak favorit Anda. Tata letak disimpan untuk
sesi berikutnya.

Anda juga dapat menggunakan [Ctrl]+[G] untuk menyembunyikan jendela
grafik. Selanjutnya Anda dapat beralih antara grafik dan teks dengan
tombol [TAB].

Seperti yang Anda baca, notebook ini berisi tulisan (teks) berwarna
hijau, yang dapat Anda edit dengan mengklik kanan teks atau tekan menu
Edit -\textgreater{} Edit Comment atau tekan [F5], dan juga baris perintah EMT yang
ditandai dengan "\textgreater{}" dan berwarna merah. Anda dapat menyisipkan baris
perintah baru dengan cara menekan tiga tombol bersamaan:
[Shift]+[Ctrl]+[Enter].

\end{eulercomment}
\eulersubheading{Komentar (Teks Uraian)}
\begin{eulercomment}
Komentar atau teks penjelasan dapat berisi beberapa "markup" dengan
sintaks sebagai berikut.

\end{eulercomment}
\begin{eulerttcomment}
   - * Judul
   - ** Sub-Judul
   - latex: F (x) = \(\backslash\)int_a^x f (t) \(\backslash\), dt
   - mathjax: \(\backslash\)frac\{x^2-1\}\{x-1\} = x + 1
   - maxima: 'integrate(x^3,x) = integrate(x^3,x) + C
   - http://www.euler-math-toolbox.de
   - See: http://www.google.de | Google
   - image: logo.png
   - ---
\end{eulerttcomment}
\begin{eulercomment}

Hasil sintaks-sintaks di atas (tanpa diawali tanda strip) adalah
sebagai berikut.

\begin{eulercomment}
\eulerheading{Judul}
\begin{eulercomment}
\end{eulercomment}
\eulersubheading{Sub-Judul}
\begin{eulercomment}
\end{eulercomment}
\begin{eulerformula}
\[
F(x) = \int_a^x f(t) \, dt
\]
\end{eulerformula}
\begin{eulerformula}
\[
\frac{x^2-1}{x-1} = x + 1
\]
\end{eulerformula}
\begin{eulerformula}
\[
\int {x^3}{\;dx}=\frac{x^4}{4}+\mbox{ C }
\]
\end{eulerformula}
\begin{eulercomment}
http://www.euler-math-toolbox.de\\
See: http://www.google.de \textbar{} Google\\
image: logo.png
\end{eulercomment}
\eulerheading{Baris Perintah}
\begin{eulercomment}
EMT berorientasi pada baris perintah dan dapat menuliskan satu atau
lebih perintah dalam satu baris perintah. Setiap perintah harus
diakhiri dengan koma atau titik koma.

- Titik koma menyembunyikan output (hasil) dari perintah.\\
- Sebuah koma mencetak hasilnya.\\
- Setelah perintah terakhir, koma diasumsikan secara otomatis (boleh
tidak ditulis).

Dalam contoh berikut, kita mendefinisikan variabel r yang diberi nilai
1,25. Output dari definisi ini adalah nilai variabel. Tetapi karena
tanda titik koma, nilai ini tidak ditampilkan. Pada kedua perintah di
belakangnya, hasil kedua perhitungan tersebut ditampilkan.
\end{eulercomment}
\begin{eulerprompt}
>r=1.25; pi*r^2, 2*pi*r
\end{eulerprompt}
\begin{euleroutput}
  4.90873852123
  7.85398163397
\end{euleroutput}
\eulerheading{Satuan}
\begin{eulercomment}
EMT dapat mengubah unit satuan menjadi sistem standar internasional
(SI). Tambahkan satuan di belakang angka untuk konversi sederhana.
\end{eulercomment}
\begin{eulerprompt}
>1miles  // 1 mil = 1609,344 m
\end{eulerprompt}
\begin{euleroutput}
  1609.344
\end{euleroutput}
\begin{eulercomment}
Beberapa satuan yang sudah dikenal di dalam EMT adalah sebagai
berikut. Semua unit diakhiri dengan tanda dolar (\textdollar{}), namun boleh tidak
perlu ditulis dengan mengaktifkan easyunits.

kilometer\textdollar{}:=1000;\\
km\textdollar{}:=kilometer\textdollar{};\\
cm\textdollar{}:=0.01;\\
mm\textdollar{}:=0.001;\\
minute\textdollar{}:=60;\\
min\textdollar{}:=minute\textdollar{};\\
minutes\textdollar{}:=minute\textdollar{};\\
hour\textdollar{}:=60*minute\textdollar{};\\
h\textdollar{}:=hour\textdollar{};\\
hours\textdollar{}:=hour\textdollar{};\\
day\textdollar{}:=24*hour\textdollar{};\\
days\textdollar{}:=day\textdollar{};\\
d\textdollar{}:=day\textdollar{};\\
year\textdollar{}:=365.2425*day\textdollar{};\\
years\textdollar{}:=year\textdollar{};\\
y\textdollar{}:=year\textdollar{};\\
inch\textdollar{}:=0.0254;\\
in\textdollar{}:=inch\textdollar{};\\
feet\textdollar{}:=12*inch\textdollar{};\\
foot\textdollar{}:=feet\textdollar{};\\
ft\textdollar{}:=feet\textdollar{};\\
yard\textdollar{}:=3*feet\textdollar{};\\
yards\textdollar{}:=yard\textdollar{};\\
yd\textdollar{}:=yard\textdollar{};\\
mile\textdollar{}:=1760*yard\textdollar{};\\
miles\textdollar{}:=mile\textdollar{};\\
kg\textdollar{}:=1;\\
sec\textdollar{}:=1;\\
ha\textdollar{}:=10000;\\
Ar\textdollar{}:=100;\\
Tagwerk\textdollar{}:=3408;\\
Acre\textdollar{}:=4046.8564224;\\
pt\textdollar{}:=0.376mm;

Untuk konversi ke dan antar unit, EMT menggunakan operator khusus,
yakni -\textgreater{}.
\end{eulercomment}
\begin{eulerprompt}
>4km -> miles, 4inch -> " mm"
\end{eulerprompt}
\begin{euleroutput}
  2.48548476895
  101.6 mm
\end{euleroutput}
\eulerheading{Format Tampilan Nilai}
\begin{eulercomment}
Akurasi internal untuk nilai bilangan di EMT adalah standar IEEE,
sekitar 16 digit desimal. Aslinya, EMT tidak mencetak semua digit
suatu bilangan. Ini untuk menghemat tempat dan agar terlihat lebih
baik. Untuk mengatrtamilan satu bilangan, operator berikut dapat
digunakan.

\end{eulercomment}
\begin{eulerprompt}
>pi
\end{eulerprompt}
\begin{euleroutput}
  3.14159265359
\end{euleroutput}
\begin{eulerprompt}
>longest pi
\end{eulerprompt}
\begin{euleroutput}
        3.141592653589793 
\end{euleroutput}
\begin{eulerprompt}
>long pi
\end{eulerprompt}
\begin{euleroutput}
  3.14159265359
\end{euleroutput}
\begin{eulerprompt}
>short pi
\end{eulerprompt}
\begin{euleroutput}
  3.1416
\end{euleroutput}
\begin{eulerprompt}
>shortest pi
\end{eulerprompt}
\begin{euleroutput}
     3.1 
\end{euleroutput}
\begin{eulerprompt}
>fraction pi
\end{eulerprompt}
\begin{euleroutput}
  312689/99532
\end{euleroutput}
\begin{eulerprompt}
>short 1200*1.03^10, long E, longest pi
\end{eulerprompt}
\begin{euleroutput}
  1612.7
  2.71828182846
        3.141592653589793 
\end{euleroutput}
\begin{eulercomment}
Format aslinya untuk menampilkan nilai menggunakan sekitar 10 digit.
Format tampilan nilai dapat diatur secara global atau hanya untuk satu
nilai.

Anda dapat mengganti format tampilan bilangan untuk semua perintah
selanjutnya. Untuk mengembalikan ke format aslinya dapat digunakan
perintah "defformat" atau "reset".
\end{eulercomment}
\begin{eulerprompt}
>longestformat; pi, defformat; pi
\end{eulerprompt}
\begin{euleroutput}
  3.141592653589793
  3.14159265359
\end{euleroutput}
\begin{eulercomment}
Kernel numerik EMT bekerja dengan bilangan titik mengambang (floating point)
dalam presisi ganda IEEE (berbeda dengan bagian simbolik EMT). Hasil numerik
dapat ditampilkan dalam bentuk pecahan.
\end{eulercomment}
\begin{eulerprompt}
>1/7+1/4, fraction %
\end{eulerprompt}
\begin{euleroutput}
  0.392857142857
  11/28
\end{euleroutput}
\eulerheading{Perintah Multibaris}
\begin{eulercomment}
Perintah multi-baris membentang di beberapa baris yang terhubung
dengan "..." di setiap akhir baris, kecuali baris terakhir. Untuk
menghasilkan tanda pindah baris tersebut, gunakan tombol
[Ctrl]+[Enter]. Ini akan menyambung perintah ke baris berikutnya dan
menambahkan "..." di akhir baris sebelumnya. Untuk menggabungkan suatu
baris ke baris sebelumnya, gunakan [Ctrl]+[Backspace].

Contoh perintah multi-baris berikut dapat dijalankan setiap kali
kursor berada di salah satu barisnya. Ini juga menunjukkan bahwa ...
harus berada di akhir suatu baris meskipun baris tersebut memuat
komentar.
\end{eulercomment}
\begin{eulerprompt}
>a=4; b=15; c=2; // menyelesaikan a*x^2+b*x+c=0 secara manual ...
>D=sqrt(b^2/(a^2*4)-c/a); ...
>-b/(2*a) + D, ...
>-b/(2*a) - D
\end{eulerprompt}
\begin{euleroutput}
  -0.138444501319
  -3.61155549868
\end{euleroutput}
\eulerheading{Menampilkan Daftar Variabe}
\begin{eulercomment}
Untuk menampilkan semua variabel yang sudah pernah Anda definisikan
sebelumnya (dan dapat dilihat kembali nilainya), gunakan perintah
"listvar".
\end{eulercomment}
\begin{eulerprompt}
>listvar
\end{eulerprompt}
\begin{euleroutput}
  z                   2+3i
  p1                  (x^3+1)/(x+1)
  A                   Type: Real Matrix (3x3)
  r                   1.25
  a                   4
  b                   15
  D                   1.73655549868123
  v                   Type: Real Matrix (1x6)
  c                   2
  w                   Type: Real Matrix (1x11)
\end{euleroutput}
\begin{eulercomment}
Perintah listvar hanya menampilkan variabel buatan pengguna.
Dimungkinkan untuk menampilkan variabel lain, dengan menambahkan
string  termuat di dalam nama variabel yang diinginkan.

Perlu Anda perhatikan, bahwa EMT membedakan huruf besar dan huruf
kecil. Jadi variabel "d" berbeda dengan variabel "D".

Contoh berikut ini menampilkan semua unit yang diakhiri dengan "m"
dengan mencari semua variabel yang berisi "m\textdollar{}".
\end{eulercomment}
\begin{eulerprompt}
>listvar m$
\end{eulerprompt}
\begin{euleroutput}
  km$                 1000
  cm$                 0.01
  mm$                 0.001
  nm$                 1853.24496
  gram$               0.001
  m$                  1
  hquantum$           6.62606957e-34
  atm$                101325
\end{euleroutput}
\eulerheading{Menampilkan Panduan}
\begin{eulercomment}
Untuk mendapatkan panduan tentang penggunaan perintah atau fungsi di EMT, buka
jendela panduan dengan menekan [F1] dan cari fungsinya. Anda juga dapat
mengklik dua kali pada fungsi yang tertulis di baris perintah atau di teks
untuk membuka jendela panduan.

Coba klik dua kali pada perintah "intrandom" berikut ini!
\end{eulercomment}
\begin{eulerprompt}
>intrandom(10,6)
\end{eulerprompt}
\begin{euleroutput}
  [3,  4,  4,  6,  2,  6,  1,  6,  1,  3]
\end{euleroutput}
\begin{eulercomment}
Di jendela panduan, Anda dapat mengklik kata apa saja untuk menemukan
referensi atau fungsi.

Misalnya, coba klik kata "random" di jendela panduan. Kata tersebut
boleh ada dalam teks atau di bagian "See:" pada panduan. Anda akan
menemukan penjelasan fungsi "random", untuk menghasilkan bilangan acak
berdistribusi uniform antara 0,0 dan 1,0. Dari panduan untuk "random"
Anda dapat menampilkan panduan untuk fungsi "normal", dll.
\end{eulercomment}
\begin{eulerprompt}
>random(10)
\end{eulerprompt}
\begin{euleroutput}
  [0.52143,  0.428893,  0.168134,  0.182742,  0.288048,  0.750042,
  0.472935,  0.324407,  0.340388,  0.195494]
\end{euleroutput}
\begin{eulerprompt}
>normal(10)
\end{eulerprompt}
\begin{euleroutput}
  [-0.277172,  -1.4568,  0.658046,  1.53204,  -2.47296,  -0.937381,
  0.836566,  -0.233827,  0.221796,  -0.541047]
\end{euleroutput}
\eulerheading{Matriks dan Vektor}
\begin{eulercomment}
EMT merupakan suatu aplikasi matematika yang mengerti "bahasa matriks". Artinya,
EMT menggunakan vektor dan matriks untuk perhitungan-perhitungan tingkat lanjut.
Suatu vektor atau matriks dapat didefinisikan dengan tanda kurung siku.
Elemen-elemennya dituliskan di dalam tanda kurung siku, antar elemen dalam satu
baris dipisahkan oleh koma(,), antar baris dipisahkan oleh titik koma (;).

Vektor dan matriks dapat diberi nama seperti variabel biasa.
\end{eulercomment}
\begin{eulerprompt}
>v=[4,5,6,3,2,1]
\end{eulerprompt}
\begin{euleroutput}
  [4,  5,  6,  3,  2,  1]
\end{euleroutput}
\begin{eulerprompt}
>A=[1,2,3;4,5,6;7,8,9]
\end{eulerprompt}
\begin{euleroutput}
              1             2             3 
              4             5             6 
              7             8             9 
\end{euleroutput}
\begin{eulercomment}
Karena EMT mengerti bahasa matriks, EMT memiliki kemampuan yang sangat canggih
untuk melakukan perhitungan matematis untuk masalah-masalah aljabar linier,
statistika, dan optimisasi.

Vektor juga dapat didefinisikan dengan menggunakan rentang nilai dengan interval
tertentu menggunakan tanda titik dua (:),seperti contoh berikut ini.
\end{eulercomment}
\begin{eulerprompt}
>c=1:5
\end{eulerprompt}
\begin{euleroutput}
  [1,  2,  3,  4,  5]
\end{euleroutput}
\begin{eulerprompt}
>w=0:0.1:1
\end{eulerprompt}
\begin{euleroutput}
  [0,  0.1,  0.2,  0.3,  0.4,  0.5,  0.6,  0.7,  0.8,  0.9,  1]
\end{euleroutput}
\begin{eulerprompt}
>mean(w^2)
\end{eulerprompt}
\begin{euleroutput}
  0.35
\end{euleroutput}
\eulerheading{Bilangan Kompleks}
\begin{eulercomment}
EMT juga dapat menggunakan bilangan kompleks. Tersedia banyak fungsi
untuk bilangan kompleks di EMT. Bilangan imaginer

\end{eulercomment}
\begin{eulerformula}
\[
i = \sqrt{-1}
\]
\end{eulerformula}
\begin{eulercomment}
dituliskan dengan huruf I (huruf besar I), namun akan ditampilkan
dengan huruf i (i kecil).

\end{eulercomment}
\begin{eulerttcomment}
  re(x) : bagian riil pada bilangan kompleks x.
  im(x) : bagian imaginer pada bilangan kompleks x.
  complex(x) : mengubah bilangan riil x menjadi bilangan kompleks.
  conj(x) : Konjugat untuk bilangan bilangan komplkes x.
  arg(x) : argumen (sudut dalam radian) bilangan kompleks x.
  real(x) : mengubah x menjadi bilangan riil.
\end{eulerttcomment}
\begin{eulercomment}

Apabila bagian imaginer x terlalu besar, hasilnya akan menampilkan
pesan kesalahan.

\end{eulercomment}
\begin{eulerttcomment}
  >sqrt(-1) // Error!
  >sqrt(complex(-1))
\end{eulerttcomment}
\begin{eulerprompt}
>z=2+3*I, re(z), im(z), conj(z), arg(z), deg(arg(z)), deg(arctan(3/2))
\end{eulerprompt}
\begin{euleroutput}
  2+3i
  2
  3
  2-3i
  0.982793723247
  56.309932474
  56.309932474
\end{euleroutput}
\begin{eulerprompt}
>deg(arg(I)) // 90°
\end{eulerprompt}
\begin{euleroutput}
  90
\end{euleroutput}
\begin{eulerprompt}
>sqrt(-1)
\end{eulerprompt}
\begin{euleroutput}
  Floating point error!
  Error in sqrt
  Error in:
  sqrt(-1) ...
          ^
\end{euleroutput}
\begin{eulerprompt}
>sqrt(complex(-1))
\end{eulerprompt}
\begin{euleroutput}
  0+1i
\end{euleroutput}
\begin{eulercomment}
EMT selalu menganggap semua hasil perhitungan berupa bilangan riil dan tidak
akan secara otomatis mengubah ke bilangan kompleks.

Jadi akar kuadrat -1 akan menghasilkan kesalahan, tetapi akar kuadrat kompleks
didefinisikan untuk bidang koordinat dengan cara seperti biasa. Untuk mengubah
bilangan riil menjadi kompleks, Anda dapat menambahkan 0i atau menggunakan
fungsi "complex".
\end{eulercomment}
\begin{eulerprompt}
>complex(-1), sqrt(%)
\end{eulerprompt}
\begin{euleroutput}
  -1+0i 
  0+1i
\end{euleroutput}
\eulerheading{Matematika Simbolik}
\begin{eulercomment}
Untuk melakukan perhitungan matematika simbolis di EMT, awali perintah
Maxima dengan tanda "\&". Setiap ekspresi yang dimulai dengan "\&"
adalah ekspresi simbolis dan dikerjakan oleh Maxima.
\end{eulercomment}
\begin{eulerprompt}
>&(a+b)^2
\end{eulerprompt}
\begin{euleroutput}
  
                                        2
                                 (b + a)
  
\end{euleroutput}
\begin{eulerprompt}
>&expand((a+b)^2), &factor(x^2+5*x+6)
\end{eulerprompt}
\begin{euleroutput}
  
                              2            2
                             b  + 2 a b + a
  
  
                             (x + 2) (x + 3)
  
\end{euleroutput}
\begin{eulerprompt}
>&solve(a*x^2+b*x+c,x) // rumus abc
\end{eulerprompt}
\begin{euleroutput}
  
                       2                         2
               - sqrt(b  - 4 a c) - b      sqrt(b  - 4 a c) - b
          [x = ----------------------, x = --------------------]
                        2 a                        2 a
  
\end{euleroutput}
\begin{eulerprompt}
>&(a^2-b^2)/(a+b), &ratsimp(%) // ratsimp menyederhanakan bentuk pecahan
\end{eulerprompt}
\begin{euleroutput}
  
                                  2    2
                                 a  - b
                                 -------
                                  b + a
  
  
                                  a - b
  
\end{euleroutput}
\begin{eulerprompt}
>10! // nilai faktorial (modus EMT)
\end{eulerprompt}
\begin{euleroutput}
  3628800
\end{euleroutput}
\begin{eulerprompt}
>&10! //nilai faktorial (simbolik dengan Maxima)
\end{eulerprompt}
\begin{euleroutput}
  
                                 3628800
  
\end{euleroutput}
\begin{eulercomment}
Untuk menggunakan perintah Maxima secara langsung (seperti perintah pada layar
Maxima) awali perintahnya dengan tanda "::" pada baris perintah EMT. Sintaks
Maxima disesuaikan dengan sintaks EMT (disebut "modus kompatibilitas").
\end{eulercomment}
\begin{eulerprompt}
>factor(1000) // mencari semua faktor 1000 (EMT)
\end{eulerprompt}
\begin{euleroutput}
  [2,  2,  2,  5,  5,  5]
\end{euleroutput}
\begin{eulerprompt}
>:: factor(1000) // faktorisasi prima 1000 (dengan Maxima) 
\end{eulerprompt}
\begin{euleroutput}
  
                                   3  3
                                  2  5
  
\end{euleroutput}
\begin{eulerprompt}
>:: factor(20!)
\end{eulerprompt}
\begin{euleroutput}
  
                          18  8  4  2
                         2   3  5  7  11 13 17 19
  
\end{euleroutput}
\begin{eulercomment}
Jika Anda sudah mahir menggunakan Maxima, Anda dapat menggunakan sintaks asli
perintah Maxima dengan menggunakan tanda ":::" untuk mengawali setiap perintah
Maxima di EMT. Perhatikan, harus ada spasi antara ":::" dan perintahnya.
\end{eulercomment}
\begin{eulerprompt}
>::: binomial(5,2); // nilai C(5,2)
\end{eulerprompt}
\begin{euleroutput}
  
                                    10
  
\end{euleroutput}
\begin{eulerprompt}
>::: binomial(m,4); // C(m,4)=m!/(4!(m-4)!)
\end{eulerprompt}
\begin{euleroutput}
  
                        (m - 3) (m - 2) (m - 1) m
                        -------------------------
                                   24
  
\end{euleroutput}
\begin{eulerprompt}
>::: trigexpand(cos(x+y)); // rumus cos(x+y)=cos(x) cos(y)-sin(x)sin(y) 
\end{eulerprompt}
\begin{euleroutput}
  
                      cos(x) cos(y) - sin(x) sin(y)
  
\end{euleroutput}
\begin{eulerprompt}
>::: trigexpand(sin(x+y));
\end{eulerprompt}
\begin{euleroutput}
  
                      cos(x) sin(y) + sin(x) cos(y)
  
\end{euleroutput}
\begin{eulerprompt}
>::: trigsimp(((1-sin(x)^2)*cos(x))/cos(x)^2+tan(x)*sec(x)^2) //menyederhanakan fungsi trigonometri
\end{eulerprompt}
\begin{euleroutput}
  
                                         4
                             sin(x) + cos (x)
                             ----------------
                                    3
                                 cos (x)
  
\end{euleroutput}
\begin{eulercomment}
Untuk menyimpan ekspresi simbolik ke dalam suatu variabel digunakan tanda "\&=".
\end{eulercomment}
\begin{eulerprompt}
>p1 &= (x^3+1)/(x+1)
\end{eulerprompt}
\begin{euleroutput}
  
                                   3
                                  x  + 1
                                  ------
                                  x + 1
  
\end{euleroutput}
\begin{eulerprompt}
>&ratsimp(p1)
\end{eulerprompt}
\begin{euleroutput}
  
                                 2
                                x  - x + 1
  
\end{euleroutput}
\begin{eulercomment}
Untuk mensubstitusikan suatu nilai ke dalam variabel dapat digunakan perintah
"with".
\end{eulercomment}
\begin{eulerprompt}
>&p1 with x=3 // (3^3+1)/(3+1)
\end{eulerprompt}
\begin{euleroutput}
  
                                    7
  
\end{euleroutput}
\begin{eulerprompt}
>&p1 with x=a+b, &ratsimp(%) //substitusi dengan variabel baru
\end{eulerprompt}
\begin{euleroutput}
  
                                      3
                               (b + a)  + 1
                               ------------
                                b + a + 1
  
  
                       2                  2
                      b  + (2 a - 1) b + a  - a + 1
  
\end{euleroutput}
\begin{eulerprompt}
>&diff(p1,x) //turunan p1 terhadap x
\end{eulerprompt}
\begin{euleroutput}
  
                                2      3
                             3 x      x  + 1
                             ----- - --------
                             x + 1          2
                                     (x + 1)
  
\end{euleroutput}
\begin{eulerprompt}
>&integrate(p1,x) // integral p1 terhadap x
\end{eulerprompt}
\begin{euleroutput}
  
                               3      2
                            2 x  - 3 x  + 6 x
                            -----------------
                                    6
  
\end{euleroutput}
\eulerheading{Tampilan Matematika Simbolik dengan LaTeX}
\begin{eulercomment}
Menampilkan hasil perhitunagn simbolik secara lebih bagus menggunakan
LaTeX. Untuk melakukan hal ini, tambahkan tanda dolar (\textdollar{}) di depan
tanda \& pada setiap perintah Maxima.\\
Perhatikan, hal ini hanya dapat menghasilkan tampilan yang diinginkan
apabila komputer sudah terpasang software LaTeX.
\end{eulercomment}
\begin{eulerprompt}
>$&(a+b)^2
\end{eulerprompt}
\begin{eulerformula}
\[
\left(b+a\right)^2
\]
\end{eulerformula}
\begin{eulerprompt}
>$&expand((a+b)^2), $&factor(x^2+5*x+6)
\end{eulerprompt}
\begin{eulerformula}
\[
b^2+2\,a\,b+a^2
\]
\end{eulerformula}
\begin{eulerformula}
\[
\left(x+2\right)\,\left(x+3\right)
\]
\end{eulerformula}
\begin{eulerprompt}
>$&solve(a*x^2+b*x+c,x) // rumus abc
\end{eulerprompt}
\begin{eulerformula}
\[
\left[ x=\frac{-\sqrt{b^2-4\,a\,c}-b}{2\,a} , x=\frac{\sqrt{b^2-4\,
 a\,c}-b}{2\,a} \right] 
\]
\end{eulerformula}
\begin{eulerprompt}
>$&(a^2-b^2)/(a+b), $&ratsimp(%)
\end{eulerprompt}
\begin{eulerformula}
\[
\frac{a^2-b^2}{b+a}
\]
\end{eulerformula}
\begin{eulerformula}
\[
a-b
\]
\end{eulerformula}
\begin{eulercomment}
b. Hal hal yang dilakukan dalam mempelajari materi\\
- Mencari informasi tentang EMT\\
- Memasang Software EMT\\
- Belajar menjalankan perintah perintah di EMT.

c. Kendala kendala dan usaha untuk mengatasi kendala tersebut\\
- Kesulitan dalam memasang software EMT, solusinya dengan melihat
panduan cara memasang EMT dan melihat youtube.\\
- Kesulitan dalam menggunakan beberapa perintah di software EMT,
solusinya dengan mempelajari materi yang ada di besmart dan melihat
pandauan EMT.\\
\end{eulercomment}
\eulersubheading{}
\begin{eulerprompt}
> 
\end{eulerprompt}
\eulerheading{3. Penggunaan software EMT untuk aplikasi Aljabar}
\begin{eulercomment}
a. Hal hal yang dipelajari beserta contohnya\\
- Melakukan operasi bentuk-bentuk aljabar\\
- Melakukan perhitungan dengan berbagai operasi dan fungsi matematika\\
- Melakukan perhitungan menggunakan bilangan kompleks\\
- Melakukan perhitungan menggunakan fungsi-fungsi buatan sendiri\\
- Menyelesaikan persamaan dan sistem persamaan\\
- Menyelesaikan pertidaksamaan dan sistem pertidaksamaan\\
- Melakukan manipuasi dan perhitungan menggunakan matriks dan vektor\\
- Menggunakan aljabar untuk menyelesaikan masalah sehari-hari atau
dalam matematika dan bidang lain.\\
\end{eulercomment}
\eulersubheading{Contoh pertama}
\begin{eulercomment}
Menyederhanakan bentuk aljabar:

\end{eulercomment}
\begin{eulerformula}
\[
6x^{-3}y^5\times -7x^2y^{-9}
\]
\end{eulerformula}
\begin{eulercomment}
\end{eulercomment}
\begin{eulerprompt}
>$&6*x^(-3)*y^5*-7*x^2*y^(-9)
\end{eulerprompt}
\begin{eulerformula}
\[
-\frac{42}{x\,y^4}
\]
\end{eulerformula}
\begin{eulercomment}
Menyederhanakan fungsi :\\
\end{eulercomment}
\begin{eulerformula}
\[
2y^2+2x^2-3y^2+2x^2
\]
\end{eulerformula}
\begin{eulerprompt}
>$&2*y^2+2*x^2-3*y^2+2*x^2 
\end{eulerprompt}
\begin{eulerformula}
\[
4\,x^2-y^2
\]
\end{eulerformula}
\begin{eulercomment}
\end{eulercomment}
\eulersubheading{Baris Perintah}
\begin{eulercomment}
Baris perintah berikut hanya akan mencetak hasil dari ekspresi, bukan
dari penugasan atau perintah format.
\end{eulercomment}
\begin{eulerprompt}
>r:=8; h:=4; pi*r^2*h/3
\end{eulerprompt}
\begin{euleroutput}
  268.082573106
\end{euleroutput}
\begin{eulercomment}
Perintah harus dipisahkan dengan spasi. Baris perintah berikut
mencetak kedua hasilnya.
\end{eulercomment}
\begin{eulerprompt}
>pi*2*r*h, %+2*pi*r*h // Ingat tanda % menyatakan hasil perhitungan terakhir sebelumnya
\end{eulerprompt}
\begin{euleroutput}
  201.06192983
  402.123859659
\end{euleroutput}
\begin{eulercomment}
Baris perintah dieksekusi dalam urutan yang pengguna tekan tombol
"return". Jadi, Anda akan mendapatkan nilai baru setiap kali Anda
menjalankan baris kedua.
\end{eulercomment}
\begin{eulerprompt}
>x :=9; 
>x := cos(x) // nilai cosinus (x dalam radian)
\end{eulerprompt}
\begin{euleroutput}
  -0.911130261885
\end{euleroutput}
\begin{eulerprompt}
>x := cos(x)
\end{eulerprompt}
\begin{euleroutput}
  0.612853011666
\end{euleroutput}
\begin{eulercomment}
Jika dua garis terhubung dengan "..." kedua garis akan selalu
dieksekusi secara bersamaan.
\end{eulercomment}
\begin{eulerprompt}
>x := 1.5; ...
>x := (x+2/x)/2, x := (x+2/x)/2, x := (x+2/x)/2, 
\end{eulerprompt}
\begin{euleroutput}
  1.41666666667
  1.41421568627
  1.41421356237
\end{euleroutput}
\begin{eulercomment}
Ini juga merupakan cara yang baik untuk menyebarkan perintah panjang
pada dua atau lebih baris. Anda dapat menekan Ctrl+Return untuk
membagi garis menjadi dua pada posisi kursor saat ini, atau Ctrl+Back
untuk menggabungkan garis.

Untuk melipat semua multi-garis tekan Ctrl + L. Kemudian garis-garis
berikutnya hanya akan terlihat, jika salah satunya memiliki fokus.
Untuk melipat satu multi-baris, mulailah baris pertama dengan "\%+".
\end{eulercomment}
\begin{eulerprompt}
>%+ x=4+5; ...
\end{eulerprompt}
\begin{eulercomment}
Garis yang dimulai dengan \%\% tidak akan terlihat sama sekali.
\end{eulercomment}
\begin{euleroutput}
  81
\end{euleroutput}
\begin{eulerprompt}
> 
\end{eulerprompt}
\begin{eulercomment}
Struktur kondisional juga berfungsi.
\end{eulercomment}
\begin{eulerprompt}
>if E^pi>pi^E; then "Thought so!", endif;
\end{eulerprompt}
\begin{euleroutput}
  Thought so!
\end{euleroutput}
\begin{eulercomment}
Ketika Anda menjalankan sebuah perintah, kursor dapat berada di posisi
mana saja dalam baris perintah. Anda dapat kembali ke perintah
sebelumnya atau melompat ke perintah berikutnya dengan menggunakan
tombol panah. Atau Anda dapat mengklik pada bagian komentar di atas
perintah untuk menuju ke perintah tersebut.

Ketika Anda memindahkan kursor sepanjang baris, pasangan-pasangan
tanda kurung atau tanda kurung akan disorot. Perhatikan juga baris
status. Setelah tanda kurung buka dari fungsi sqrt(), baris status
akan menampilkan teks bantuan untuk fungsi tersebut. Jalankan perintah
dengan tombol "return".
\end{eulercomment}
\begin{eulerprompt}
>sqrt(sin(90°)/cos(20°))
\end{eulerprompt}
\begin{euleroutput}
  1.03158992457
\end{euleroutput}
\begin{eulerprompt}
>3^-7
\end{eulerprompt}
\begin{euleroutput}
  0.000457247370828
\end{euleroutput}
\begin{eulercomment}
Untuk melihat bantuan untuk perintah terbaru, buka jendela bantuan
dengan tombol F1. Di sana, Anda dapat memasukkan teks untuk mencari
informasi. Pada baris kosong, bantuan untuk jendela bantuan akan
ditampilkan. Anda dapat menekan tombol escape untuk menghapus baris
tersebut atau untuk menutup jendela bantuan.

Anda juga dapat melakukan klik ganda pada setiap perintah untuk
membuka bantuan untuk perintah tersebut. Cobalah melakukan klik ganda
pada perintah "exp" di bawah ini dalam baris perintah.




\end{eulercomment}
\begin{eulerprompt}
>exp(log(2.9))
\end{eulerprompt}
\begin{euleroutput}
  2.9
\end{euleroutput}
\begin{eulerprompt}
>sin(90°)
\end{eulerprompt}
\begin{euleroutput}
  1
\end{euleroutput}
\eulersubheading{Bilangan Asli}
\begin{eulercomment}
Tipe data utama dalam Euler adalah bilangan real. Real
direpresentasikan dalam format IEEE dengan akurasi sekitar 16 digit
desimal.
\end{eulercomment}
\begin{eulerprompt}
>longest 1/3
\end{eulerprompt}
\begin{euleroutput}
       0.3333333333333333 
\end{euleroutput}
\begin{eulerprompt}
>longest 1/9
\end{eulerprompt}
\begin{euleroutput}
       0.1111111111111111 
\end{euleroutput}
\begin{eulerprompt}
>printhex(1/89)
\end{eulerprompt}
\begin{euleroutput}
  2.E05C0B81702E0*16^-2
\end{euleroutput}
\begin{eulerprompt}
>printdual(1/10)
\end{eulerprompt}
\begin{euleroutput}
  1.1001100110011001100110011001100110011001100110011010*2^-4
\end{euleroutput}
\begin{eulerprompt}
>printhex(1/99)
\end{eulerprompt}
\begin{euleroutput}
  2.95FAD40A57EB6*16^-2
\end{euleroutput}
\begin{eulerprompt}
> 
\end{eulerprompt}
\eulersubheading{String}
\begin{eulercomment}
Sebuah string dalam Euler didefinisikan dengan "...".
\end{eulercomment}
\begin{eulerprompt}
>"A string can contain anything."
\end{eulerprompt}
\begin{euleroutput}
  A string can contain anything.
\end{euleroutput}
\begin{eulercomment}
String dapat digabungkan dengan \textbar{} atau dengan +. Ini juga berfungsi
dengan angka, yang dikonversi menjadi string dalam kasus itu.
\end{eulercomment}
\begin{eulerprompt}
>"The area of the circle with radius " + 2 + " cm is " + pi*4 + " cm^2."
\end{eulerprompt}
\begin{euleroutput}
  The area of the circle with radius 2 cm is 12.5663706144 cm^2.
\end{euleroutput}
\begin{eulercomment}
Fungsi print juga mengonversi angka menjadi string. Ini dapat
mengambil sejumlah digit dan sejumlah tempat (0 untuk keluaran padat),
dan secara optimal satu unit.
\end{eulercomment}
\begin{eulerprompt}
>"Golden Ratio : " + print((1+sqrt(5))/2,5,0)
\end{eulerprompt}
\begin{euleroutput}
  Golden Ratio : 1.61803
\end{euleroutput}
\begin{eulercomment}
Ada string khusus tidak ada, yang tidak dicetak. Itu dikembalikan oleh
beberapa fungsi, ketika hasilnya tidak masalah. (Ini dikembalikan
secara otomatis, jika fungsi tidak memiliki pernyataan pengembalian.)
\end{eulercomment}
\begin{eulerprompt}
>none
\end{eulerprompt}
\begin{eulercomment}
Untuk mengonversi string menjadi angka, cukup evaluasi saja. Ini juga
berfungsi untuk ekspresi (lihat di bawah).
\end{eulercomment}
\begin{eulerprompt}
>"1234.5"()
\end{eulerprompt}
\begin{euleroutput}
  1234.5
\end{euleroutput}
\begin{eulercomment}
Untuk mendefinisikan vektor string, gunakan notasi vektor [...].
\end{eulercomment}
\begin{eulerprompt}
>v:=["affe","charlie","bravo"]
\end{eulerprompt}
\begin{euleroutput}
  affe
  charlie
  bravo
\end{euleroutput}
\begin{eulercomment}
Vektor string kosong dilambangkan dengan [none]. Vektor string dapat
digabungkan.
\end{eulercomment}
\begin{eulerprompt}
>w:=[none]; w|v|v
\end{eulerprompt}
\begin{euleroutput}
  affe
  charlie
  bravo
  affe
  charlie
  bravo
\end{euleroutput}
\begin{eulercomment}
I
\end{eulercomment}
\eulersubheading{Nilai Boolean}
\begin{eulercomment}
Nilai Boolean direpresentasikan dengan 1=true atau 0=false dalam
Euler. String dapat dibandingkan, seperti halnya angka.
\end{eulercomment}
\begin{eulerprompt}
>2<1, "apel"<"banana"
\end{eulerprompt}
\begin{euleroutput}
  0
  1
\end{euleroutput}
\begin{eulercomment}
"dan" adalah operator "\&\&" dan "atau" adalah operator "\textbar{}\textbar{}", seperti
dalam bahasa C. (Kata-kata "dan" dan "atau" hanya dapat digunakan
dalam kondisi untuk "jika".)
\end{eulercomment}
\begin{eulerprompt}
>2<E && E<3
\end{eulerprompt}
\begin{euleroutput}
  1
\end{euleroutput}
\eulerheading{Ekspresi}
\begin{eulercomment}
String atau nama dapat digunakan untuk menyimpan ekspresi matematika,
yang dapat dievaluasi oleh EMT. Untuk ini, gunakan tanda kurung
setelah ekspresi. Jika Anda bermaksud menggunakan string sebagai
ekspresi, gunakan konvensi untuk menamakannya "fx" atau "fxy" dll.
Ekspresi lebih diutamakan daripada fungsi.

Variabel global dapat digunakan dalam evaluasi.
\end{eulercomment}
\begin{eulerprompt}
>r:=2; fx:="pi*r^2"; longest fx()
\end{eulerprompt}
\begin{euleroutput}
        12.56637061435917 
\end{euleroutput}
\begin{eulercomment}
Parameter ditetapkan ke x, y, dan z dalam urutan itu. Parameter
tambahan dapat ditambahkan menggunakan parameter yang ditetapkan.
\end{eulercomment}
\begin{eulerprompt}
>fx:="a*sin(x)^2"; fx(5,a=-1)
\end{eulerprompt}
\begin{euleroutput}
  -0.919535764538
\end{euleroutput}
\begin{eulercomment}
\begin{eulercomment}
\eulerheading{Matematika Simbolik}
\begin{eulercomment}
\end{eulercomment}
\begin{eulerprompt}
>$&44!
\end{eulerprompt}
\begin{eulerformula}
\[
2658271574788448768043625811014615890319638528000000000
\]
\end{eulerformula}
\begin{eulercomment}
Dengan cara ini, Anda dapat menghitung hasil yang besar dengan tepat.
Mari kita hitung

lateks: C(44,10) = \textbackslash{}frac\{44!\}\{34! \textbackslash{}cdot 10!\}
\end{eulercomment}
\begin{eulerprompt}
>$& 44!/(34!*10!) // nilai C(44,10)
\end{eulerprompt}
\begin{eulerformula}
\[
2481256778
\]
\end{eulerformula}
\begin{eulerprompt}
>::: factor(10!)
\end{eulerprompt}
\begin{euleroutput}
  
                                 8  4  2
                                2  3  5  7
  
\end{euleroutput}
\begin{eulerprompt}
>:: factor(20!)
\end{eulerprompt}
\begin{euleroutput}
  
                          18  8  4  2
                         2   3  5  7  11 13 17 19
  
\end{euleroutput}
\begin{eulerprompt}
>::: av:g$ av^3;
\end{eulerprompt}
\begin{euleroutput}
  
                                     3
                                    g
  
\end{euleroutput}
\begin{eulerprompt}
>fx &= x^3*exp(x), &fx
\end{eulerprompt}
\begin{euleroutput}
  
                                   3  x
                                  x  E
  
  
                                   3  x
                                  x  E
  
\end{euleroutput}
\begin{eulercomment}
Variabel tersebut dapat digunakan dalam ekspresi simbolik lainnya.
Perhatikan, bahwa dalam perintah berikut sisi kanan \&= dievaluasi
sebelum penugasan ke Fx.
\end{eulercomment}
\begin{eulerprompt}
>&(fx with x=5), $%, &float(%)
\end{eulerprompt}
\begin{euleroutput}
  
                                       5
                                  125 E
  
\end{euleroutput}
\begin{eulerformula}
\[
125\,e^5
\]
\end{eulerformula}
\begin{euleroutput}
  
                            18551.64488782208
  
\end{euleroutput}
\begin{eulerprompt}
>fx(5)
\end{eulerprompt}
\begin{euleroutput}
  18551.6448878
\end{euleroutput}
\eulerheading{Fungsi}
\begin{eulercomment}
Dalam EMT, fungsi adalah program yang didefinisikan dengan perintah
"fungsi". Ini bisa berupa fungsi satu baris atau fungsi multibaris.\\
Fungsi satu baris dapat berupa numerik atau simbolis. Fungsi satu
baris numerik didefinisikan oleh ":=".
\end{eulercomment}
\begin{eulerprompt}
>function f(x) := x*sqrt(x^2+1)
\end{eulerprompt}
\begin{eulercomment}
Untuk gambaran umum, kami menunjukkan semua kemungkinan definisi untuk
fungsi satu baris. Suatu fungsi dapat dievaluasi sama seperti fungsi
Euler bawaan lainnya.
\end{eulercomment}
\begin{eulerprompt}
>f(2)
\end{eulerprompt}
\begin{euleroutput}
  4.472135955
\end{euleroutput}
\begin{eulercomment}
Fungsi ini akan bekerja untuk vektor juga, dengan mematuhi bahasa
matriks Euler, karena ekspresi yang digunakan dalam fungsi
divektorkan.
\end{eulercomment}
\begin{eulerprompt}
>f(0:0.1:1)
\end{eulerprompt}
\begin{euleroutput}
  [0,  0.100499,  0.203961,  0.313209,  0.430813,  0.559017,  0.699714,
  0.854459,  1.0245,  1.21083,  1.41421]
\end{euleroutput}
\begin{eulercomment}
Fungsi dapat diplot. Alih-alih ekspresi, kita hanya perlu memberikan
nama fungsi.

Berbeda dengan ekspresi simbolik atau numerik, nama fungsi harus
diberikan dalam string.
\end{eulercomment}
\begin{eulerprompt}
>solve("f",1,y=1)
\end{eulerprompt}
\begin{euleroutput}
  0.786151377757
\end{euleroutput}
\begin{eulercomment}
Secara default, jika Anda perlu menimpa fungsi bawaan, Anda harus
menambahkan kata kunci "menimpa". Menimpa fungsi bawaan berbahaya dan
dapat menyebabkan masalah untuk fungsi lain tergantung pada fungsi
tersebut.

Anda masih dapat memanggil fungsi bawaan sebagai "\_...", jika itu
adalah fungsi di inti Euler.
\end{eulercomment}
\begin{eulerprompt}
>function overwrite sin (x) := _sin(x°) // redine sine in degrees
>sin(45)
\end{eulerprompt}
\begin{euleroutput}
  0.707106781187
\end{euleroutput}
\begin{eulercomment}
Lebih baik kita menghapus redefinisi dosa ini.
\end{eulercomment}
\begin{eulerprompt}
>forget sin; sin(pi/4)
\end{eulerprompt}
\begin{euleroutput}
  0.707106781187
\end{euleroutput}
\eulersubheading{Parameter Default}
\begin{eulercomment}
Fungsi numerik dapat memiliki parameter default.
\end{eulercomment}
\begin{eulerprompt}
>function f(x,a=1) := a*x^2
\end{eulerprompt}
\begin{eulercomment}
Menghilangkan parameter ini menggunakan nilai default.
\end{eulercomment}
\begin{eulerprompt}
>f(4)
\end{eulerprompt}
\begin{euleroutput}
  16
\end{euleroutput}
\begin{eulercomment}
Menyetelnya akan menimpa nilai default.
\end{eulercomment}
\begin{eulerprompt}
>f(4,5)
\end{eulerprompt}
\begin{euleroutput}
  80
\end{euleroutput}
\begin{eulercomment}
Parameter yang ditetapkan menimpanya juga. Ini digunakan oleh banyak
fungsi Euler seperti plot2d, plot3d.
\end{eulercomment}
\begin{eulerprompt}
>f(4,a=1)
\end{eulerprompt}
\begin{euleroutput}
  16
\end{euleroutput}
\begin{eulercomment}
Untuk meringkas

- \&= mendefinisikan fungsi simbolis,\\
- := mendefinisikan fungsi numerik,\\
- \&\&= mendefinisikan fungsi simbolis murni.

\begin{eulercomment}
\eulerheading{Memecahkan Ekspresi}
\begin{eulercomment}
Ekspresi dapat diselesaikan secara numerik dan simbolis.

Untuk menyelesaikan ekspresi sederhana dari satu variabel, kita dapat
menggunakan fungsi solve(). Perlu nilai awal untuk memulai pencarian.
Secara internal, solve() menggunakan metode secant.
\end{eulercomment}
\begin{eulerprompt}
>solve("x^2-2",1)
\end{eulerprompt}
\begin{euleroutput}
  1.41421356237
\end{euleroutput}
\begin{eulercomment}
Ini juga berfungsi untuk ekspresi simbolis. Ambil fungsi berikut.
\end{eulercomment}
\begin{eulerprompt}
>$&solve(x^2=2,x)
\end{eulerprompt}
\begin{eulerformula}
\[
\left[ x=-\sqrt{2} , x=\sqrt{2} \right] 
\]
\end{eulerformula}
\eulerheading{Menyelesaikan Pertidaksamaan}
\begin{eulercomment}
Untuk menyelesaikan pertidaksamaan, EMT tidak akan dapat melakukannya,
melainkan dengan bantuan Maxima, artinya secara eksak (simbolik).
Perintah Maxima yang digunakan adalah fourier\_elim(), yang harus
dipanggil dengan perintah "load(fourier\_elim)" terlebih dahulu.
\end{eulercomment}
\begin{eulerprompt}
>&load(fourier_elim)
\end{eulerprompt}
\begin{euleroutput}
  
          C:/Program Files/Euler x64/maxima/share/maxima/5.35.1/share/f\(\backslash\)
  ourier_elim/fourier_elim.lisp
  
\end{euleroutput}
\begin{eulerprompt}
>$&fourier_elim([x^2 - 1>0],[x]) // x^2-1 > 0
\end{eulerprompt}
\begin{eulerformula}
\[
\left[ 1<x \right] \lor \left[ x<-1 \right] 
\]
\end{eulerformula}
\begin{eulerprompt}
>$&fourier_elim([x^2 - 1<0],[x]) // x^2-1 < 0
\end{eulerprompt}
\begin{eulerformula}
\[
\left[ -1<x , x<1 \right] 
\]
\end{eulerformula}
\eulerheading{Bahasa Matriks}
\begin{eulercomment}
Dokumentasi inti EMT berisi diskusi terperinci tentang bahasa matriks
Euler.

Vektor dan matriks dimasukkan dengan tanda kurung siku, elemen
dipisahkan dengan koma, baris dipisahkan dengan titik koma.
\end{eulercomment}
\begin{eulerprompt}
>A=[1,2;80,9]
\end{eulerprompt}
\begin{euleroutput}
              1             2 
             80             9 
\end{euleroutput}
\begin{eulercomment}
Produk matriks dilambangkan dengan titik.
\end{eulercomment}
\begin{eulerprompt}
>b=[3;4]
\end{eulerprompt}
\begin{euleroutput}
              3 
              4 
\end{euleroutput}
\begin{eulerprompt}
>b' // transpose b
\end{eulerprompt}
\begin{euleroutput}
  [3,  4]
\end{euleroutput}
\begin{eulerprompt}
>inv(A) //inverse A
\end{eulerprompt}
\begin{euleroutput}
     -0.0596026      0.013245 
       0.529801   -0.00662252 
\end{euleroutput}
\begin{eulerprompt}
>A.b //perkalian matriks
\end{eulerprompt}
\begin{euleroutput}
             11 
            276 
\end{euleroutput}
\begin{eulerprompt}
>A.inv(A)
\end{eulerprompt}
\begin{euleroutput}
              1             0 
              0             1 
\end{euleroutput}
\begin{eulercomment}
Poin utama dari bahasa matriks adalah bahwa semua fungsi dan operator
bekerja elemen untuk elemen.
\end{eulercomment}
\begin{eulerprompt}
>A.A
\end{eulerprompt}
\begin{euleroutput}
            161            20 
            800           241 
\end{euleroutput}
\begin{eulerprompt}
>A^2 //perpangkatan elemen2 A
\end{eulerprompt}
\begin{euleroutput}
              1             4 
           6400            81 
\end{euleroutput}
\begin{eulerprompt}
>A.A.A
\end{eulerprompt}
\begin{euleroutput}
           1761           502 
          20080          3769 
\end{euleroutput}
\eulerheading{Fungsi Matriks Lainnya (Membangun Matriks)}
\begin{eulercomment}
Untuk membangun matriks, kita dapat menumpuk satu matriks di atas yang
lain. Jika keduanya tidak memiliki jumlah kolom yang sama, kolom yang
lebih pendek akan diisi dengan 0.
\end{eulercomment}
\begin{eulerprompt}
>v=1:3; v_v
\end{eulerprompt}
\begin{euleroutput}
              1             2             3 
              1             2             3 
\end{euleroutput}
\eulerheading{Menyortir dan Mengacak}
\begin{eulercomment}
Fungsi sort() mengurutkan vektor baris.
\end{eulercomment}
\begin{eulerprompt}
>sort([5,6,4,8,1,9])
\end{eulerprompt}
\begin{euleroutput}
  [1,  4,  5,  6,  8,  9]
\end{euleroutput}
\begin{eulercomment}
Seringkali perlu untuk mengetahui indeks dari vektor yang diurutkan
dalam vektor aslinya. Ini dapat digunakan untuk menyusun ulang vektor
lain dengan cara yang sama.

Mari kita mengocok vektor.
\end{eulercomment}
\begin{eulerprompt}
>v=shuffle(1:10)
\end{eulerprompt}
\begin{euleroutput}
  [3,  5,  8,  6,  7,  1,  4,  10,  2,  9]
\end{euleroutput}
\eulerheading{Aljabar linier}
\begin{eulercomment}
EMT memiliki banyak fungsi untuk menyelesaikan sistem linier, sistem
sparse, atau masalah regresi.

Untuk sistem linier Ax=b, Anda dapat menggunakan algoritma Gauss,
matriks invers atau kecocokan linier. Operator A\textbackslash{}b menggunakan versi
algoritma Gauss.
\end{eulercomment}
\begin{eulerprompt}
>A=[1,2;3,4]; b=[5;6]; A\(\backslash\)b
\end{eulerprompt}
\begin{euleroutput}
             -4 
            4.5 
\end{euleroutput}
\begin{eulercomment}
Untuk contoh lain, kami membuat matriks 200x200 dan jumlah barisnya.
Kemudian kita selesaikan Ax=b menggunakan matriks invers. Kami
mengukur kesalahan sebagai deviasi maksimal semua elemen dari 1, yang
tentu saja merupakan solusi yang benar.
\end{eulercomment}
\begin{eulerprompt}
>A=normal(200,200); b=sum(A); longest totalmax(abs(inv(A).b-1))
\end{eulerprompt}
\begin{euleroutput}
    2.666755705149626e-13 
\end{euleroutput}
\begin{eulercomment}
Jika sistem tidak memiliki solusi, kecocokan linier meminimalkan norma
kesalahan Ax-b.
\end{eulercomment}
\begin{eulerprompt}
>A=[1,2,3;4,5,6;7,8,9]
\end{eulerprompt}
\begin{euleroutput}
              1             2             3 
              4             5             6 
              7             8             9 
\end{euleroutput}
\begin{eulercomment}
Determinan matriks ini adalah 0.
\end{eulercomment}
\begin{eulerprompt}
>det(A)
\end{eulerprompt}
\begin{euleroutput}
  0
\end{euleroutput}
\eulerheading{Matriks Simbolik}
\begin{eulercomment}
Maxima memiliki matriks simbolis. Tentu saja, Maxima dapat digunakan
untuk masalah aljabar linier sederhana seperti itu. Kita dapat
mendefinisikan matriks untuk Euler dan Maxima dengan \&:=, dan kemudian
menggunakannya dalam ekspresi simbolis. Bentuk [...] biasa untuk
mendefinisikan matriks dapat digunakan di Euler untuk mendefinisikan
matriks simbolik.
\end{eulercomment}
\begin{eulerprompt}
>A &= [a,1,1;1,a,1;1,1,a]; $A
\end{eulerprompt}
\begin{eulerformula}
\[
\begin{pmatrix}a & 1 & 1 \\ 1 & a & 1 \\ 1 & 1 & a \\ \end{pmatrix}
\]
\end{eulerformula}
\begin{eulerprompt}
>$&det(A), $&factor(%)
\end{eulerprompt}
\begin{eulerformula}
\[
a\,\left(a^2-1\right)-2\,a+2
\]
\end{eulerformula}
\begin{eulerformula}
\[
\left(a-1\right)^2\,\left(a+2\right)
\]
\end{eulerformula}
\begin{eulerprompt}
>$&invert(A) with a=0
\end{eulerprompt}
\begin{eulerformula}
\[
\begin{pmatrix}-\frac{1}{2} & \frac{1}{2} & \frac{1}{2} \\ \frac{1
 }{2} & -\frac{1}{2} & \frac{1}{2} \\ \frac{1}{2} & \frac{1}{2} & -
 \frac{1}{2} \\ \end{pmatrix}
\]
\end{eulerformula}
\begin{eulerprompt}
>A &= [1,a;b,2]; $A
\end{eulerprompt}
\begin{eulerformula}
\[
\begin{pmatrix}1 & a \\ b & 2 \\ \end{pmatrix}
\]
\end{eulerformula}
\begin{eulercomment}
Seperti semua variabel simbolik, matriks ini dapat digunakan dalam
ekspresi simbolik lainnya.
\end{eulercomment}
\begin{eulerprompt}
>$&det(A-x*ident(2)), $&solve(%,x)
\end{eulerprompt}
\begin{eulerformula}
\[
\left(1-x\right)\,\left(2-x\right)-a\,b
\]
\end{eulerformula}
\begin{eulerformula}
\[
\left[ x=\frac{3-\sqrt{4\,a\,b+1}}{2} , x=\frac{\sqrt{4\,a\,b+1}+3
 }{2} \right] 
\]
\end{eulerformula}
\begin{eulercomment}
Nilai eigen juga dapat dihitung secara otomatis. Hasilnya adalah
vektor dengan dua vektor nilai eigen dan multiplisitas.
\end{eulercomment}
\begin{eulerprompt}
>$&eigenvalues([a,1;1,a])
\end{eulerprompt}
\begin{eulerformula}
\[
\left[ \left[ a-1 , a+1 \right]  , \left[ 1 , 1 \right]  \right] 
\]
\end{eulerformula}
\begin{eulercomment}
Untuk mengekstrak vektor eigen tertentu perlu pengindeksan yang
cermat.
\end{eulercomment}
\begin{eulerprompt}
>$&eigenvectors([a,1;1,a]), &%[2][1][1]
\end{eulerprompt}
\begin{eulerformula}
\[
\left[ \left[ \left[ a-1 , a+1 \right]  , \left[ 1 , 1 \right] 
  \right]  , \left[ \left[ \left[ 1 , -1 \right]  \right]  , \left[ 
 \left[ 1 , 1 \right]  \right]  \right]  \right] 
\]
\end{eulerformula}
\begin{euleroutput}
  
                                 [1, - 1]
  
\end{euleroutput}
\eulersubheading{Penggunaan dalam kehidupan sehari-hari}
\begin{eulercomment}
Demo - Suku Bunga



Asumsikan Anda memiliki modal awal 5000 (katakanlah dalam dolar).
\end{eulercomment}
\begin{eulerprompt}
>K=5000
\end{eulerprompt}
\begin{euleroutput}
  5000
\end{euleroutput}
\begin{eulercomment}
Sekarang kita asumsikan tingkat bunga 3\% per tahun. Mari kita
tambahkan satu tarif sederhana dan hitung hasilnya.
\end{eulercomment}
\begin{eulerprompt}
>K*1.03
\end{eulerprompt}
\begin{euleroutput}
  5150
\end{euleroutput}
\begin{eulercomment}
Euler akan memahami sintaks berikut juga.
\end{eulercomment}
\begin{eulerprompt}
>K+K*3%
\end{eulerprompt}
\begin{euleroutput}
  5150
\end{euleroutput}
\begin{eulercomment}
Tetapi lebih mudah menggunakan faktornya
\end{eulercomment}
\begin{eulerprompt}
>q=1+3%, K*q
\end{eulerprompt}
\begin{euleroutput}
  1.03
  5150
\end{euleroutput}
\begin{eulercomment}
Selama 10 tahun, kita cukup mengalikan faktornya dan mendapatkan nilai
akhir dengan suku bunga majemuk.
\end{eulercomment}
\begin{eulerprompt}
>K*q^10
\end{eulerprompt}
\begin{euleroutput}
  6719.58189672
\end{euleroutput}
\begin{eulercomment}
Untuk tujuan kita, kita dapat mengatur format menjadi 2 digit setelah
titik desimal.
\end{eulercomment}
\begin{eulerprompt}
>format(12,2); K*q^10
\end{eulerprompt}
\begin{euleroutput}
      6719.58 
\end{euleroutput}
\begin{eulercomment}
Mari kita cetak yang dibulatkan menjadi 2 digit dalam kalimat lengkap.
\end{eulercomment}
\begin{eulerprompt}
>"Starting from " + K + "$ you get " + round(K*q^10,2) + "$."
\end{eulerprompt}
\begin{euleroutput}
  Starting from 5000$ you get 6719.58$.
\end{euleroutput}
\begin{eulercomment}
Bagaimana jika kita ingin mengetahui hasil antara dari tahun 1 sampai
tahun 9? Untuk ini, bahasa matriks Euler sangat membantu. Anda tidak
harus menulis loop, tetapi cukup masukkan
\end{eulercomment}
\begin{eulerprompt}
>K*q^(0:10)
\end{eulerprompt}
\begin{euleroutput}
  Real 1 x 11 matrix
  
      5000.00     5150.00     5304.50     5463.64     ...
\end{euleroutput}
\begin{eulercomment}
Bagaimana keajaiban ini bekerja? Pertama ekspresi 0:10 mengembalikan
vektor bilangan bulat.
\end{eulercomment}
\begin{eulerprompt}
>short 0:10
\end{eulerprompt}
\begin{euleroutput}
  [0,  1,  2,  3,  4,  5,  6,  7,  8,  9,  10]
\end{euleroutput}
\begin{eulercomment}
Kemudian semua operator dan fungsi dalam Euler dapat diterapkan pada
elemen vektor untuk elemen. Jadi
\end{eulercomment}
\begin{eulerprompt}
>short q^(0:10)
\end{eulerprompt}
\begin{euleroutput}
  [1,  1.03,  1.0609,  1.0927,  1.1255,  1.1593,  1.1941,  1.2299,
  1.2668,  1.3048,  1.3439]
\end{euleroutput}
\begin{eulercomment}
adalah vektor faktor q\textasciicircum{}0 sampai q\textasciicircum{}10. Ini dikalikan dengan K, dan kami
mendapatkan vektor nilai.
\end{eulercomment}
\begin{eulerprompt}
>VK=K*q^(0:10);
\end{eulerprompt}
\begin{eulercomment}
Tentu saja, cara realistis untuk menghitung suku bunga ini adalah
dengan membulatkan ke sen terdekat setelah setiap tahun. Mari kita
tambahkan fungsi untuk ini.
\end{eulercomment}
\begin{eulerprompt}
>function oneyear (K) := round(K*q,2)
\end{eulerprompt}
\begin{eulercomment}
Mari kita bandingkan dua hasil, dengan dan tanpa pembulatan.
\end{eulercomment}
\begin{eulerprompt}
>longest oneyear(1234.57), longest 1234.57*q
\end{eulerprompt}
\begin{euleroutput}
                  1271.61 
                1271.6071 
\end{euleroutput}
\begin{eulercomment}
b. Hal hal yang dilakukan dalam mempelajari materi

- Mencari materi diinternet mengenai aljabar dan pengguaan dalam
sehari-hari\\
- Mencari perintah EMT aljabar di website\\
- Membuat latihan soal berdasarkan soal yang ditemukan di buku.

c. Kendala kendala dan usaha untuk mengatasi kendala tersebut\\
- Kesulitan dalam memahami perintah di EMT, solusinya dengan mencari
informasi di internet.\\
\end{eulercomment}
\eulersubheading{}
\begin{eulerprompt}
> 
\end{eulerprompt}
\eulerheading{4. Penggunaan software EMT untuk menggambar grafik  2 dimensi (2D)}
\begin{eulercomment}
a. Hal hal yang dipelajari beserta contohnya\\
- Menggambar Grafik Fungsi Satu Variabel dalam Bentuk Ekspresi
Langsung,\\
- Menggambar Grafik Fungsi Satu Variabel yang Rumusnya Disimpan dalam
Variabel Ekspresi,\\
- Menggambar Grafik Fungsi Satu Variabel yang Fungsinya Didefinisikan
sebagai Fungsi Numerik,\\
- Menggambar Grafik Fungsi Satu Variabel yang Fungsinya Didefinisikan
sebagai Fungsi Simbolik,\\
- Menggambar Beberapa Kurva Sekaligus,\\
- Menggambar Beberapa Kurva dalam Satu Bidang Koordinat,\\
- Menuliskan Label Sumbu Koordinat, Label Kurva, dan Keterangan Kurva
(Legend),\\
- Mengatur Ukuran Gambar, Format (Style), dan Warna Kurva,\\
- Menggambar Sekumpulan Kurva dengan satu Perintah plot2d,\\
- Menggambar Kurva Fungsi Parametrik,\\
- Menggambar Kurva Fungsi Implisit,\\
- Menggambar Kurva Fungsi Kompleks,\\
- Menggambar Daerah yang Dibatasi oleh Beberapa Kurva,\\
- Menggambar Segi banyak.


\end{eulercomment}
\eulersubheading{Menggambar Grafik Fungsi Satu Variabel dalam Bentuk Ekspresi}
\begin{eulercomment}
Langsung Ekspresi tunggal

Di dalam program numerik EMT, ekspresi adalah string. Jika ditandai
sebagai simbolis, mereka akan mencetak melalui Maxima, jika tidak
melalui EMT. Ekspresi dalam string digunakan untuk membuat plot dan
banyak fungsi numerik. Untuk ini, variabel dalam ekspresi harus "x".

expresi dalam string
\end{eulercomment}
\begin{eulerprompt}
>expr := "x^5-x^2-3"
\end{eulerprompt}
\begin{euleroutput}
  x^5-x^2-3
\end{euleroutput}
\begin{eulercomment}
plot ekspresi
\end{eulercomment}
\begin{eulerprompt}
>plot2d(expr,-2,2) :
\end{eulerprompt}
\eulerimg{27}{images/Nur Alya Fadilah_Aplikom-030.png}
\eulersubheading{Menggambar Grafik Fungsi Satu Variabel yang rumusnya Disimpan dalam}
\begin{eulercomment}
Variabel Ekspresi

\end{eulercomment}
\begin{eulerttcomment}
 ekspresi
\end{eulerttcomment}
\begin{eulerprompt}
>expr &= x^5-1
\end{eulerprompt}
\begin{euleroutput}
  
                                   5
                                  x  - 1
  
\end{euleroutput}
\begin{eulercomment}
plot dari ekspresi diatas 
\end{eulercomment}
\begin{eulerprompt}
>aspect(2); plot2d(expr,-1,1):
\end{eulerprompt}
\eulerimg{13}{images/Nur Alya Fadilah_Aplikom-031.png}
\begin{eulercomment}
contoh 1
\end{eulercomment}
\begin{eulerprompt}
>expr := "x^10-x-5"
\end{eulerprompt}
\begin{euleroutput}
  x^10-x-5
\end{euleroutput}
\begin{eulerprompt}
>aspect(2) ; plot2d(expr,-1,1):
\end{eulerprompt}
\eulerimg{13}{images/Nur Alya Fadilah_Aplikom-032.png}
\begin{eulercomment}
menggunakan variabel lokal
\end{eulercomment}
\eulersubheading{Menggambar Fungsi Numerik}
\begin{eulercomment}
Fungsi Numerik adalah sebuah fungsi dengan himpunan bilangan cacah
sebagai domain dan himpunan mendasar yang melibatkan hubungan
matematis antara bilangan yang menjadi domain dan bilangan sebagai
kodomain.
\end{eulercomment}
\begin{eulerprompt}
> 
\end{eulerprompt}
\begin{eulercomment}
Fungsi numerik  memiliki  1  atau  lebih  variabel  independen, yang
sering dilambangkan sebagai "X". Variabel X adalah nilai atau
parameter yang dapat berubah, dan fungsi numerik menggambarkan
bagaimana variabel ini memengaruhi variabel dependen. Variabel
dependen adalah hasil perhitungan atau keluaran dari fungsi numerik
yang bergantung pada nilai atau perubahan dalam variabel independen.

\end{eulercomment}
\begin{eulercomment}
Dalam EMT cara mendefinisikan fungsi menggunakan syntak function.
untuk mendefinisikan fungsi numerik menggunakan tanda ":="

Fungsi  numerik  menjelaskan bagaimana bilangan  dalam  domain
berhubungan dengan bilangan sebagai kodomain, biasanya diberikan dalam
bentuk rumus matematik(persamaan) atau aturan yang memetakan setiap
domain kedalam kodomain yang sesuai. contoh:

f(x)=2x+3
\end{eulercomment}
\begin{eulerprompt}
> 
\end{eulerprompt}
\begin{eulercomment}
(x)(variabel dependen) adalah fungsi yang memetakan setiap nilai
x(variabel independen)kedalam nilai 2x+3. Terdapat berbagai jenis
fungsi yang termasuk ke dalam fungsi numerik, diantaranya:

Fungsi linier dengan bentuk umum\\
f (x) = ax + b
\end{eulercomment}
\begin{eulercomment}
Fungsi kuadrat dengan bentuk umum

f (x) = ax2 + bx + c
\end{eulercomment}
\begin{eulercomment}
Fungsi eksponensial dengan bentuk umum

f (x) = ax
\end{eulercomment}
\begin{eulercomment}
Fungsi logaritma dengan bentuk umum

f (x) = log a(x)

\end{eulercomment}
\begin{eulercomment}
Fungsi trigonometri dengan bentuk umum

f (x) = sin(x), f (x) = cos(x)

\end{eulercomment}
\begin{eulercomment}
Salah satu  cara  yang  umum  digunakan  untuk  memvisualisasikan
fungsi numerik adalah dengan menggambar grafiknya. Grafik ini
menggambarkan bagaimana variabel dependen berubah seiring perubahan
variabel independen dan membantu dalam memahami sifat-sifat fungsi,
seperti titik ekstrim
\end{eulercomment}
\eulersubheading{Menggambar Beberapa Kurva Sekaligus}
\begin{eulercomment}
Dalam subtopik ini, kita akan membahas mengenai cara menggambar
beberapa kurva sekaligus. Dalam hal ini kita dapat menggambar beberapa
kurva dalam jendela grafik yang berbeda secara bersama-sama. Untuk
membuat ini kita dapat menggunakan perintah figure(). Berikut contoh
dari menggambar beberapa kurva sekaligus

Menggambar plot fungsi\\
\end{eulercomment}
\begin{eulerformula}
\[
x^n, 1 \leq n \leq 4
\]
\end{eulerformula}
\begin{eulerprompt}
>reset;
>figure(2,2);...
>for n=1 to 4; figure(n); plot2d("x^"+n); end;...
>figure(0):
\end{eulerprompt}
\eulerimg{27}{images/Nur Alya Fadilah_Aplikom-034.png}
\begin{eulercomment}
Penjelasan sintaks dari plot fungsi

\end{eulercomment}
\begin{eulerformula}
\[
x^n,  1 \leq n \leq 4
\]
\end{eulerformula}
\begin{eulercomment}
- reset;\\
Perintah ini berguna untuk menghapus grafik yang telah ada sebelumnya,
sehingga kita dapat memulai dari awal untuk menggambar grafik\\
- figure(2x2);\\
Perintah figure() digunakan untuk membuat jendela grafik dengan ukuran\\
axb. Dalam kasus ini perintah figure(2,2) memiliki makna bahwa jendela
grafik yang dibuat berukuran 2x2. Artinya, akan ada empat jendela
grafik yang akan ditampilkan dengan tata letak 2 baris dan 2 kolom.\\
- for n=1 to 4;\\
Perintah ini digunakan untuk melakukan pengulangan (looping) perintah
sebanyak empat kali, yaitu dari 1 hingga 4.\\
- figure(n);\\
Perintah ini digunakan untuk beralih dari jendela grafik satu ke
jendela grafik lainnya (jendela grafik ke-n).\\
- plot2d("x\textasciicircum{}"+n);\\
Perintah plot2d() digunakan untuk membuat plot fungsi matematika.\\
Dalam hal ini fungsi yang diplot adalah x\textasciicircum{}n, di mana n adalah nilai
dari variabel yang sedang diulang. Dengan kata lain, ini akan membuat\\
plot dari x\textasciicircum{}1, x\textasciicircum{}2, x\textasciicircum{}3, dan x\textasciicircum{}4 dalam jendela grafik yang sesuai\\
- end;\\
Perintah ini menandakan akhir dari looping.\\
- figure(0);\\
Perintah ini digunakan untuk beralih kembali ke jendela grafik utama.
\end{eulercomment}
\eulersubheading{Menggambar Beberapa Kurva pada bidang koordinat yang sama}
\begin{eulercomment}
Plot lebih dari satu fungsi (multiple function) ke dalam satu jendela
dapat dilakukan dengan berbagai cara. Salah satu caranya adalah
menggunakan \textgreater{}add untuk beberapa panggilan ke plot2d secara
keseluruhan, kecuali panggilan pertama.

Berikut contohnya:\\
menggambar kurva\\
\end{eulercomment}
\begin{eulerformula}
\[
 f(x)=cos(x)
\]
\end{eulerformula}
\begin{eulerformula}
\[
f(x)= x^2
\]
\end{eulerformula}
\begin{eulerprompt}
>aspect(); plot2d("cos(x)",r=3); plot2d("x^2",style=".",>add):
\end{eulerprompt}
\eulerimg{27}{images/Nur Alya Fadilah_Aplikom-038.png}
\begin{eulerformula}
\[
f(x)=cos(x)-1
\]
\end{eulerformula}
\begin{eulerformula}
\[
f(x)= sin(x)-1
\]
\end{eulerformula}
\begin{eulerprompt}
>aspect(2); plot2d("cos(x)-1",-1,6); plot2d("sin(x)-1",style="--",>add):
\end{eulerprompt}
\eulerimg{13}{images/Nur Alya Fadilah_Aplikom-039.png}
\begin{eulercomment}
Selain menggunakan \textgreater{}add kita juga bisa menambahkannya secara langsung

Berikut contohnya:\\
Menggambar kurva\\
\end{eulercomment}
\begin{eulerformula}
\[
f(x)= 2x+1
\]
\end{eulerformula}
\begin{eulerformula}
\[
f(x)= -2x+1
\]
\end{eulerformula}
\begin{eulerprompt}
>plot2d(["2x+1","x"],0,8):
\end{eulerprompt}
\eulerimg{13}{images/Nur Alya Fadilah_Aplikom-040.png}
\begin{eulerformula}
\[
f(x)=sin(2x)
\]
\end{eulerformula}
\begin{eulerformula}
\[
f(x)=cos(3x)
\]
\end{eulerformula}
\begin{eulerprompt}
>aspect(1.5); plot2d(["sin(2x)","cos(3x)"],0,8):
\end{eulerprompt}
\eulerimg{17}{images/Nur Alya Fadilah_Aplikom-041.png}
\begin{eulercomment}
Kegunaan \textgreater{}add yang lain juga bisa untuk menambahkan titik pada kurva.

Berikut contohnya:\\
Menambahkan sebuah titik di\\
\end{eulercomment}
\begin{eulerformula}
\[
f(x)= x+4
\]
\end{eulerformula}
\begin{eulerprompt}
>aspect(); plot2d("x+4",-2,5,); plot2d(2,6,>points,>add):
\end{eulerprompt}
\eulerimg{27}{images/Nur Alya Fadilah_Aplikom-042.png}
\eulersubheading{Menuliskan Label koordinat,label kurva, dan keterangan}
\begin{eulercomment}
kurva(legend) Dalam EMT, untuk menambahkan judul dapat dilakukan
dengan title="..."\\
untuk menambahkan sumbu x dan sumbu y dapat dilakukan dengan x1="...",
y1="..."\\
sebagai contoh:
\end{eulercomment}
\begin{eulerprompt}
>plot2d("x^2-4*x"):
\end{eulerprompt}
\eulerimg{27}{images/Nur Alya Fadilah_Aplikom-043.png}
\begin{eulercomment}
untuk menambahkan judul dapat dilakukan dengan title="..."\\
untuk menambahkan sumbu x dan sumbu y dapat dilakukan dengan x1="...",
y1="..."
\end{eulercomment}
\begin{eulerprompt}
>plot2d("x^2-4*x",title="FUNGSI y=x^2-4*x",yl="Sumbu y",xl="Sumbu x"):
\end{eulerprompt}
\eulerimg{27}{images/Nur Alya Fadilah_Aplikom-044.png}
\eulersubheading{Mengatur ukuran gambar,format(style),dan warna kurva}
\begin{eulercomment}
Untuk mengubah ukuran, dapat dilakukan dengan menggunakan
aspect="...", semakin besar nilai aspect, maka ukuran kurva akan
semakin kecil, begitupun sebaliknya

untuk mengganti style, dapat dipilih dengan berbagai pilihan\\
style="...", dapat dipilih dari, misal : "-","\_',"-.",".-.","-.-".

untuk warna dapat dipilih sebagai salah satu warna default\\
color="...", warna default= red,green,blue,yellow, dll

sebagai contoh:
\end{eulercomment}
\begin{eulerprompt}
>aspect(1); plot2d("exp(x^2-3)"):
\end{eulerprompt}
\eulerimg{27}{images/Nur Alya Fadilah_Aplikom-045.png}
\begin{eulercomment}
ukuran kurva dapat diganti dengan mengganti nilai aspect="...",
semakin besar nilai aspect, maka ukuran kurva akan semakin kecil Untuk
mengganti warna dapat ditambahkan dengan color="...", sedangkan untuk
mengganti format(style) dapat dilakukan dengan menambahkan style="..."
\end{eulercomment}
\begin{eulerprompt}
>aspect(2); plot2d("exp(x^2-3)", color=red, style="--"):
\end{eulerprompt}
\eulerimg{13}{images/Nur Alya Fadilah_Aplikom-046.png}
\eulersubheading{Menggambar Sekumpulan Kurva dalam satu perintah plot2d.}
\begin{eulercomment}
Dalam pembahasan sub-bab 9 kali ini akan membahas mengenai bagaimana
menggambar sekumpulan kurva dalam satu perintah plot2d. Menggambar
sekumpulan kurva dalam satu perintah plot2d adalah teknik yang
digunakan untuk memvisualisasikan beberapa fungsi dalam satu grafik.
Ini memudahkan perbandingan antara beberapa kurva.\\
\end{eulercomment}
\eulersubheading{}
\begin{eulerprompt}
>plot2d(["x^2","2*x"],-3,3):
\end{eulerprompt}
\eulerimg{13}{images/Nur Alya Fadilah_Aplikom-047.png}
\eulersubheading{Menggambar Kurva Fungsi Parametrik}
\begin{eulercomment}
kurva yang didefinisikan oleh sebuah persamaan yang menghubungkan
koordinat x dan y Contohnya\\
\end{eulercomment}
\begin{eulerformula}
\[
y=x^2
\]
\end{eulerformula}
\begin{eulercomment}
Atau\\
\end{eulercomment}
\begin{eulerformula}
\[
x^2+y^2=13
\]
\end{eulerformula}
\begin{eulercomment}
dimana persamaan-persamaan ini tidak dikaitkan dengan panjang kurva s
, waktu t, dan besaran lainnya. Besaran besaran ini disebut parameter\\
persamaan parametrik adalah persamaan yang menyatakan hubungan
variabel x, y dituliskan dengan\\
\end{eulercomment}
\begin{eulerformula}
\[
x=f(t)
\]
\end{eulerformula}
\begin{eulerformula}
\[
y=g(t)
\]
\end{eulerformula}
\begin{eulercomment}
dengan a\textless{}=t\textless{}=b tiap nilai t menentukan titik(x,y) pada kurva. Jadi ,
dengan berubahnya nilai t. titik\\
\end{eulercomment}
\begin{eulerformula}
\[
(x,y) = (f(t),g(t))
\]
\end{eulerformula}
\begin{eulercomment}
bergerak sepanjang kurva yang disebut kurva parametrik


Dalam contoh berikut, kita memplot spiral

\end{eulercomment}
\begin{eulerformula}
\[
\gamma(t) = t \cdot (\cos(2\pi t),\sin(2\pi t))
\]
\end{eulerformula}
\begin{eulercomment}
Kita perlu menggunakan banyak titik untuk tampilan yang halus
\end{eulercomment}
\begin{eulerprompt}
>t=linspace(0,1,1000); ...
>plot2d(t*cos(2*pi*t),t*sin(2*pi*t),r=1):
\end{eulerprompt}
\eulerimg{13}{images/Nur Alya Fadilah_Aplikom-054.png}
\begin{eulerttcomment}
 r digunakan untuk mengatur radius marker titik-titik yang akan
\end{eulerttcomment}
\begin{eulercomment}
digunakan dalam plot.



Sebagai alternatif, dimungkinkan untuk menggunakan dua ekspresi untuk
kurva. Berikut ini plot kurva yang sama seperti di atas.
\end{eulercomment}
\begin{eulerprompt}
>plot2d("x*cos(2*pi*x)","x*sin(2*pi*x)",xmin=0,xmax=1,r=1):
\end{eulerprompt}
\eulerimg{13}{images/Nur Alya Fadilah_Aplikom-055.png}
\eulersubheading{Menggambar Grafik Bilangan Kompleks}
\begin{eulercomment}
Bilangan kompleks secara visual dapat direpresentasikan sebagai
sepasang angka (a, b) membentuk vektor pada diagram yang disebut
diagram Argand, mewakili yang bidang kompleks. Sumbu-x adalah sumbu
nyata dan sumbu-y adalah sumbu imajiner.

Menggambar kurva fungsi kompleks sendiri adalah proses visualisasi
grafis dari fungsi matematika kompleks (yaitu fungsi yang melibatkan
bilangan kompleks, yaitu bilangan dengan bagian real dan imajiner)
berperilaku dalam koordinas kompleks. Hal tersebut memungkinkan untuk
melihat bagaimana pola, bentuk, dan sifat dari fungsi kompleks
tersebut.

Array bilangan kompleks juga dapat diplot. Kemudian titik-titik grid
akan terhubung. Jika sejumlah garis kisi ditentukan (atau vektor garis
kisi 1x2) dalam argumen cgrid, hanya garis kisi tersebut yang
terlihat.

Matriks bilangan kompleks akan secara otomatis diplot sebagai kisi di
bidang kompleks.

\textgreater{} Definisi fungsi kompleks, mendefinisikan fungsi kompleks yang
dianalisis atau digambarkan. Fungsi ini memiliki variabel kompleks z,
yang melibatkan bagian real dan imajiner.\\
\textgreater{} Selanjutnya kita dapat menggunakan fungsi linspace. Fungsi linspace
sendiri adalah salah satu fungsi yang umum digunakan dalam
pemrograman, terutama dalam konteks pemrograman numerik dan ilmu data.
Ini sering digunakan untuk menghasilkan urutan nilai dalam rentang
tertentu dengan jumlah titik yang sama di antara dua titik ujungnya.
Penggunaannya tidak terbatas pada pemrosesan sinyal atau
elektromagnetik, tetapi bisa digunakan dalam berbagai konteks di mana
Anda perlu membuat urutan nilai.\\
\textgreater{} Penentuan rentang, memilih rentang nilai z yang ingin ditampilkan di
dalam plot. Rentang ini mencakup wilayah kompleks tertentu yang ingin
diamati.\\
\textgreater{} Menggunakan sintaks plot2d.\\
\textgreater{} Penyesuaian plot, mengubah plot sesuai yang diinginkan (mengubah
warna, format (style), dan sebagainya).

Dalam contoh berikut, kami memplot gambar lingkaran satuan di bawah
fungsi eksponensial. Parameter cgrid menyembunyikan beberapa kurva
grid.

\begin{eulercomment}
\eulerheading{Contoh}
\begin{eulerprompt}
>aspect(); r=linspace(0,1,50); a=linspace(0,2pi,80)'; z=r*exp(I*a);...
>plot2d(z,a=-1.25,b=1.25,c=-1.25,d=1.25,cgrid=10):
\end{eulerprompt}
\eulerimg{27}{images/Nur Alya Fadilah_Aplikom-056.png}
\begin{eulercomment}
Penjelasan:

plot2d("sin(x)\textasciicircum{}3", "cos(x)", xmin=0, xmax=2*pi, \textgreater{}filled, style="/");\\
Ini adalah perintah untuk membuat plot dari dua fungsi matematika,
yaitu sin(x)\textasciicircum{}3 dan cos(x), dalam satu plot yang sama. Berikut adalah
rincian perintah ini:

"sin(x)\textasciicircum{}3" adalah ekspresi pertama yang akan diplotkan. Ini adalah
fungsi trigonometri sin(x) yang dipangkatkan tiga. Fungsi ini
tergantung pada variabel x.

"cos(x)" adalah ekspresi kedua yang akan diplotkan. Ini adalah fungsi
trigonometri cos(x). Fungsi ini juga tergantung pada variabel x.

xmin=0 dan xmax=2*pi mengatur rentang (range) plot untuk sumbu x dari
0 hingga 2p. Ini adalah rentang yang akan ditampilkan dalam plot.

\textgreater{}filled mengisi area di bawah kurva fungsi dengan warna, sehingga area
di bawah kurva fungsi akan diisi dengan warna.

style="/" mengatur gaya plot menjadi garis miring ("\textbackslash{}"). Ini akan
menghasilkan plot dengan garis-garis miring.
\end{eulercomment}
\eulersubheading{Menggambar Segi Banyak}
\begin{eulercomment}
Data plot merupakan poligon atau segi banyak. Kita juga dapat membuat
kurva atau mengisi kurva.\\
Fungsi perintah yang digunakan untuk menggambar segi banyak atau
poligon.

Membentuk poligon dengan fungsi:\\
x=linspace(0,2pi,n); plot2d(cos(x),sin(x),r=1,\textgreater{}filled,style="..."):\\
atau\\
x=linspace(0,2pi,n);
plot2d(sin(x),cos(x),r=1,\textgreater{}filled,style="...",fillcolor=red):

Keterangan\\
- filled=true, mengisi plot.\\
- style="...": Pilih dari "#", "/", "\textbackslash{}", "\textbackslash{}/" dan gaya gaya lainnya.\\
- fillcolor: untuk memeberikan warna.

Warna isian ditentukan oleh argumen "fillcolor", dan pada \textless{}outline
opsional mencegah menggambar batas untuk semua gaya kecuali yang
default.

Poligon dalam EMT dapat digambar dengan fungsi maksimal. Dengan fungsi
maksimal ini, poligon yang dihasilkan dapat berupa poligon tak
beraturan.

A=[2,1;1,2;-1,0;0,-1];\\
function f(x,y) := max([x,y].A');\\
plot2d("f",r=4,level=[0;3],color=green,n=111):

Keterangan:\\
-A adalah titik koordinat dari poligon yang akan dibuat.\\
-"r" untuk menentukan ukuran bidang koordinat.

Berikut adalah himpunan nilai maksimal dari empat kondisi linear yang
kurang dari atau sama dengan 3. Ini merupakan A[k].v\textless{}=3 untuk semua
baris A. Untuk mendapatkan sudut yang bagus, kita menggunakan n yang
relatif besar.


1. Menggambar Segitiga
\end{eulercomment}
\begin{eulerprompt}
>x=linspace(0,2pi,3); ...
>plot2d(sin(x),cos(x),r=1):
\end{eulerprompt}
\eulerimg{27}{images/Nur Alya Fadilah_Aplikom-057.png}
\begin{eulercomment}
Segitiga diatas digambar dari kurva tertutup dengan 3 titik.

Kita dapat membuat segitiga dengan gaya yang berbeda-beda. Seperti
pada contoh berikut ini.
\end{eulercomment}
\begin{eulerprompt}
>x=linspace(0,2pi,3); ...
>plot2d(sin(x),cos(x),>filled,style="/",fillcolor=red,r=1):
\end{eulerprompt}
\eulerimg{27}{images/Nur Alya Fadilah_Aplikom-058.png}
\begin{eulerprompt}
>x=linspace(0,2pi,3); ...
>plot2d(sin(x),cos(x),>filled,style="#",fillcolor=blue,r=2):
\end{eulerprompt}
\eulerimg{27}{images/Nur Alya Fadilah_Aplikom-059.png}
\begin{eulercomment}
Dua gambar segitiga diatas memiliki gaya yang berbeda, dengan
menggunakan fungsi perintah "style=". Gambar segitiga juga dapat
dibuat dengan posisi yang berbeda, tergantung pada fungsi yang akan
diplot.


2. Menggambar Segiempat
\end{eulercomment}
\begin{eulerprompt}
>x=linspace(0,2pi,4); ...
>plot2d(cos(x),sin(x),r=1.5):
\end{eulerprompt}
\eulerimg{27}{images/Nur Alya Fadilah_Aplikom-060.png}
\begin{eulerprompt}
>x=linspace(0,2pi,4); 
>plot2d(cos(x),sin(x),r=2,>filled,outline=1):
\end{eulerprompt}
\eulerimg{27}{images/Nur Alya Fadilah_Aplikom-061.png}
\begin{eulercomment}
Gambar diatas merupakan salah satu contoh segiempat yang dapat
digambar di EMT. Fungsi perintah yang digunakan masih sama seperti
fungsi perintah untuk menggambar segitiga. 

Selain fungsi perintah diatas, untuk menggambar segi banyak, dapat
menggunakan fungsi maksimum.
\end{eulercomment}
\begin{eulerprompt}
>A=[2,1;1,2;-1,0;0,-1];
>function f(x,y) := max([x,y].A');
>plot2d("f",r=4,level=[0;3],color=yellow,n=111):
\end{eulerprompt}
\eulerimg{27}{images/Nur Alya Fadilah_Aplikom-062.png}
\begin{eulerprompt}
>A=[1,1;-1,1;-1,-1;1,-1];
>function f(x,y) := max([x,y].A');
>plot2d("f",r=1,level=[0;1],color=gray,n=90):
\end{eulerprompt}
\eulerimg{27}{images/Nur Alya Fadilah_Aplikom-063.png}
\begin{eulercomment}
Dengan fungsi maksimal ini, kita dapat menggambar segiempat atau segi
banyak sebarang.


3. Menggambar Segilima
\end{eulercomment}
\begin{eulerprompt}
>t=linspace(0,2pi,5); plot2d(sin(t),cos(t),r=1.5):
\end{eulerprompt}
\eulerimg{27}{images/Nur Alya Fadilah_Aplikom-064.png}
\begin{eulerprompt}
>t=linspace(0,2pi,5); ...
>plot2d(sin(t),cos(t),r=1.5,>filled,style="\(\backslash\)",fillcolor=orange):
\end{eulerprompt}
\eulerimg{27}{images/Nur Alya Fadilah_Aplikom-065.png}
\begin{eulerprompt}
>A=[0,5;3,2;1,-4;-1,-4;-3,2];
>function f(x,y) := max([x,y].A');
>plot2d("f",r=1,level=[0;2],color=cyan,n=111):
\end{eulerprompt}
\eulerimg{27}{images/Nur Alya Fadilah_Aplikom-066.png}
\begin{eulercomment}
4. Menggambar Segienam
\end{eulercomment}
\begin{eulerprompt}
>t=linspace(0,2pi,6); ...
>plot2d(cos(t),sin(t),r=1.2):
\end{eulerprompt}
\eulerimg{27}{images/Nur Alya Fadilah_Aplikom-067.png}
\begin{eulerprompt}
>t=linspace(0,2pi,6); ...
>plot2d(cos(t),sin(t),>filled,style="/",fillcolor=olive,r=1.2):
\end{eulerprompt}
\eulerimg{27}{images/Nur Alya Fadilah_Aplikom-068.png}
\begin{eulercomment}
5. Menggambar dekagon
\end{eulercomment}
\begin{eulerprompt}
>t=linspace(0,2pi,10); ...
>plot2d(cos(t),sin(t),r=1.2):
\end{eulerprompt}
\eulerimg{27}{images/Nur Alya Fadilah_Aplikom-069.png}
\begin{eulerprompt}
>t=linspace(0,2pi,10); ...
>plot2d(cos(t),sin(t),>filled,style="\(\backslash\)/",fillcolor=darkgray,r=1.2):
\end{eulerprompt}
\eulerimg{27}{images/Nur Alya Fadilah_Aplikom-070.png}
\begin{eulercomment}
b. Hal hal yang dilakukan dalam mempelajari materi\\
- Mempelajari EMT plot2D dan gambar gambar bangun.\\
- membuat persamaan yang ditemukan ke aplikasi EMT

c. Kendala kendala dan usaha untuk mengatasi kendala tersebut\\
- Tidak terdapat kendala dalam mempelajari materi ini karena materi
mudah dipahami.\\
\end{eulercomment}
\eulersubheading{}
\eulerheading{5. Penggunaan software EMT untuk menggambar grafik 3 dimensi (3D)}
\begin{eulercomment}
a. Hal hal yang dipelajari beserta contohnya\\
- Menggambar Grafik Fungsi Dua Variabel dalam Bentuk Ekspresi Langsung\\
- Menggambar Grafik Fungsi Dua Variabel yang Rumusnya Disimpan dalam
Variabel Ekspresi\\
- Menggambar Grafik Fungsi Dua Variabel yang Fungsinya Didefinisikan
sebagai Fungsi Numerik\\
- Menggambar Grafik Fungsi Dua Variabel yang Fungsinya Didefinisikan
sebagai Fungsi Simbolik\\
- Menggambar Data \textdollar{}x\textdollar{}, \textdollar{}y\textdollar{}, \textdollar{}z\textdollar{} pada ruang Tiga Dimensi (3D)\\
- Membuat Gambar Grafik Tiga Dimensi (3D) yang Bersifat Interaktif dan
animasi grafik 3D\\
- Menggambar Fungsi Parametrik Tiga Dimensi (3D)\\
- Menggambar Fungsi Implisit Tiga Dimensi (3D)\\
- Menggambar Titik pada ruang Tiga Dimensi (3D)\\
- Mengatur tampilan, warna dan sudut pandang gambar permukaan Tiga
Dimensi (3D)\\
- Menggambar Grafik Tiga Dimensi alam modus anaglif\\
- Menggambar Diagram Batang Tiga Dimensi\\
- Menggambar Permukaan Benda Putar

\end{eulercomment}
\eulersubheading{Grafik Fungsi Linear}
\begin{eulercomment}
Fungsi linear dua variabel biasanya dinyatakan dalam bentuk\\
\end{eulercomment}
\begin{eulerformula}
\[
f(x,y)=ax+by+c
\]
\end{eulerformula}
\begin{eulerprompt}
>plot3d("x^3+2*y^2"):
\end{eulerprompt}
\eulerimg{27}{images/Nur Alya Fadilah_Aplikom-071.png}
\begin{eulerprompt}
>plot3d("x^2+3*y^2"):
\end{eulerprompt}
\eulerimg{27}{images/Nur Alya Fadilah_Aplikom-072.png}
\eulersubheading{Grafik Fungsi Kuadrat}
\begin{eulercomment}
Fungsi kuadrat dua variabel biasanya dinyatakan dalam bentuk\\
\end{eulercomment}
\begin{eulerformula}
\[
f(x,y)=ax^2+by^2+cxy+dx+ey+f
\]
\end{eulerformula}
\begin{eulerprompt}
>plot3d("2*x^2*y+2*y^2"):
\end{eulerprompt}
\eulerimg{27}{images/Nur Alya Fadilah_Aplikom-074.png}
\eulersubheading{Grafik Fungsi Akar Kuadrat}
\begin{eulerprompt}
>plot3d("sqrt(2*x^2+3*y^2)"):
\end{eulerprompt}
\eulerimg{27}{images/Nur Alya Fadilah_Aplikom-075.png}
\eulersubheading{Grafik Fungsi Trigonometri}
\begin{eulerprompt}
>plot3d("2*cos(x)*sin(y)"):
\end{eulerprompt}
\eulerimg{27}{images/Nur Alya Fadilah_Aplikom-076.png}
\begin{eulerprompt}
>aspect(2); plot3d("2*x^2+sin(y)",-6,6,0,6*pi):
\end{eulerprompt}
\eulerimg{13}{images/Nur Alya Fadilah_Aplikom-077.png}
\begin{eulerprompt}
>plot3d("10*2^(2*x*y)"):
\end{eulerprompt}
\eulerimg{13}{images/Nur Alya Fadilah_Aplikom-078.png}
\begin{eulerprompt}
>plot3d("-3^(x*y)"):
\end{eulerprompt}
\eulerimg{13}{images/Nur Alya Fadilah_Aplikom-079.png}
\eulersubheading{Grafik Fungsi Logaritma}
\begin{eulercomment}
Fungsi logaritma dua variabel bisa dinyatakan sebagai\\
\end{eulercomment}
\begin{eulerformula}
\[
f(x,y)=log_b(xy)
\]
\end{eulerformula}
\begin{eulerprompt}
>plot3d("log(2*x*y)"):
\end{eulerprompt}
\eulerimg{13}{images/Nur Alya Fadilah_Aplikom-081.png}
\begin{eulerprompt}
>plot3d("log(3*x*y)"):
\end{eulerprompt}
\eulerimg{13}{images/Nur Alya Fadilah_Aplikom-082.png}
\begin{eulerprompt}
> 
\end{eulerprompt}
\eulersubheading{Menggambar Grafik Fungsi Dua Variabel yang disimpan dalam variabel}
\begin{eulercomment}
Ekspresi

contoh:

ekspresi dalam string
\end{eulercomment}
\begin{eulerprompt}
>expr := "x^2+sin(y)"
\end{eulerprompt}
\begin{euleroutput}
  x^2+sin(y)
\end{euleroutput}
\begin{eulercomment}
plot ekspresi
\end{eulercomment}
\begin{eulerprompt}
>plot3d(expr,-5,5,0,6*pi):
\end{eulerprompt}
\eulerimg{13}{images/Nur Alya Fadilah_Aplikom-083.png}
\eulersubheading{Menggambar Grafik Fungsi Dua Variabel yang * Fungsinya}
\begin{eulercomment}
Didefinisikan sebagai Fungsi Numerik



\end{eulercomment}
\eulersubheading{Fungsi Numerik}
\begin{eulercomment}
Fungsi numerik adalah suatu fungsi matematika yang menghasilkan nilai
numerik sebagai output-nya. Fungsi ini dapat dinyatakan dalam bentuk
persamaan matematika atau algoritma komputasi.

Contoh:

Fungsi\\
\end{eulercomment}
\begin{eulerformula}
\[
f(x,y) = 5x+y
\]
\end{eulerformula}
\begin{eulercomment}
Misal input nilai x=2 dan y=3, maka akan dihasilkan nilai z yaitu

\end{eulercomment}
\begin{eulerformula}
\[
z = f(x,y) = 5(2)+3 = 10+3 = 13
\]
\end{eulerformula}
\begin{eulercomment}
\end{eulercomment}
\eulersubheading{Gambar Grafik Fungsi}
\begin{eulercomment}
Fungsi satu baris numerik didefinisikan oleh ":=".

Langkah-langkah untuk memvisualisasikan grafik fungsi dua variabel
yang fungsi nya didefinisikan sebagai fungsi numerik dalam plot3d:

1. Buat fungsi numerik yang akan digunakan untuk memvisualisasikan
data.\\
function f(x,y):=ax+by\\
dimana a dan b adalah konstanta

2. Gunakan fungsi plot3d() untuk membuat grafik tiga dimensi dari
fungsi numerik.\\
plot3d("f"):

\end{eulercomment}
\eulersubheading{Contoh}
\begin{eulercomment}
Fungsi matematika f(x,y) dapat digambarkan dalam bentuk grafik tiga
dimensi menggunakan perintah plot3d. Berikut adalah contoh penggunaan
perintah plot3d untuk menggambarkan fungsi tersebut:

1. Fungsi Linear Dua Variabel

\end{eulercomment}
\begin{eulerformula}
\[
f(x,y)=5x+3y+1
\]
\end{eulerformula}
\begin{eulerprompt}
>function f(x,y):= 5*x+3*y+1
>plot3d("f"):
\end{eulerprompt}
\eulerimg{13}{images/Nur Alya Fadilah_Aplikom-084.png}
\begin{eulercomment}
- Fungsi f(x,y) didefinisikan sebagai 5x+3y+1.\\
- Perintah "plot3d("f")" digunakan untuk memplot grafik 3D dari fungsi
f(x,y) menggunakan fungsi plot3d di EMT.\\
- Grafik yang dihasilkan akan menampilkan fungsi dalam tiga dimensi,
dengan sumbu x dan y mewakili variabel masukan dan sumbu z mewakili
nilai keluaran fungsi. Grafik akan menunjukkan bentuk fungsi dan
perubahannya seiring dengan perubahan variabel masukan.

\end{eulercomment}
\eulersubheading{}
\begin{eulercomment}
2. Fungsi Kuadrat Dua Variabel

\end{eulercomment}
\begin{eulerformula}
\[
f(x,y)=x^2+3y^2+21
\]
\end{eulerformula}
\begin{eulerprompt}
>function f(x,y):= x^2+(3*y)^2+27
>plot3d("f"):
\end{eulerprompt}
\eulerimg{13}{images/Nur Alya Fadilah_Aplikom-086.png}
\begin{eulercomment}
- Perintah "function f(x,y):= x\textasciicircum{}2+(3*y)\textasciicircum{}2+27" berarti mendefinisikan
fungsi matematika f(x,y) sebagai x pangkat 2 ditambah 3 kali y pangkat
2 ditambah 27.\\
- Perintah "plot3d("f")" berarti membuat grafik tiga dimensi dari
fungsi f(x,y) yang telah didefinisikan sebelumnya.

\end{eulercomment}
\eulersubheading{}
\begin{eulercomment}
3. Fungsi Logaritma Dua Variabel

\end{eulercomment}
\begin{eulerformula}
\[
f(x,y)= \log(2xy)
\]
\end{eulerformula}
\begin{eulerprompt}
>function f(x,y):= log((2*x)*y)
>plot3d("f"):
\end{eulerprompt}
\eulerimg{13}{images/Nur Alya Fadilah_Aplikom-088.png}
\begin{eulercomment}
- Input yang diberikan adalah fungsi matematika dua variabel, f(x,y),
yang didefinisikan sebagai logaritma hasil kali 2x dan y.\\
- Perintah "plot3d("f")" digunakan untuk memplot grafik fungsi f(x,y)
dalam ruang tiga dimensi.

\end{eulercomment}
\eulersubheading{}
\begin{eulercomment}
4. Fungsi Eksponen Dua Variabel

\end{eulercomment}
\begin{eulerformula}
\[
f(x,y)=x^{5y+10}
\]
\end{eulerformula}
\begin{eulerprompt}
>function f(x,y):= x^(5*y+10)
>plot3d("f"):
\end{eulerprompt}
\eulerimg{13}{images/Nur Alya Fadilah_Aplikom-090.png}
\begin{eulercomment}
- Perintah `fungsi f(x,y):= x\textasciicircum{}(5y+10)` adalah fungsi matematika dua
variabel `x` dan `y` dan dengan rumus x\textasciicircum{}(5y+10)\\
- Perintah `plot3d("f")` digunakan untuk memplot fungsi dalam ruang
tiga dimensi. Plot yang dihasilkan akan menampilkan nilai fungsi
sebagai permukaan pada bidang x-y, dengan tinggi permukaan mewakili
nilai fungsi pada titik tersebut.

\end{eulercomment}
\eulersubheading{}
\begin{eulercomment}
5. Fungsi Trigonometri Dua Variabel

\end{eulercomment}
\begin{eulerformula}
\[
f(x,y)=cos(x)+sin(y)
\]
\end{eulerformula}
\begin{eulerprompt}
>function f(x,y):= cos(x)*sin(y)
>plot3d("f"):
\end{eulerprompt}
\eulerimg{13}{images/Nur Alya Fadilah_Aplikom-091.png}
\begin{eulercomment}
- Perintah "function f(x,y):= cos(x)*sin(y)" adalah perintah untuk
mendefinisikan fungsi matematika f(x,y) yang menghasilkan nilai
cosinus dari x dikalikan dengan sinus dari y.\\
- Perintah "plot3d("f")" adalah perintah untuk membuat grafik tiga
dimensi dari fungsi f(x,y) yang telah didefinisikan sebelumnya.



\end{eulercomment}
\eulersubheading{Menggambar Grafik Fungsi Dua Variabel yang * Fungsinya}
\begin{eulercomment}
Didefinisikan sebagai Fungsi Simbolik

\end{eulercomment}
\eulersubheading{Fungsi Simbolik}
\begin{eulercomment}
Fungsi simbolik adalah fungsi yang dinyatakan dengan menggunakan
simbol-simbol matematika, seperti huruf dan operasi matematika,
daripada menggunakan angka konkret atau ekspresi numerik. Fungsi
simbolik sering digunakan untuk menggambarkan hubungan antara
variabel-variabel matematika dalam bentuk yang lebih umum dan abstrak.

Contoh fungsi simbolik yang umum adalah:

\end{eulercomment}
\begin{eulerformula}
\[
g(x,y) = 2x + y
\]
\end{eulerformula}
\begin{eulercomment}
Dalam contoh di atas, g(x) adalah fungsi simbolik yang mengaitkan
setiap nilai x dengan hasil dari ekspresi matematika 2x + 3. Fungsi
ini dapat digunakan untuk menghitung nilai fungsi untuk berbagai nilai
x.

\end{eulercomment}
\eulersubheading{Perbedaan Fungsi Numerik dan Fungsi Simbolik}
\begin{eulercomment}
1. Fungsi Numerik\\
Fungsi numerik dinyatakan dalam bentuk yang lebih konkret menggunakan
angka-angka nyata.

Contoh fungsi numerik adalah

\end{eulercomment}
\begin{eulerformula}
\[
g(x,y) = 2x + y + 3
\]
\end{eulerformula}
\begin{eulercomment}
dimana kita memberikan nilai numerik kepada "x dan y"\\
misalnya, x = 5 dan y = 2, maka hasilnya adalah angka konkret yaitu
g(5,2) = 15

2. Fungsi Simbolik\\
Fungsi simbolik dinyatakan menggunakan simbol-simbol matematika
seperti huruf (variabel) dan operasi matematika.

Contoh fungsi simbolik adalah

\end{eulercomment}
\begin{eulerformula}
\[
g(x,y) = 2x + y + 3
\]
\end{eulerformula}
\begin{eulercomment}
"g" adalah simbol fungsi\\
"x,y" adalah variabel,\\
2x + 3 adalah ekspresi matematika yang menggambarkan hubungan antara
"x,y" dan hasil fungsi.

\end{eulercomment}
\eulersubheading{Gambar Grafik Fungsi}
\begin{eulercomment}
Fungsi satu baris simbolik didefinisikan oleh "\&=".

Langkah-langkah untuk memvisualisasikan grafik fungsi dua variabel
yang fungsi nya didefinisikan sebagai fungsi simbolik dalam plot3d:

1. Buat fungsi simbolik yang akan digunakan untuk memvisualisasikan
data.\\
function g(x,y)\&= ax+by;\\
dimana a dan b adalah konstanta

2. Gunakan fungsi plot3d() untuk membuat grafik tiga dimensi dari
fungsi simbolik\\
plot3d("g"):

3. Menentukan rentang variabel\\
misal\\
plot3d("g(x,y)",-10,10,-5,5):\\
dengan batasan x dari -10 hingga 10 dan batasan y dari -5 hingga 5

\end{eulercomment}
\eulersubheading{Contoh}
\begin{eulercomment}
1. Fungsi Linear Dua Variabel

\end{eulercomment}
\begin{eulerformula}
\[
g(x,y)=x-2y+6
\]
\end{eulerformula}
\begin{eulercomment}
\end{eulercomment}
\begin{eulerprompt}
>function g(x,y)&= x-2*y+6;
>plot3d("g(x,y)"):
\end{eulerprompt}
\eulerimg{13}{images/Nur Alya Fadilah_Aplikom-092.png}
\begin{eulercomment}
- Fungsi g(x,y) adalah fungsi matematika yang mengambil dua variabel,
x dan y, dan menghasilkan sebuah nilai berdasarkan rumus x - 2y + 6.\\
- Perintah "plot3d" digunakan untuk menghasilkan grafik tiga dimensi
dari fungsi tersebut.
\end{eulercomment}
\begin{eulerprompt}
>plot3d("g(x,y)",-1,2,0,2*pi):
\end{eulerprompt}
\eulerimg{13}{images/Nur Alya Fadilah_Aplikom-093.png}
\begin{eulercomment}
- Perintah "plot3d("g(x,y)",-1,2,0,2*pi)" adalah perintah untuk
menggambar grafik fungsi tiga dimensi "g(x,y)" pada rentang x dari -1
hingga 2 dan rentang y dari 0 hingga 2pi.

\end{eulercomment}
\eulersubheading{}
\begin{eulercomment}
2. Fungsi Kuadrat Dua Variabel

\end{eulercomment}
\begin{eulerformula}
\[
g(x,y)=x^2+y^2+5
\]
\end{eulerformula}
\begin{eulerprompt}
>function g(x,y)&= x^2+y^2+5;
>plot3d("g(x,y)"):
\end{eulerprompt}
\eulerimg{13}{images/Nur Alya Fadilah_Aplikom-095.png}
\begin{eulercomment}
- Fungsi g(x,y) adalah fungsi matematika yang mengambil dua variabel,
x dan y, dan menghasilkan sebuah nilai berdasarkan rumus x\textasciicircum{}2+y\textasciicircum{}2+5\\
- Perintah "plot3d" digunakan untuk menghasilkan grafik tiga dimensi
dari fungsi tersebut.
\end{eulercomment}
\begin{eulerprompt}
>plot3d("g(x,y)",-10,10,-1,5):
\end{eulerprompt}
\eulerimg{13}{images/Nur Alya Fadilah_Aplikom-096.png}
\begin{eulercomment}
- Perintah "plot3d("g(x,y)",-10,10,-1,5)" adalah perintah untuk
menggambar grafik fungsi tiga dimensi g(x,y) pada rentang x dari -10
hingga 10 dan rentang y dari -1 hingga 5

\end{eulercomment}
\eulersubheading{}
\begin{eulercomment}
3. Fungsi Logaritma Dua Variabel

\end{eulercomment}
\begin{eulerformula}
\[
g(x,y) = \log(xy5)
\]
\end{eulerformula}
\begin{eulerprompt}
>function g(x,y)&= log(x*y*5);
>plot3d("g(x,y)"):
\end{eulerprompt}
\eulerimg{13}{images/Nur Alya Fadilah_Aplikom-098.png}
\begin{eulercomment}
- Fungsi g(x,y) adalah fungsi matematika yang mengambil dua variabel,
x dan y, dan menghasilkan sebuah nilai berdasarkan rumus logaritma x
dikalikan y dikalikan 5\\
- Perintah "plot3d" digunakan untuk menghasilkan grafik tiga dimensi
dari fungsi tersebut.
\end{eulercomment}
\begin{eulerprompt}
>plot3d("g(x,y)",-1,1,-5,5):
\end{eulerprompt}
\eulerimg{13}{images/Nur Alya Fadilah_Aplikom-099.png}
\begin{eulercomment}
- Perintah "plot3d("g(x,y)",-1,1,-5,5)" adalah perintah untuk
menggambar grafik fungsi tiga dimensi g(x,y) pada rentang x dari -1
hingga 1 dan rentang y dari -5 hingga 5
\end{eulercomment}
\begin{eulerprompt}
>plot3d("g(x,y)",-1,1,-5,5,zoom=4.5):
\end{eulerprompt}
\eulerimg{13}{images/Nur Alya Fadilah_Aplikom-100.png}
\begin{eulercomment}
- plot3d: perintah untuk membuat grafik 3D.\\
- "g(x,y)": fungsi matematika yang akan digunakan untuk membuat
grafik.\\
- -1,1: rentang nilai variabel x yang akan digunakan dalam grafik.\\
- -5,5: rentang nilai variabel y yang akan digunakan dalam grafik.\\
- zoom=4.5: perintah untuk memperbesar tampilan grafik dengan faktor
4.5.

\end{eulercomment}
\eulersubheading{}
\begin{eulercomment}
4. Fungsi Eksponen Dua Variabel

\end{eulercomment}
\begin{eulerformula}
\[
g(x,y) = 2^{xy5}
\]
\end{eulerformula}
\begin{eulerprompt}
>function g(x,y)&= 2^(x*y*5);
>plot3d("g(x,y)"):
\end{eulerprompt}
\eulerimg{13}{images/Nur Alya Fadilah_Aplikom-102.png}
\begin{eulercomment}
- Fungsi g(x,y) adalah fungsi matematika yang mengambil dua variabel,
x dan y, dan menghasilkan sebuah nilai berdasarkan rumus 2\textasciicircum{}(xy5)\\
- Perintah "plot3d" digunakan untuk menghasilkan grafik tiga dimensi
dari fungsi tersebut.
\end{eulercomment}
\begin{eulerprompt}
>plot3d("g(x,y)",-1,5,-1,3,frame=3,zoom=3):
\end{eulerprompt}
\eulerimg{13}{images/Nur Alya Fadilah_Aplikom-103.png}
\begin{eulercomment}
- Peintah plot3d("g(x,y)",-1,5,-1,3,frame=3,zoom=3) adalah perintah
untuk membuat plot tiga dimensi dari fungsi `g(x,y)` dengan batas `x`
dari `-1` hingga `5` dan batas `y` dari `-1` hingga `3`.

- plot3d: perintah untuk membuat plot tiga dimensi.\\
- "g(x,y)"`: fungsi yang akan diplot.\\
- (-1,5): batas `x` dari `-1` hingga `5`.\\
- (-1,3): batas `y` dari `-1` hingga `3`.\\
- frame=3: menampilkan frame nomor 3.\\
- zoom=3: memperbesar tampilan plot sebanyak 3 kali.

\end{eulercomment}
\eulersubheading{}
\begin{eulercomment}
5. Fungsi Trigonometri Dua Variabel

\end{eulercomment}
\begin{eulerformula}
\[
g(x,y)=tan(x) - cot(y)
\]
\end{eulerformula}
\begin{eulerprompt}
>function g(x,y)&= tan(x)-cos(y);
>plot3d("g(x,y)"):
\end{eulerprompt}
\eulerimg{13}{images/Nur Alya Fadilah_Aplikom-104.png}
\begin{eulercomment}
- Fungsi g(x,y) adalah fungsi matematika yang mengambil dua variabel,
x dan y, dan menghasilkan sebuah nilai berdasarkan rumus tan(x)-cos(y)\\
- Perintah "plot3d" digunakan untuk menghasilkan grafik tiga dimensi
dari fungsi tersebut.
\end{eulercomment}
\begin{eulerprompt}
>plot3d("g(x,y)",-1,3,0,2*pi,frame=1,zoom=3.5):
\end{eulerprompt}
\eulerimg{13}{images/Nur Alya Fadilah_Aplikom-105.png}
\begin{eulercomment}
- Peintah plot3d("g(x,y)",-1,3,0,2*pi,frame=5,zoom=3) adalah perintah
untuk membuat plot tiga dimensi dari fungsi `g(x,y)` dengan batas `x`
dari `-1` hingga `3` dan batas `y` dari `0` hingga `2pi`.

- plot3d: perintah untuk membuat plot tiga dimensi.\\
- "g(x,y)"`: fungsi yang akan diplot.\\
- (-1,3): batas `x` dari `-1` hingga `3`.\\
- (0,2pi): batas `y` dari `0` hingga `2pi`.\\
- frame=1: menampilkan frame nomor 1.\\
- zoom=3.5: memperbesar tampilan plot sebanyak 3.5 kali.

\end{eulercomment}
\eulersubheading{}
\begin{eulercomment}
6. Fungsi Akar Kuadrat

\end{eulercomment}
\begin{eulerformula}
\[
P(x,y)= \sqrt{10x^2+2y^2}
\]
\end{eulerformula}
\begin{eulerprompt}
>function P(x,y) &= sqrt(10*x^2+2*y^2);
>plot3d("P(x,y)"):
\end{eulerprompt}
\eulerimg{13}{images/Nur Alya Fadilah_Aplikom-107.png}
\begin{eulercomment}
- Fungsi P(x,y) adalah fungsi matematika yang mengambil dua variabel,
x dan y, dan menghasilkan sebuah nilai berdasarkan rumus akar kuadrat
dari 10x\textasciicircum{}2+2y\textasciicircum{}2\\
- Perintah "plot3d" digunakan untuk menghasilkan grafik tiga dimensi
dari fungsi tersebut.
\end{eulercomment}
\begin{eulerprompt}
>plot3d("P(x,y)",-2,2,0,3*pi,frame=5,zoom=2,scale=1):
\end{eulerprompt}
\eulerimg{13}{images/Nur Alya Fadilah_Aplikom-108.png}
\begin{eulercomment}
- P(x,y): Merupakan fungsi yang akan digambarkan dalam grafik tiga
dimensi.\\
- (-2,2): Merupakan rentang nilai dari sumbu x yang akan digunakan
dalam grafik.\\
- (0,3pi): Merupakan rentang nilai dari sumbu y yang akan digunakan
dalam grafik. Nilai pi dikalikan dengan 3 agar rentang nilai y
mencakup tiga putaran lingkaran penuh.\\
- frame=5: Menentukan nomor bingkai (frame) yang akan digunakan dalam
animasi grafik.\\
- zoom=2: Menentukan faktor pembesaran grafik. Dengan memperbesar
tampilan, kita dapat melihat detail yang lebih kecil pada plot.\\
- scale=1: Menentukan skala grafik. Dengan mengatur skala, kita dapat
mengubah jarak antara titik-titik pada sumbu tersebut.
\end{eulercomment}
\eulersubheading{Menggambar Data $x$, $y$, $z$ * pada ruang Tiga Dimensi (3D)}
\begin{eulercomment}
Definisi

\end{eulercomment}
\begin{eulerttcomment}
  Menggambar data pada ruang tiga dimensi (3D) adalah proses
\end{eulerttcomment}
\begin{eulercomment}
visualisasi data yang mengubah informasi dalam tiga dimensi, yaitu
panjang, lebar, dan tinggi, menjadi representasi visual yang dapat
dipahami dan dianalisis.

Tujuan:

\end{eulercomment}
\begin{eulerttcomment}
  Tujuan dari menggambar data 3D adalah untuk membantu pemahaman dan
\end{eulerttcomment}
\begin{eulercomment}
interpretasi data yang lebih baik, terutama ketika data tersebut
memiliki komponen yang tidak dapat direpresentasikan dengan baik dalam
dua dimensi.

Sama seperti plot2d, plot3d menerima data. Untuk objek 3D, Anda perlu
menyediakan matriks nilai x-, y- dan z, atau tiga fungsi atau ekspresi
fx(x,y), fy(x,y), fz(x,y).

\end{eulercomment}
\begin{eulerformula}
\[
\gamma(t,s) = (x(t,s),y(t,s),z(t,s))
\]
\end{eulerformula}
\begin{eulercomment}
Karena x,y,z adalah matriks, kita asumsikan bahwa (t,s) melalui sebuah
kotak persegi. Hasilnya, Anda dapat memplot gambar persegi panjang di
ruang angkasa.

Kita dapat menggunakan bahasa matriks Euler untuk menghasilkan
koordinat secara efektif.

Dalam contoh berikut, kami menggunakan vektor nilai t dan vektor kolom
nilai s untuk membuat parameter permukaan bola. Dalam gambar kita
dapat menandai daerah, dalam kasus kita daerah kutub.

\end{eulercomment}
\eulersubheading{Contoh 1}
\begin{eulerprompt}
>t=-1:0.1:1; s=(-1:0.1:1)'; plot3d(t,s,t*s,grid=10):
\end{eulerprompt}
\eulerimg{13}{images/Nur Alya Fadilah_Aplikom-110.png}
\begin{eulercomment}
Baris pertama kode "t=-1:0.1:1" membuat vektor baris t yang berisi
nilai dari -1 hingga 1 dengan interval 0.1. Baris kedua
"s=(-1:0.1:1)'" membuat vektor kolom s yang berisi nilai dari -1
hingga 1 dengan interval 0.1. Operator transpose ' digunakan untuk
mengubah vektor baris t menjadi vektor kolom.\\
Baris ketiga "plot3d(t,s,ts,grid=10)" membuat plot tiga dimensi dari
fungsi f(x,y) = xy pada domain [-1,1] x [-1,1]. Plot dibuat
menggunakan fungsi plot3d, yang mengambil tiga argumen: koordinat x,
y, dan z dari titik-titik yang akan diplot. Dalam hal ini, koordinat x
diberikan oleh vektor t, koordinat y diberikan oleh vektor s, dan
koordinat z diberikan oleh hasil perkalian t dan s, yaitu ts.
Parameter grid diatur menjadi 10, yang menunjukkan jumlah garis grid
yang akan ditampilkan pada plot.

\end{eulercomment}
\eulersubheading{Contoh 2}
\begin{eulercomment}
Tentu saja, titik cloud juga dimungkinkan. Untuk memplot data titik
dalam ruang, kita membutuhkan tiga vektor untuk koordinat titik-titik
tersebut.

Gayanya sama seperti di plot2d dengan points=true;
\end{eulercomment}
\begin{eulerprompt}
>n=500;...
>plot3d(normal(1,n),normal(1,n),normal(1,n),points=true,style="."):
\end{eulerprompt}
\eulerimg{13}{images/Nur Alya Fadilah_Aplikom-111.png}
\begin{eulercomment}
Kode "n=500;
plot3d(normal(1,n),normal(1,n),normal(1,n),points=true,style=".")"
digunakan untuk membuat plot tiga dimensi dari tiga vektor normal yang
dihasilkan secara acak dengan menggunakan fungsi "normal()" pada Euler
Math Toolbox (EMT). Parameter "n=500" menunjukkan bahwa setiap vektor
normal memiliki 500 elemen. Parameter "points=true" digunakan untuk
menampilkan titik-titik pada plot, sedangkan parameter "style='.'"
digunakan untuk mengatur gaya titik pada plot menjadi titik bulat.

\end{eulercomment}
\eulersubheading{Contoh 3}
\begin{eulercomment}
\end{eulercomment}
\begin{eulerttcomment}
 Dengan lebih banyak usaha, kami dapat menghasilkan banyak permukaan.
\end{eulerttcomment}
\begin{eulercomment}
\end{eulercomment}
\begin{eulerttcomment}
 Dalam contoh berikut, kita membuat tampilan bayangan dari bola yang
\end{eulerttcomment}
\begin{eulercomment}
terdistorsi. Koordinat biasa untuk bola adalah

\end{eulercomment}
\begin{eulerformula}
\[
\gamma(t,s) = (\cos(t)\cos(s),\sin(t)\sin(s),\cos(s))
\]
\end{eulerformula}
\begin{eulercomment}
dengan

\end{eulercomment}
\begin{eulerformula}
\[
0 \le t \le 2\pi, \quad \frac{-\pi}{2} \le s \le \frac{\pi}{2}.
\]
\end{eulerformula}
\begin{eulercomment}
Kami mendistorsi ini dengan sebuah faktor

\end{eulercomment}
\begin{eulerformula}
\[
d(t,s) = \frac{\cos(4t)+\cos(8s)}{4}
\]
\end{eulerformula}
\begin{eulerprompt}
>t=linspace(0,2pi,320); s=linspace(-pi/2,pi/2,160)';...
>d=1+0.2*(cos(4*t)+cos(8*s));...
>plot3d(cos(t)*cos(s)*d,sin(t)*cos(s)*d,sin(s)*d,hue=1,...
>light=[1,0,1],frame=0,zoom=5):
\end{eulerprompt}
\eulerimg{13}{images/Nur Alya Fadilah_Aplikom-115.png}
\begin{eulercomment}
Kode ini terdiri dari beberapa baris. Baris pertama
"t=linspace(0,2pi,320)" membuat vektor t yang berisi 320 nilai yang
sama terdistribusi secara merata antara 0 dan 2p. Baris kedua
"s=linspace(-pi/2,pi/2,160)'" membuat vektor s yang berisi 160 nilai
yang sama terdistribusi secara merata antara -p/2 dan p/2. Operator
transpose ' digunakan untuk mengubah vektor baris s menjadi vektor
kolom.

Baris ketiga "d=1+0.2*(cos(4t)+cos(8s))" membuat vektor d yang berisi
nilai dari 1 + 0.2 * (cos(4t) + cos(8s)). Baris keempat
"plot3d(cos(t)*cos(s)*d,sin(t)*cos(s)*d,sin(s)*d,hue=1,light=[1,0,1],frame=0,zoom=5)"
membuat plot tiga dimensi dari fungsi f(x,y) = 2x\textasciicircum{}2 + y\textasciicircum{}3. Plot dibuat
menggunakan fungsi plot3d, yang mengambil empat argumen: koordinat x,
y, dan z dari titik-titik yang akan diplot, serta beberapa parameter
lainnya. Dalam hal ini, koordinat x diberikan oleh ekspresi
cos(t)*cos(s)*d, koordinat y diberikan oleh ekspresi sin(t)*cos(s)*d,
dan koordinat z diberikan oleh ekspresi sin(s)*d. Parameter "hue=1"
digunakan untuk mengatur warna pada plot berdasarkan nilai fungsinya.
Parameter "light=[1,0,1]" digunakan untuk mengatur pencahayaan pada
plot. Parameter "frame=0" digunakan untuk menghilangkan frame pada
plot. Parameter "zoom=5" digunakan untuk mengatur level zoom pada
plot.
\end{eulercomment}
\eulersubheading{Grafik Tiga Dimensi yang * Bersifat Interaktif dan animasi grafik}
\begin{eulercomment}
3D
\end{eulercomment}
\begin{eulercomment}
Membuat gambar grafik tiga dimensi (3D) yang bersifat interaktif dan
animasi grafik 3D adalah proses menciptakan visualisasi tiga dimensi
yang memungkinkan pengguna berinteraksi dengan objek-objek 3D.
Interaktivitas dalam gambar 3D memungkinkan pengguna untuk melakukan
tindakan seperti mengubah sudut pandang, memindahkan objek, atau
berinteraksi dengan elemen-elemen dalam adegan 3D. Animasi grafik 3D
dapat mencakup pergerakan, tetapi juga dapat berarti perubahan dalam
tampilan atau atribut objek tanpa pergerakan fisik yang mencolok.

CONTOH GAMBAR
\end{eulercomment}
\begin{eulerprompt}
>function testplot () := plot3d("x^2+y^3"); ...
>rotate("testplot"); testplot(): 
\end{eulerprompt}
\eulerimg{13}{images/Nur Alya Fadilah_Aplikom-116.png}
\begin{eulerprompt}
>function testplot () := plot3d("x^2+y",distance=3,zoom=1,angle=pi/2,height=0); ...
>rotate("testplot"); testplot(): 
\end{eulerprompt}
\eulerimg{13}{images/Nur Alya Fadilah_Aplikom-117.png}
\begin{eulercomment}
Hilangkan command angle untuk bisa merotasikan grafik,dan height = 0
untuk membuat posisi sejajar dengan mata jadi tidak mempengaruhi
pergerakan hanya berbeda sudut pandang saja
\end{eulercomment}
\begin{eulerprompt}
>plot3d("exp(-x^2+y^2)",>user, ...
>  title="Turn with the vector keys (press return to finish)"):
\end{eulerprompt}
\eulerimg{13}{images/Nur Alya Fadilah_Aplikom-118.png}
\begin{eulerprompt}
>plot3d("exp(x^2+y^2)",>user, ...
>title="Coba gerakan)")
\end{eulerprompt}
\begin{eulercomment}
Interaksi pengguna dimungkinkan dengan parameter. Pengguna dapat
menekan tombol berikut.\\
1. kiri, kanan, atas, bawah: memutar sudut pandang\\
2. +,-: memperbesar atau memperkecil\\
3. a: menghasilkan anaglyph (lihat di bawah)\\
4. l: beralih memutar sumber cahaya (lihat di bawah)\\
5. spasi: disetel ulang ke default\\
6. enter: akhiri interaksi
\end{eulercomment}
\begin{eulerprompt}
>plot3d("exp(-(x^2+y^2)/5)",r=10,n=80,fscale=4,scale=1.2,frame=3,>user):
\end{eulerprompt}
\eulerimg{13}{images/Nur Alya Fadilah_Aplikom-119.png}
\begin{eulercomment}
Parameter "r=10" menunjukkan jari-jari bola yang digunakan untuk
membuat plot tiga dimensi. Dalam hal ini, jari-jari bola yang
digunakan adalah 10.\\
Parameter "n=80" menunjukkan jumlah titik yang digunakan untuk membuat
plot. Semakin besar nilai n, semakin banyak titik yang digunakan untuk
membuat plot, sehingga plot akan menjadi lebih halus dan akurat.\\
Parameter "fscale=4" menunjukkan faktor skala pada sumbu z. Dalam hal
ini, faktor skala pada sumbu z adalah 4.\\
Parameter "scale=1.2" menunjukkan faktor skala pada plot. Semakin
besar nilai scale, semakin besar ukuran plot yang dihasilkan.\\
Parameter "frame=3" menunjukkan jenis frame yang digunakan pada plot.
Dalam hal ini, jenis frame yang digunakan adalah frame kotak dengan
sumbu x, y, dan z yang ditampilkan.
\end{eulercomment}
\begin{eulerprompt}
>plot3d("x^2+y",distance=3,zoom=1,angle=pi/2,height=0):
\end{eulerprompt}
\eulerimg{13}{images/Nur Alya Fadilah_Aplikom-120.png}
\begin{eulercomment}
Tampilan dapat diubah dengan berbagai cara.

- distance: jarak pandang ke plot.\\
- zoom: nilai zoom.\\
- angle: sudut terhadap sumbu y negatif dalam radian.\\
- height: ketinggian tampilan dalam radian.
\end{eulercomment}
\begin{eulerprompt}
>plot3d("x^4+y^2",a=0,b=1,c=-1,d=1, angle=-20, height=20, ...
>  center=[0.4,0,0], zoom=5):
\end{eulerprompt}
\eulerimg{13}{images/Nur Alya Fadilah_Aplikom-121.png}
\begin{eulercomment}
Plot selalu terlihat berada di tengah kubus plot. Anda dapat
memindahkan bagian tengah dengan parameter center.

Parameter center digunakan untuk memindahkan pusat plot ke lokasi
tertentu dalam ruang. Dalam hal ini, pusat plot diatur ke titik (0.4,
0, 0) dalam ruang tiga dimensi. Parameter center berguna ketika kita
ingin mengubah sudut pandang plot atau ketika kita ingin menyelaraskan
plot dengan objek lain dalam scene. Dengan menentukan pusat plot, kita
dapat mengontrol posisi kamera dan arah tampilan plot.

Ada beberapa parameter untuk menskalakan fungsi atau mengubah tampilan
grafik.

fscale: menskalakan ke nilai fungsi (defaultnya adalah \textless{}fscale).\\
scale: angka atau vektor 1x2 untuk diskalakan ke arah x dan y.\\
frame: jenis bingkai (default 1).
\end{eulercomment}
\begin{eulerprompt}
>function testplot () := plot3d("5*exp(-x^2-y^2)",r=2,<fscale,<scale,distance=13,height=50, ...
>center=[0,0,-2],frame=3); ...
>rotate("testplot"); testplot():
\end{eulerprompt}
\eulerimg{13}{images/Nur Alya Fadilah_Aplikom-122.png}
\begin{eulerprompt}
>plot3d("x^2+1",a=-1,b=1,rotate=true,grid=5):
\end{eulerprompt}
\eulerimg{13}{images/Nur Alya Fadilah_Aplikom-123.png}
\begin{eulercomment}
Penjelasan:\\
Secara umum, parameter "a" dan "b" digunakan untuk menentukan rentang
nilai variabel independen dalam suatu fungsi. Dalam kasus ini, "a=-1"
dan "b=1" menunjukkan bahwa fungsi tersebut akan diplot pada interval
[-1, 1]. Parameter "rotate=true" menunjukkan bahwa grafik akan diputar
untuk memberikan tampilan bentuk tiga dimensi yang lebih baik.
Parameter "grid=5" menunjukkan bahwa grid dengan jarak 5 unit akan
ditampilkan pada grafik.

Parameter memutar memutar fungsi dalam x di sekitar sumbu x.

- rotate=1: Menggunakan sumbu x\\
- rotate=2: Menggunakan sumbu z
\end{eulercomment}
\begin{eulerprompt}
>plot3d("x^2+1",a=-1,b=1,rotate=2,grid=5):
\end{eulerprompt}
\eulerimg{13}{images/Nur Alya Fadilah_Aplikom-124.png}
\begin{eulerprompt}
>function testplot () := plot3d("sqrt(25-x^2)",a=0,b=5,rotate=1); ...
>rotate("testplot"); testplot():
\end{eulerprompt}
\eulerimg{13}{images/Nur Alya Fadilah_Aplikom-125.png}
\begin{eulerprompt}
>function testplot () := plot3d("x^4+y^2",a=0,b=1,c=-1,d=1,height=20, ...
>center=[0.4,0,0],zoom=5); ...
>rotate("testplot"); testplot():
\end{eulerprompt}
\eulerimg{13}{images/Nur Alya Fadilah_Aplikom-126.png}
\begin{eulerprompt}
>function testplot () := plot3d("1/(x^2+y^2+1)",r=5,>polar, ...
>fscale=2,>hue,n=100,zoom=4,>contour,color=red); ...
>rotate("testplot"); testplot():
\end{eulerprompt}
\eulerimg{13}{images/Nur Alya Fadilah_Aplikom-127.png}
\begin{eulercomment}
Parameter "r=5" menunjukkan jari-jari bola yang digunakan untuk
membuat plot tiga dimensi. Dalam hal ini, jari-jari bola yang
digunakan adalah 5.\\
Parameter "\textgreater{}polar" menunjukkan bahwa plot yang dibuat adalah plot
polar tiga dimensi. Plot polar adalah plot yang dibuat dengan
menggunakan koordinat polar, yaitu koordinat yang terdiri dari jarak
dan sudut.\\
Parameter "fscale=2" menunjukkan faktor skala pada sumbu z. Dalam hal
ini, faktor skala pada sumbu z adalah 2.\\
Parameter "\textgreater{}hue" menunjukkan bahwa warna pada plot akan diatur
berdasarkan nilai fungsinya. Semakin tinggi nilai fungsinya, semakin
terang warnanya.\\
Parameter "n=100" menunjukkan jumlah titik yang digunakan untuk
membuat plot. Semakin besar nilai n, semakin banyak titik yang
digunakan untuk membuat plot, sehingga plot akan menjadi lebih halus
dan akurat.\\
Parameter "zoom=4" menunjukkan level zoom pada plot.\\
Parameter "\textgreater{}contour" menunjukkan bahwa garis kontur akan ditampilkan
pada plot.\\
Parameter "color=blue" menunjukkan warna garis kontur pada plot. Dalam
hal ini, warna yang digunakan adalah biru.

Untuk plotnya, Euler menambahkan garis grid. Sebaliknya dimungkinkan
untuk menggunakan garis level dan satu warna atau warna spektral.
Euler dapat menggambar ketinggian fungsi pada sebuah plot dengan
bayangan. Di semua plot 3D, Euler dapat menghasilkan anaglyph
merah/cyan.

-hue: Mengaktifkan bayangan cahaya, bukan kabel.\\
-contour: Membuat plot garis kontur otomatis pada plot.\\
-level=... (atau level): Vektor nilai garis kontur.
\end{eulercomment}
\begin{eulerprompt}
>function testplot () := plot3d("x^2-y^2",0,5,0,5,level=-1:0.1:1,color=blue); ...
>rotate("testplot"); testplot():
\end{eulerprompt}
\eulerimg{13}{images/Nur Alya Fadilah_Aplikom-128.png}
\begin{eulercomment}
Parameter "level=-1:0.1:1" menunjukkan rentang nilai fungsinya yang
akan ditampilkan pada plot. Dalam hal ini, rentang nilai fungsinya
adalah dari -1 hingga 1 dengan interval 0.1.
\end{eulercomment}
\begin{eulerprompt}
>function testplot () := plot3d("x^2+y^4",>cp,cpcolor=green,cpdelta=0.2); ...
>rotate("testplot"); testplot():
\end{eulerprompt}
\eulerimg{13}{images/Nur Alya Fadilah_Aplikom-129.png}
\begin{eulercomment}
Parameter "\textgreater{}cp" menunjukkan bahwa titik kontrol akan ditambahkan pada
plot. Titik kontrol digunakan untuk menentukan bentuk dan posisi plot
tiga dimensi.\\
Parameter "cpcolor=green" menunjukkan warna titik kontrol yang akan
digunakan. Dalam hal ini, warna yang digunakan adalah hijau.\\
Parameter "cpdelta=0.2" menunjukkan jarak antara titik kontrol.
Semakin kecil nilai cpdelta, semakin banyak titik kontrol yang akan
ditambahkan pada plot.
\end{eulercomment}
\begin{eulerprompt}
>plot3d("-x^2-y^2", ...
>hue=true,light=[0,1,1],amb=0,user=true, ...
> title="Press l and cursor keys (return to exit)"):
\end{eulerprompt}
\eulerimg{13}{images/Nur Alya Fadilah_Aplikom-130.png}
\begin{eulercomment}
parameter "hue=true" menunjukkan bahwa warna pada plot akan diatur
berdasarkan nilai fungsinya. Semakin tinggi nilai fungsinya, semakin
terang warnanya.\\
Parameter "light=light=[0,1,1] menunjukkan intensitas cahaya pada
plot. Nilai light=[0,1,1] menunjukkan bahwa cahaya datang dari arah
positif y dan z.\\
Parameter "amb=0" menunjukkan intensitas cahaya ambient pada plot.
Nilai 0 menunjukkan bahwa tidak ada cahaya ambient yang digunakan.
\end{eulercomment}
\begin{eulerprompt}
>function testplot () := plot3d("-x^2-y^2",color=rgb(0.2,0.2,0),hue=true,frame=false, ...
> zoom=3,contourcolor=red,level=-2:0.1:1,dl=0.01); ...
>rotate("testplot"); testplot():
\end{eulerprompt}
\eulerimg{13}{images/Nur Alya Fadilah_Aplikom-131.png}
\begin{eulercomment}
Parameter "frame=false" digunakan untuk menghilangkan frame pada plot
tiga dimensi. Parameter "color=rgb(0.2,0.2,0)" menunjukkan warna dasar
plot. Dalam hal ini, warna yang digunakan adalah hitam dengan nilai
RGB (0.2, 0.2, 0). Parameter "dl=0.01" menunjukkan jarak antara
titik-titik pada plot. Semakin kecil nilai dl, semakin banyak titik
yang digunakan untuk membuat plot, sehingga plot akan menjadi lebih
halus dan akurat. Namun, semakin kecil nilai dl, semakin lama waktu
yang dibutuhkan untuk membuat plot.
\end{eulercomment}
\begin{eulerprompt}
>function testplot () := plot3d("x^2+y^3",>contour,>spectral); ...
>rotate("testplot"); testplot():
\end{eulerprompt}
\eulerimg{13}{images/Nur Alya Fadilah_Aplikom-132.png}
\begin{eulerprompt}
>function testplot () := plot3d("x^2+y^3", >transparent, grid=10, wirecolor=red); ...
>rotate("testplot"); testplot():
\end{eulerprompt}
\eulerimg{13}{images/Nur Alya Fadilah_Aplikom-133.png}
\eulersubheading{Fungsi Parametrik 3D}
\begin{eulercomment}
Fungsi parametrik merupakan jenis fungsi matematika yang menggambarkan
hubungan antara dua atau lebih variabel, dimana masing-masing
koordinat (x, y, z...) dinyatakan sebagai fungsi lain dari beberapa
parameter. Fungsi parametrik dapat digunakan untuk menggambar kurva,
lintasan, atau hubungan antara berbagai variabel yang bergantung pada
parameter-parameter tertentu.

Sebagai contoh :
\end{eulercomment}
\begin{eulerprompt}
>plot3d("cos(x)*cos(y)^3","sin(x)*cos(y)^3","sin(y)", a=0,b=2*pi,c=pi/2,d=-pi/2,...
>>hue,color=blue,light=[0,1,3],<frame,...
>n=90,grid=[20,50],wirecolor=black,zoom=5):
\end{eulerprompt}
\eulerimg{13}{images/Nur Alya Fadilah_Aplikom-134.png}
\begin{eulerprompt}
>plot3d("cos(x)*cos(y)","sin(x)*cos(y)","cos(x)", a=0,b=2*pi,c=pi/2,d=-pi/2,...
>>hue,color=blue,light=[0,1,3],<frame,...
>n=90,grid=[20,50],wirecolor=black,zoom=5):
\end{eulerprompt}
\eulerimg{13}{images/Nur Alya Fadilah_Aplikom-135.png}
\begin{eulerprompt}
>plot3d("cos(x)^3*sin(y)","sin(x)^2*sin(y)","cos(x)^2", a=0,b=2*pi,c=pi/2,d=-pi/2,...
>>hue,color=blue,light=[0,1,5],<frame,...
>n=90,grid=[20,50],wirecolor=black,zoom=5):
\end{eulerprompt}
\eulerimg{13}{images/Nur Alya Fadilah_Aplikom-136.png}
\begin{eulerprompt}
>plot3d("cos(x)^2*cos(y)","sin(x)^2*cos(y)","cos(x)^2", a=0,b=2*pi,c=pi/2,d=-pi/2,...
>>hue,color=blue,light=[0,1,5],<frame,...
>n=90,grid=[10,50],wirecolor=black,zoom=5):
\end{eulerprompt}
\eulerimg{13}{images/Nur Alya Fadilah_Aplikom-137.png}
\begin{eulerprompt}
>plot3d("cos(x)*cos(y)","sin(x)*cos(y)","sin(y)", a=0,b=2*pi,c=pi/2,d=-pi/2,...
>>hue,color=blue,light=[0,1,3],<frame,...
>n=90,grid=[20,50],wirecolor=black,zoom=5):
\end{eulerprompt}
\eulerimg{13}{images/Nur Alya Fadilah_Aplikom-138.png}
\eulersubheading{Menggambar Fungsi Implisit Implisit}
\begin{eulercomment}
Fungsi implisit (implicit function) adalah fungsi yang memuat lebih
dari satu variabel, berjenis variabel bebas dan variabel terikat yang
berada dalam satu ruas sehingga tidak bisa dipisahkan pada ruas yang
berbeda.

\end{eulercomment}
\begin{eulerformula}
\[
F(x,y,z)=0
\]
\end{eulerformula}
\begin{eulercomment}
(1 persamaan dan 3 variabel), terdiri dari 2 variabel bebas dan 1
terikat

Misalnya,\\
\end{eulercomment}
\begin{eulerformula}
\[
F(x, y, z) = x^2 + y^2 + z^2 = 1
\]
\end{eulerformula}
\begin{eulercomment}
adalah persamaan implisit yang menggambarkan bola dengan jari-jari 1
dan pusat di (0,0,0).

\end{eulercomment}
\begin{eulerprompt}
>plot3d("x^2+y^3+z*y-1", r=5, implicit=3):
\end{eulerprompt}
\eulerimg{13}{images/Nur Alya Fadilah_Aplikom-141.png}
\begin{eulerprompt}
>c=1; d=1;
>plot3d("((x^2+y^2-c^2)^2+(z^2-1)^2)*((y^2+z^2-c^2)^2+(x^2-1)^2)*((z^2+x^2-c^2)^2+(y^2-1)^2)-d", r=2, <frame,>implicit,>user):
\end{eulerprompt}
\eulerimg{13}{images/Nur Alya Fadilah_Aplikom-142.png}
\begin{eulerprompt}
>plot3d("x^2+y^2+4*x*z+z^3",>implicit, r=2, zoom=2.5):
\end{eulerprompt}
\eulerimg{13}{images/Nur Alya Fadilah_Aplikom-143.png}
\begin{eulercomment}
Selain plot kontur yang sudah di jelaskan sebelumnya, pada EMT juga
ada plot umplisit dalam tiga dimensi. Euler menghasilkan potongan
melalui objek. Fitur plot3d termasuk plot implisit. Plot-plot ini
menunjukkan himpunan nol dari sebuah fungsi dalam tiga variabel.

Solusi dari\\
\end{eulercomment}
\begin{eulerformula}
\[
f(x,y,z) = 0
\]
\end{eulerformula}
\begin{eulercomment}
dapat divisualisasikan dalam potongan yang sejajar dengan bidang x-y,
bidang x-z, dan bidang y-z.

- implisit = 1: potong sejajar dengan bidang-y-z\\
- implicit = 2: memotong sejajar dengan bidang x-z\\
- implicit=4: memotong sejajar dengan bidang x-y

Ambil contoh dari persamaan latex pada fungsi implisit tadi dan
tambahkan nilai-nilai ini, sehingga kita dapat memplot persamaan ini\\
\end{eulercomment}
\begin{eulerformula}
\[
M = {(x,y,z) :{ x^2+y^3+zy=1}}
\]
\end{eulerformula}
\begin{eulerprompt}
>plot3d("x^2+y^3+z*y", r=1, implicit=2):
\end{eulerprompt}
\eulerimg{13}{images/Nur Alya Fadilah_Aplikom-146.png}
\begin{eulercomment}
Contoh fungsi implisit yang lainnya
\end{eulercomment}
\begin{eulerprompt}
>plot3d("x^3+y^3+z*y-1",r=7,implicit=4):
\end{eulerprompt}
\eulerimg{13}{images/Nur Alya Fadilah_Aplikom-147.png}
\begin{eulerprompt}
>plot3d("2*x^2 + 3*y^2 + z^2 - 25",r=8,implicit=2):
\end{eulerprompt}
\eulerimg{13}{images/Nur Alya Fadilah_Aplikom-148.png}
\begin{eulerprompt}
>plot3d("x^5 + 5*y^3 + 3*z^2 - 5*x - 7*y - 5*z + 10",r=5,implicit=2):
\end{eulerprompt}
\eulerimg{13}{images/Nur Alya Fadilah_Aplikom-149.png}
\begin{eulerprompt}
>plot3d("x^3+y^5+5*x*z+z^3",>implicit,r=3,zoom=2):
\end{eulerprompt}
\eulerimg{13}{images/Nur Alya Fadilah_Aplikom-150.png}
\begin{eulerprompt}
>plot3d("x^2+y^2+4*x*z+z^3-5",>implicit,r=2,zoom=2.5):
\end{eulerprompt}
\eulerimg{13}{images/Nur Alya Fadilah_Aplikom-151.png}
\eulersubheading{Menggambar Titik pada Ruang Tiga Dimensi}
\begin{eulercomment}
Untuk menggambar titik pada ruang tiga dimensi kita memerlukan tiga
vektor untuk koordinat titik serta menambahkan parameter points=true.

\end{eulercomment}
\begin{eulerprompt}
>n=510; ...
>plot3d(normal(1,n),normal(1,n),normal(1,n),points=true,style="."):
\end{eulerprompt}
\eulerimg{13}{images/Nur Alya Fadilah_Aplikom-152.png}
\begin{eulercomment}
\textgreater{}n=510; ...\\
\textgreater{}plot3d(normal(1,n),normal(1,n),normal(1,n),points=true,style="."):

- plot3d() untuk menjalankan perintah membuat plot 3D.\\
- normal(1,n) sebagai titik koordinat yang akan diplot pada sumbu
x,y,z dengan nilai angka dstribusi normal yang dicetak secara random
sebanyak n sehingga membentuk matriks 1xn atau 1x500.\\
- points=true sebagai parameter yang memerintahkan plot3d akan
menampilkan points(titik-titik).

\end{eulercomment}
\begin{eulerprompt}
>n=30; ...
>plot3d(normal(1,n),normal(1,n),normal(1,n),points=true,style="*"):
\end{eulerprompt}
\eulerimg{13}{images/Nur Alya Fadilah_Aplikom-153.png}
\begin{eulercomment}
1. n = 30; Ini adalah pernyataan untuk menginisialisasi variabel n
dengan nilai 30. Variabel ini kemungkinan akan digunakan untuk
menentukan jumlah titik yang akan digunakan dalam plot.

2. normal(1, n): Fungsi normal digunakan untuk menghasilkan
nilai-nilai acak yang terdistribusi secara normal (disebut juga
Gaussian) dengan rata-rata 1 dan deviasi standar 1. Ini berarti Anda
akan mendapatkan n nilai acak yang terdistribusi secara normal dengan
rata-rata 1 dan deviasi standar 1. Anda melakukan ini untuk
mendapatkan koordinat x, y, dan z untuk plot 3D.

3. plot3d(...): Ini adalah fungsi yang digunakan untuk membuat plot 3D
dengan parameter-parameter berikut:

\end{eulercomment}
\begin{eulerttcomment}
 -normal(1, n): Ini adalah data koordinat x, y, dan z yang telah
\end{eulerttcomment}
\begin{eulercomment}
dihasilkan sebelumnya.\\
\end{eulercomment}
\begin{eulerttcomment}
 -points = true: Ini mengatur agar titik-titik data ditampilkan dalam
\end{eulerttcomment}
\begin{eulercomment}
plot.\\
\end{eulercomment}
\begin{eulerttcomment}
 -style = "*": Ini mengatur gaya plot menjadi tanda bintang (*).
\end{eulerttcomment}
\begin{eulerprompt}
>x=[1,2,3,4]; y=[4,5,6,1]; z=[6,1,2,3];
>plot3d(x,y,z,points=true,style="."):
\end{eulerprompt}
\eulerimg{13}{images/Nur Alya Fadilah_Aplikom-154.png}
\begin{eulercomment}
\textgreater{}x=[1,2,3,4]; y=[4,5,6,1]; z=[6,1,2,3];\\
\textgreater{}plot3d(x,y,z,points=true,style="."):

- plot3d() untuk menjalankan perintah membuat plot 3D.\\
- Titik koordinat yang akan diplot pada sumbu x,y,z telah
didefinisikan oleh vektor baris x,y,z sebelumnya.\\
- points=true sebagai parameter yang memerintahkan plot3d akan
menampilkan points(titik-titik).

\end{eulercomment}
\begin{eulerprompt}
>x=[1,1,0,-4,4]; y=[2,-11,7,1,9]; z=[0,8,4,7,7];
>plot3d(x,y,z,points=true,zoom=3,style="/"):
\end{eulerprompt}
\eulerimg{13}{images/Nur Alya Fadilah_Aplikom-155.png}
\begin{eulercomment}
1.x, y, dan z: Ini adalah tiga vektor yang berisi koordinat
titik-titik dalam tiga dimensi. Dalam contoh ini, x berisi [1, 1, 0,
-4, 4], y berisi [2, -11, 7, 1, 9], dan z berisi [0, 8, 4, 7, 7].
Setiap elemen dalam vektor-vektor ini mewakili koordinat satu titik
dalam ruang 3D.

2.plot3d(...): Ini adalah fungsi yang digunakan untuk membuat plot 3D
dengan parameter-parameter berikut:

-.x, y, dan z: Ini adalah data koordinat yang akan digunakan untuk
membuat plot.\\
-points = true: Ini mengatur agar titik-titik data ditampilkan dalam
plot. Dengan points = true, Anda akan melihat titik-titik yang
mewakili koordinat data.\\
-zoom = 3: Ini mengatur tingkat zoom plot. Dengan zoom = 3, plot akan
diperbesar sebanyak tiga kali dari ukuran defaultnya.\\
-style = "/": Ini mengatur gaya plot menjadi tanda garis miring ("/").
\end{eulercomment}
\begin{eulerprompt}
>a=random(1,5); b=linspace(10,18,4); c=normal(1,5); ...
>plot3d(a,b,c,scale=[5,1,3],points=true,style="'"):
\end{eulerprompt}
\eulerimg{13}{images/Nur Alya Fadilah_Aplikom-156.png}
\begin{eulercomment}
\textgreater{}a=random(1,5); b=linspace(10,18,4); c=normal(1,5); ...\\
\end{eulercomment}
\begin{eulerttcomment}
 plot3d(a,b,c,scale=[5,1,3],points=true,style="'"):
\end{eulerttcomment}
\begin{eulercomment}

- plot3d() untuk menjalankan perintah membuat plot 3D.\\
- Titik koordinat yang akan diplot pada sumbu x,y,z didefinisikan oleh
x=a=random(1,5) yaitu bilangan acak dari 0-1 sebanyak 5 bilangan,
y=b=linspace(10,18,4) yaitu bilangan dari 10 hingga 18 dengan selisih
yang sama sebanyak 5 bilangan, z=c=normal(1,5) yaitu bilangan acak
distribusi normal sebanyak 5 bilangan.\\
- points=true sebagai parameter yang memerintahkan plot3d akan
menampilkan points(titik-titik).
\end{eulercomment}
\begin{eulercomment}

\end{eulercomment}
\eulersubheading{Mengatur Tampilan, Warna dan * Angle Gambar Permukaan 3D}
\begin{eulercomment}
Dalam plot3d terdapat banyak function terkait tampilan gambar 3D, di
antaranya:

sliced:\\
\end{eulercomment}
\begin{eulerttcomment}
  Memplot versi irisan (0=tidak, 1=arah-x, 2=arah-y).
\end{eulerttcomment}
\begin{eulercomment}
hue :\\
\end{eulercomment}
\begin{eulerttcomment}
  Menghitung bayangan menggunakan sumber cahaya.
\end{eulerttcomment}
\begin{eulercomment}
light, amb, max :\\
\end{eulercomment}
\begin{eulerttcomment}
  Mengontrol pengaturan bayangan titik cahaya, ambient dan maksimum.
\end{eulerttcomment}
\begin{eulercomment}
contour  :\\
\end{eulercomment}
\begin{eulerttcomment}
  Menampilkan garis level tebal (dengan level otomatis).
\end{eulerttcomment}
\begin{eulercomment}
spectral:\\
\end{eulercomment}
\begin{eulerttcomment}
  Gunakan warna spektral alih-alih rona monokrom. Terdapat
  skema spektral dari spektral = 1 (> spektral) hingga spektral = 9.
  >Default >spectral adalah rona warna dan ini setara dengan
  color=-2 hingga color=-10.
\end{eulerttcomment}
\begin{eulercomment}
xhue, yhue, zhue:\\
\end{eulercomment}
\begin{eulerttcomment}
  Gunakan koordinat ini sebagai pengganti sumber cahaya.
\end{eulerttcomment}
\begin{eulercomment}
hues :\\
\end{eulercomment}
\begin{eulerttcomment}
  Matriks nilai rona dari 0 sampai 1 untuk bayangan untuk plot x-y-z.
  Matriks harus memiliki ukuran yang kompatibel dengan x, y, z.
\end{eulerttcomment}
\begin{eulercomment}
contourcolor :\\
\end{eulercomment}
\begin{eulerttcomment}
  Warna garis kontur.
\end{eulerttcomment}
\begin{eulercomment}
contourwidth :\\
\end{eulercomment}
\begin{eulerttcomment}
  Lebar garis kontur.
\end{eulerttcomment}
\begin{eulercomment}
fillcolor :\\
\end{eulercomment}
\begin{eulerttcomment}
  Warna isian untuk permukaan 3d tanpa rona.
\end{eulerttcomment}
\begin{eulercomment}
user :\\
\end{eulercomment}
\begin{eulerttcomment}
  Pengguna dapat memutar plot dengan keyboard kiri, kanan, atas,
  bawah. +,- memperbesar plot. Spasi mengatur ulang plot. Return
  mengakhiri interaksi pengguna. Tombol a menghasilkan plot anaglyph.
  Tombol l mengalihkan pergerakan sumber cahaya untuk plot rona.
  Tombol c menggerakkan plot ke atas, bawah, kiri, atau kanan.
\end{eulerttcomment}
\begin{eulercomment}
rotate :\\
\end{eulercomment}
\begin{eulerttcomment}
  Memutar plot sebuah fungsi dalam satu ekspresi dalam x.
\end{eulerttcomment}
\begin{eulercomment}
anaglyph :\\
\end{eulercomment}
\begin{eulerttcomment}
  Menghasilkan plot 3d anaglyph (>anaglyph). Plot ini membutuhkan
  kacamata merah untuk dapat dilihat dengan baik.
\end{eulerttcomment}
\begin{eulercomment}
viewangle :\\
\end{eulercomment}
\begin{eulerttcomment}
  Sudut pandang default, diputar di sekitar z-
\end{eulerttcomment}
\begin{eulercomment}
zoom :\\
\end{eulercomment}
\begin{eulerttcomment}
  Pembesaran tampilan. Standarnya adalah sekitar 2,6.
\end{eulerttcomment}
\begin{eulercomment}
view :\\
\end{eulercomment}
\begin{eulerttcomment}
  Tampilan lengkap, vektor 1x4 yang berisi jarak, zoom, sudut pandang,
  tinggi pandang.
\end{eulerttcomment}
\begin{eulercomment}
center :\\
\end{eulercomment}
\begin{eulerttcomment}
  Vektor ini memindahkan pusat plot. Hal ini diperlukan jika plot
  tidak dipusatkan di (0,0,0) secara otomatis.
\end{eulerttcomment}
\begin{eulercomment}
style :\\
\end{eulercomment}
\begin{eulerttcomment}
  Gaya plot.
\end{eulerttcomment}
\begin{eulercomment}
color :\\
\end{eulercomment}
\begin{eulerttcomment}
  Warna untuk objek dan permukaan yang diarsir
\end{eulerttcomment}
\begin{eulercomment}
wirecolor :\\
\end{eulercomment}
\begin{eulerttcomment}
  Warna untuk plot kawat
\end{eulerttcomment}
\begin{eulercomment}
cp :\\
\end{eulercomment}
\begin{eulerttcomment}
  Menggambar bidang kontur di bawah plot (>cp).
\end{eulerttcomment}
\begin{eulercomment}
cpcolor :\\
\end{eulercomment}
\begin{eulerttcomment}
  Warna untuk bidang kontur.
\end{eulerttcomment}
\begin{eulerprompt}
>plot3d("x*y",r=4,title="z=x*y",zoom=5):
\end{eulerprompt}
\eulerimg{13}{images/Nur Alya Fadilah_Aplikom-157.png}
\begin{eulerprompt}
>plot3d("x*y^3",>user,r=1,>anaglyph,title="Press cursor keys or return!"):
\end{eulerprompt}
\eulerimg{13}{images/Nur Alya Fadilah_Aplikom-158.png}
\begin{eulerprompt}
>plot3d("x^2*y^3",r=0.9,zlabel="x^2*y^3",>user,zoom=3, ...
>fillcolor=[2,6],>cp,cpcolor=blue):
\end{eulerprompt}
\eulerimg{13}{images/Nur Alya Fadilah_Aplikom-159.png}
\begin{eulerprompt}
>plot3d("x^2+y^3",angle=80°,>contour,spectral=2):
\end{eulerprompt}
\eulerimg{13}{images/Nur Alya Fadilah_Aplikom-160.png}
\begin{eulerprompt}
>plot3d("x^y-y^x",a=0,b=4,c=0,d=4,angle=90°,>contour, ...
>  contourwidth=4,contourcolor=red):
\end{eulerprompt}
\eulerimg{13}{images/Nur Alya Fadilah_Aplikom-161.png}
\begin{eulerprompt}
>plot3d("x^2+3y^2",>wire,>anaglyph,title="Use Red/Cyan Glasses!"):
\end{eulerprompt}
\eulerimg{13}{images/Nur Alya Fadilah_Aplikom-162.png}
\begin{eulerprompt}
>plot3d("x^3+10y^2",0,2,0,10,scale=[5,1,2],zoom=3,grid=10,>transparent):
\end{eulerprompt}
\eulerimg{13}{images/Nur Alya Fadilah_Aplikom-163.png}
\begin{eulerprompt}
>x=-1:0.05:1; y=x'; plot3d(x,x*y^2,y,>user,>hue,angle=20°):
\end{eulerprompt}
\eulerimg{13}{images/Nur Alya Fadilah_Aplikom-164.png}
\begin{eulerprompt}
>X=normal(3,50); plot3d(X[1],X[2],X[3],>points,style="/",zoom=3,>user):
\end{eulerprompt}
\eulerimg{13}{images/Nur Alya Fadilah_Aplikom-165.png}
\begin{eulerprompt}
>plot3d("exp(x*y)",angle=100°,>contour,color=green):
\end{eulerprompt}
\eulerimg{13}{images/Nur Alya Fadilah_Aplikom-166.png}
\begin{eulercomment}
1. "exp(x*y)": Ini adalah fungsi matematika yang akan digunakan untuk
membuat plot 3D. Fungsi ini adalah eksponensial dari hasil perkalian
antara x dan y. Dalam konteks ini, x dan y adalah variabel-variabel
dalam plot.

2. angle = 100°: Ini adalah parameter angle yang mengatur sudut
tampilan plot. Dalam hal ini, plot akan dilihat dari sudut 100
derajat.

3. contour: Ini adalah parameter yang menambahkan garis kontur ke
plot. Ini memungkinkan Anda melihat kontur atau garis isovalue dalam
plot yang menggambarkan tingkat nilai fungsi.

4. color = green: Ini adalah parameter color yang mengatur warna plot.
Dalam hal ini, plot akan menggunakan warna hijau.
\end{eulercomment}
\begin{eulerprompt}
>plot3d("x^2+y^2",>spectral,>contour,n=100):
\end{eulerprompt}
\eulerimg{13}{images/Nur Alya Fadilah_Aplikom-167.png}
\begin{eulercomment}
1. "x\textasciicircum{}2 + y\textasciicircum{}2": Ini adalah fungsi matematika yang akan digunakan untuk
membuat plot 3D. Fungsi ini adalah fungsi kuadrat dari variabel x dan
y. Dalam konteks ini, x dan y adalah variabel-variabel dalam plot.

2. spectral: Ini adalah parameter yang mengatur skema warna plot
menjadi skema warna spektral. Dengan menggunakan skema warna spektral,
berbagai nilai dalam plot akan diberikan warna yang berbeda, yang
memudahkan untuk memahami perubahan nilai dalam fungsi.

3. contour: Ini adalah parameter yang menambahkan garis kontur ke
plot. Ini memungkinkan Anda melihat kontur atau garis isovalue dalam
plot yang menggambarkan tingkat nilai fungsi.

4. n = 100: Ini adalah parameter n yang mengatur jumlah titik sampel
dalam plot. Dalam hal ini, ada 100 titik sampel yang akan digunakan
untuk menggambarkan plot. Semakin banyak titik sampel, semakin halus
plotnya.
\end{eulercomment}
\begin{eulerprompt}
>plot3d("x^2-y^2",0,1,0,1,angle=220°,level=-1:0.2:1,color=redgreen):
\end{eulerprompt}
\eulerimg{13}{images/Nur Alya Fadilah_Aplikom-168.png}
\begin{eulercomment}
1. "x\textasciicircum{}2 - y\textasciicircum{}2": Ini adalah fungsi matematika yang akan digunakan untuk
membuat plot 3D. Fungsi ini adalah perbedaan antara kuadrat variabel x
dan kuadrat variabel y. Dalam konteks ini, x dan y adalah
variabel-variabel dalam plot.

2. 0, 1, 0, 1: Ini adalah parameter yang mengatur batasan tampilan
plot. Angka-angka ini mewakili batas minimum dan maksimum untuk x dan
y. Dalam hal ini, plot akan dibatasi pada rentang x dan y antara 0 dan
1.

3. angle = 220°: Ini adalah parameter angle yang mengatur sudut
tampilan plot. Dalam hal ini, plot akan dilihat dari sudut 220
derajat.

4. level = -1:0.2:1: Ini adalah parameter level yang mengatur tingkat
nilai fungsi yang akan ditampilkan dalam plot. Rentang ini (-1 hingga
1) akan dibagi menjadi beberapa tingkat, dengan selang 0.2 antara
masing-masing tingkat. Ini akan menghasilkan garis kontur pada tingkat
nilai fungsi tertentu.

5. color = redgreen: Ini adalah parameter color yang mengatur skema
warna plot. Warna yang digunakan adalah kombinasi warna merah dan
hijau.
\end{eulercomment}
\begin{eulerprompt}
>plot3d("x^2+y^3",level=[-0.1,0.9;0,1], ...
>  >spectral,angle=30°,grid=10,contourcolor=gray):
\end{eulerprompt}
\eulerimg{13}{images/Nur Alya Fadilah_Aplikom-169.png}
\begin{eulercomment}
1."x\textasciicircum{}2 + y\textasciicircum{}3": Ini adalah fungsi matematika yang akan digunakan untuk
membuat plot 3D. Fungsi ini adalah hasil penjumlahan dari kuadrat
variabel x dan kubik variabel y. Dalam konteks ini, x dan y adalah
variabel-variabel dalam plot.

2.level = [-0.1, 0.9; 0, 1]: Ini adalah parameter level yang mengatur
tingkat nilai fungsi yang akan ditampilkan dalam plot. Parameter ini
didefinisikan sebagai matriks dua baris dengan dua kolom. Setiap baris
berisi batasan tingkat nilai fungsi yang akan ditampilkan dalam plot.
Misalnya, [-0.1, 0.9] menunjukkan bahwa tingkat nilai akan ditampilkan
dari -0.1 hingga 0.9, dan [0, 1] menunjukkan bahwa tingkat nilai kedua
akan ditampilkan dari 0 hingga 1.

3.spectral: Ini adalah parameter yang mengatur skema warna plot
menjadi skema warna spektral.

4.angle = 30°: Ini adalah parameter angle yang mengatur sudut tampilan
plot. Dalam hal ini, plot akan dilihat dari sudut 30 derajat.

5.grid = 10: Ini adalah parameter grid yang mengatur jumlah garis kisi
dalam plot. Dalam hal ini, akan ada 10 garis kisi dalam plot.

6.contourcolor = gray: Ini adalah parameter contourcolor yang mengatur
warna garis kontur dalam plot menjadi abu-abu (gray).

Dalam contoh berikut, kami memplot himpunan, di mana

\end{eulercomment}
\begin{eulerformula}
\[
f(x,y) = x^y-y^x = 0
\]
\end{eulerformula}
\begin{eulercomment}
Kami menggunakan satu garis tipis untuk garis level.
\end{eulercomment}
\begin{eulerprompt}
>plot3d("x^y-y^x",level=0,a=0,b=6,c=0,d=6,contourcolor=red,n=100):
\end{eulerprompt}
\eulerimg{13}{images/Nur Alya Fadilah_Aplikom-170.png}
\eulersubheading{Menggambar Grafik Tiga Dimensi * alam modus anaglif}
\begin{eulerprompt}
>X=cumsum(normal(3,100)); ...
> plot3d(X[1],X[2],X[3],>anaglyph,>wire):
\end{eulerprompt}
\eulerimg{13}{images/Nur Alya Fadilah_Aplikom-171.png}
\begin{eulercomment}
1.X = cumsum(normal(3, 100));: Ini adalah urutan perintah yang
melakukan beberapa operasi berurutan.

a)normal(3, 100): Ini adalah panggilan fungsi normal yang digunakan
untuk menghasilkan 100 bilangan acak dengan distribusi normal
(Gaussian) dengan rata-rata 3 dan deviasi standar 1. Hasilnya adalah
vektor tiga dimensi yang berisi 100 titik acak dalam ruang tiga
dimensi.\\
b)cumsum(...): Ini adalah panggilan fungsi cumsum yang digunakan untuk
menghitung kumulatif dari vektor 3D yang dihasilkan sebelumnya. Dengan
kata lain, ini akan menghasilkan vektor yang merupakan akumulasi
(penjumlahan berulang) dari vektor 3D tersebut. Hasilnya adalah vektor
tiga dimensi yang menggambarkan perjalanan dalam ruang 3D berdasarkan
perubahan titik acak.\\
2.plot3d(X[1], X[2], X[3], anaglyph, wire);: Ini adalah perintah untuk
membuat plot 3D dari data yang telah dihasilkan sebelumnya.

a).X[1], X[2], dan X[3] adalah komponen vektor tiga dimensi X yang
akan digunakan sebagai koordinat dalam plot 3D. X[1] digunakan sebagai
koordinat sumbu x, X[2] digunakan sebagai koordinat sumbu y, dan X[3]
digunakan sebagai koordinat sumbu z.\\
b)anaglyph: Ini adalah parameter yang mengatur plot menggunakan efek
anaglif. Anaglif adalah teknik yang digunakan untuk menghasilkan efek
tiga dimensi (3D) dengan menggunakan dua gambar yang sedikit berbeda
untuk mata kiri dan kanan, dan penonton memerlukan kacamata anaglif
khusus untuk melihat efek 3D.\\
c)wire: Ini adalah parameter yang mengatur plot sebagai plot tali
(wireframe), yang berarti hanya garis-garis yang menghubungkan
titik-titik yang akan ditampilkan dalam plot.\\
4.X[1], X[2], dan X[3]: Ini adalah komponen dari vektor X. X[1]
digunakan sebagai koordinat sumbu x, X[2] digunakan sebagai koordinat
sumbu y, dan X[3] digunakan sebagai koordinat sumbu z. Dengan
menggunakan komponen vektor ini sebagai koordinat, Anda membuat plot
3D yang merepresentasikan perubahan dalam tiga dimensi berdasarkan
data dalam vektor X.

5.anaglyph: Ini adalah parameter yang mengatur plot menggunakan efek
anaglif. Efek anaglif adalah teknik yang digunakan untuk menghasilkan
efek tiga dimensi (3D) dengan menggunakan dua gambar yang sedikit
berbeda untuk mata kiri dan kanan. Penonton memerlukan kacamata
anaglif khusus dengan lensa berwarna berbeda untuk mata kiri dan kanan
untuk melihat efek 3D ini. Penggunaan anaglyph dalam kode ini
menunjukkan bahwa plot akan dibuat dengan efek anaglif.

6.wire: Ini adalah parameter yang mengatur plot sebagai plot tali
(wireframe). Wireframe adalah gaya plot di mana hanya garis-garis yang
menghubungkan titik-titik yang akan ditampilkan dalam plot. Dengan
pengaturan wire, Anda akan melihat plot dalam bentuk rangkaian
garis-garis yang menggambarkan bentuk objek dalam tampilan 3D.
\end{eulercomment}
\begin{eulerprompt}
>plot3d("x^2+y^3",>anaglyph,>contour,angle=30°):
\end{eulerprompt}
\eulerimg{13}{images/Nur Alya Fadilah_Aplikom-172.png}
\begin{eulercomment}
1. "x\textasciicircum{}2 + y\textasciicircum{}3": Ini adalah fungsi matematika yang akan digunakan untuk
membuat plot 3D. Fungsi ini adalah hasil penjumlahan dari kuadrat
variabel x dan kubik variabel y. Dalam konteks ini, x dan y adalah
variabel-variabel dalam plot.

2. anaglyph: Ini adalah parameter yang mengatur plot menggunakan efek
anaglif. Anaglif adalah teknik yang digunakan untuk menghasilkan efek
tiga dimensi (3D) dengan menggunakan dua gambar yang sedikit berbeda
untuk mata kiri dan kanan. Penonton memerlukan kacamata anaglif khusus
dengan lensa berwarna berbeda untuk mata kiri dan kanan untuk melihat
efek 3D ini. Dengan pengaturan ini, plot akan dibuat dengan efek 3D
anaglif.

3. contour: Ini adalah parameter yang menambahkan garis kontur ke
plot. Ini memungkinkan Anda melihat kontur atau garis isovalue dalam
plot yang menggambarkan tingkat nilai fungsi.

4. angle = 30°: Ini adalah parameter angle yang mengatur sudut
tampilan plot. Dalam hal ini, plot akan dilihat dari sudut 30 derajat.
\end{eulercomment}
\begin{eulerprompt}
>u=linspace(-1,1,10); v=linspace(0,2*pi,50)'; ...
>X=(3+u*cos(v/2))*cos(v); Y=(3+u*cos(v/2))*sin(v); Z=u*sin(v/2); ...
>plot3d(X,Y,Z,>anaglyph,<frame,>wire,scale=2.3):
\end{eulerprompt}
\eulerimg{13}{images/Nur Alya Fadilah_Aplikom-173.png}
\begin{eulercomment}
1. u = linspace(-1, 1, 10);: Ini adalah perintah untuk membuat vektor
u yang berisi 10 nilai yang dihasilkan secara merata dalam rentang -1
hingga 1. Vektor ini akan digunakan dalam perhitungan selanjutnya.

2.v = linspace(0, 2 * pi, 50)';: Ini adalah perintah untuk membuat
vektor v yang berisi 50 nilai yang dihasilkan secara merata dalam
rentang 0 hingga 2p (dua kali nilai p). Vektor ini juga akan digunakan
dalam perhitungan selanjutnya.

3.X, Y, dan Z: Ini adalah perintah-perintah yang digunakan untuk
menghasilkan data koordinat dalam tiga dimensi. Data ini dihasilkan
berdasarkan persamaan yang menggunakan nilai u dan v. Data ini akan
digunakan untuk membuat plot 3D.

4.plot3d(X, Y, Z, anaglyph, \textless{}frame, wire, scale = 2.3);: Ini adalah
perintah untuk membuat plot 3D berdasarkan data X, Y, dan Z yang telah
dihasilkan sebelumnya. Parameter-parameter yang digunakan dalam
perintah ini adalah:

a)anaglyph: Ini adalah parameter yang mengatur plot menggunakan efek
anaglif, yang memberikan efek tiga dimensi (3D) saat melihat plot
dengan kacamata anaglif.\\
b)\textless{}frame: Ini adalah parameter yang mengatur agar frame (kerangka)
plot ditampilkan. Ini adalah bingkai atau batasan dari plot.\\
c)wire: Ini adalah parameter yang mengatur plot dalam bentuk rangkaian
garis-garis (wireframe).\\
d)scale = 2.3: Ini adalah parameter yang mengatur faktor skala plot
sebesar 2.3. Ini akan memperbesar plot.
\end{eulercomment}
\begin{eulerprompt}
>u:=linspace(-pi,pi,160); v:=linspace(-pi,pi,400)';  ...
>x:=(4*(1+.25*sin(3*v))+cos(u))*cos(2*v); ...
>y:=(4*(1+.25*sin(3*v))+cos(u))*sin(2*v); ...
> z=sin(u)+2*cos(3*v); ...
>plot3d(x,y,z,frame=0,scale=1.5,hue=1,light=[1,0,-1],zoom=2.8,>anaglyph):
\end{eulerprompt}
\eulerimg{13}{images/Nur Alya Fadilah_Aplikom-174.png}
\begin{eulercomment}
1. u := linspace(-pi, pi, 160);: Ini adalah perintah untuk membuat
vektor u yang berisi 160 nilai yang dihasilkan secara merata dalam
rentang dari -p hingga p.

2.v := linspace(-pi, pi, 400)';: Ini adalah perintah untuk membuat
vektor v yang berisi 400 nilai yang dihasilkan secara merata dalam
rentang dari -p hingga p. Vektor ini diubah menjadi matriks kolom
dengan penambahan tanda apostrof (') di belakangnya.

3.x, y, dan z: Ini adalah perintah-perintah yang digunakan untuk
menghasilkan data koordinat dalam tiga dimensi. Data ini dihasilkan
berdasarkan persamaan yang menggunakan nilai u dan v. Data ini akan
digunakan untuk membuat plot 3D.

4.plot3d(x, y, z, frame = 0, scale = 1.5, hue = 1, light = [1, 0, -1],
zoom = 2.8, anaglyph);: Ini adalah perintah untuk membuat plot 3D
berdasarkan data x, y, dan z yang telah dihasilkan sebelumnya.
Parameter-parameter yang digunakan dalam perintah ini adalah:

a)frame = 0: Ini adalah parameter yang mengatur agar frame (kerangka)
plot tidak ditampilkan.\\
b)scale = 1.5: Ini adalah parameter yang mengatur faktor skala plot
sebesar 1.5. Ini akan memperbesar plot.\\
c)hue = 1: Ini adalah parameter yang mengatur warna plot dengan skala
warna tunggal.\\
d)light = [1, 0, -1]: Ini adalah parameter yang mengatur pencahayaan
plot dengan arah cahaya yang ditentukan oleh vektor [1, 0, -1].\\
e)zoom = 2.8: Ini adalah parameter yang mengatur tingkat zoom plot
sebesar 2.8. Ini akan memperbesar plot.\\
f)anaglyph: Ini adalah parameter yang mengatur plot menggunakan efek
anaglif, yang memberikan efek tiga dimensi (3D) saat dilihat dengan
kacamata anaglif.\\
\end{eulercomment}
\eulersubheading{Plot Statistik batang 3d}
\begin{eulercomment}
Plot bar juga dimungkinkan. Untuk ini, kita harus menyediakan

- x: vektor baris dengan n+1 elemen\\
- y: vektor kolom dengan n+1 elemen\\
- z: matriks nilai nxn.

z bisa lebih besar, tetapi hanya nilai nxn yang akan digunakan.

Dalam contoh, pertama-tama kita menghitung nilainya. Kemudian kita
sesuaikan x dan y, sehingga vektor berpusat pada nilai yang digunakan.
\end{eulercomment}
\begin{eulerprompt}
>x=-1:0.1:1; y=x'; z=x^2+y^2; ...
>xa=(x|1.1)-0.05; ya=(y_1.1)-0.05; ...
>plot3d(xa,ya,z,bar=true):
\end{eulerprompt}
\eulerimg{13}{images/Nur Alya Fadilah_Aplikom-175.png}
\begin{eulercomment}
Dimungkinkan untuk membagi plot permukaan menjadi dua atau lebih
bagian.
\end{eulercomment}
\begin{eulerprompt}
>x=-1:0.1:1; y=x'; z=x+y; d=zeros(size(x)); ...
>plot3d(x,y,z,disconnect=2:2:20):
\end{eulerprompt}
\eulerimg{13}{images/Nur Alya Fadilah_Aplikom-176.png}
\begin{eulercomment}
Jika memuat atau menghasilkan matriks data M dari file dan perlu
memplotnya dalam 3D, Anda dapat menskalakan matriks ke [-1,1] dengan
scale(M), atau menskalakan matriks dengan \textgreater{}zscale. Ini dapat
dikombinasikan dengan faktor penskalaan individu yang diterapkan
sebagai tambahan.
\end{eulercomment}
\begin{eulerprompt}
>i=1:20; j=i'; ...
>plot3d(i*j^2+100*normal(20,20),>zscale,scale=[1,1,1.5],angle=-40°,zoom=1.8):
\end{eulerprompt}
\eulerimg{13}{images/Nur Alya Fadilah_Aplikom-177.png}
\begin{eulerprompt}
>Z=intrandom(5,100,6); v=zeros(5,6); ...
>loop 1 to 5; v[#]=getmultiplicities(1:6,Z[#]); end; ...
>columnsplot3d(v',scols=1:5,ccols=[1:5]):
\end{eulerprompt}
\eulerimg{13}{images/Nur Alya Fadilah_Aplikom-178.png}
\eulersubheading{Permukaan Benda Putar}
\begin{eulerprompt}
>plot2d("(x^2+y^2-1)^3-x^2*y^3",r=1.3, ...
>style="#",color=red,<outline, ...
>level=[-2;0],n=100):
\end{eulerprompt}
\eulerimg{13}{images/Nur Alya Fadilah_Aplikom-179.png}
\begin{eulerprompt}
>ekspresi &= (x^2+y^2-1)^3-x^2*y^3; $ekspresi
\end{eulerprompt}
\begin{eulerformula}
\[
\left(y^2+x^2-1\right)^3-x^2\,y^3
\]
\end{eulerformula}
\begin{eulercomment}
Kami ingin memutar kurva jantung di sekitar sumbu y. Berikut adalah
ungkapan, yang mendefinisikan hati:

\end{eulercomment}
\begin{eulerformula}
\[
f(x,y)=(x^2+y^2-1)^3-x^2.y^3.
\]
\end{eulerformula}
\begin{eulercomment}
Selanjutnya kita atur

\end{eulercomment}
\begin{eulerformula}
\[
x=r.cos(a),\quad y=r.sin(a).
\]
\end{eulerformula}
\begin{eulerprompt}
>function fr(r,a) &= ekspresi with [x=r*cos(a),y=r*sin(a)] | trigreduce; $fr(r,a)
\end{eulerprompt}
\begin{eulerformula}
\[
\left(r^2-1\right)^3+\frac{\left(\sin \left(5\,a\right)-\sin \left(
 3\,a\right)-2\,\sin a\right)\,r^5}{16}
\]
\end{eulerformula}
\begin{eulercomment}
Hal ini memungkinkan untuk mendefinisikan fungsi numerik, yang
memecahkan r, jika a diberikan. Dengan fungsi itu kita dapat memplot
jantung yang diputar sebagai permukaan parametrik.
\end{eulercomment}
\begin{eulerprompt}
>function map f(a) := bisect("fr",0,2;a); ...
>t=linspace(-pi/2,pi/2,100); r=f(t);  ...
>s=linspace(pi,2pi,100)'; ...
>plot3d(r*cos(t)*sin(s),r*cos(t)*cos(s),r*sin(t), ...
>>hue,<frame,color=red,zoom=4,amb=0,max=0.7,grid=12,height=50°):
\end{eulerprompt}
\eulerimg{13}{images/Nur Alya Fadilah_Aplikom-184.png}
\begin{eulercomment}
Berikut ini adalah plot 3D dari gambar di atas yang diputar di sekitar
sumbu z. Kami mendefinisikan fungsi, yang menggambarkan objek.
\end{eulercomment}
\begin{eulerprompt}
>function f(x,y,z) ...
\end{eulerprompt}
\begin{eulerudf}
  r=x^2+y^2; ...
  return (r+z^2-1)^3-r*z^3;
  endfunction
\end{eulerudf}
\begin{eulerprompt}
>plot3d("f(x,y,z)", ...
>xmin=0,xmax=1.2,ymin=-1.2,ymax=1.2,zmin=-1.2,zmax=1.4, ...
>implicit=1,angle=-30°,zoom=2.5,n=[10,60,60],>anaglyph):
\end{eulerprompt}
\eulerimg{13}{images/Nur Alya Fadilah_Aplikom-185.png}
\begin{eulercomment}
b. Hal hal yang dilakukan dalam mempelajari materi\\
- Mencari informasi mengenai materi plot 3D.\\
- mencarai latihan soal di buku dan internet.\\
- Mempelajari perintah perintah yang ada di EMT berkaitan dengan plot
3D.

c. Kendala kendala dan usaha untuk mengatasi kendala tersebut\\
- Kesulitan dalam menggunakan perintah terutama saat menggunakan
Povray, solusinya dengan mencari informasi di internet\\
\end{eulercomment}
\eulersubheading{}
\begin{eulerprompt}
> 
\end{eulerprompt}
\begin{eulercomment}
\end{eulercomment}
\eulersubheading{6. Penggunaan software EMT untuk aplikasi Geometri}
\begin{eulercomment}
a. Hal hal yang dipelajari beserta contohnya\\
- Memanggil program geometry untuk menggunakan perintah\\
- Fungsi fungsi geometri\\
- Luas, lingkaran luar, lingkaran dalam segitiga\\
- Gepmetri simbolik\\
- Garis dan lingkaran berpotongan.\\
- Garis sumbu\\
- jarak minimal pada bidang
\end{eulercomment}
\begin{eulerprompt}
>load geometry
\end{eulerprompt}
\begin{euleroutput}
  Numerical and symbolic geometry.
\end{euleroutput}
\eulersubheading{Fungsi-fungsi Geometri}
\begin{eulercomment}
Fungsi-fungsi untuk Menggambar Objek Geometri:

\end{eulercomment}
\begin{eulerttcomment}
  defaultd:=textheight()*1.5: nilai asli untuk parameter d
  setPlotrange(x1,x2,y1,y2): menentukan rentang x dan y pada bidang
\end{eulerttcomment}
\begin{eulercomment}
koordinat\\
\end{eulercomment}
\begin{eulerttcomment}
  setPlotRange(r): pusat bidang koordinat (0,0) dan batas-batas
\end{eulerttcomment}
\begin{eulercomment}
sumbu-x dan y adalah -r sd r\\
\end{eulercomment}
\begin{eulerttcomment}
  plotPoint (P, "P"): menggambar titik P dan diberi label "P"
  plotSegment (A,B, "AB", d): menggambar ruas garis AB, diberi label
\end{eulerttcomment}
\begin{eulercomment}
"AB" sejauh d\\
\end{eulercomment}
\begin{eulerttcomment}
  plotLine (g, "g", d): menggambar garis g diberi label "g" sejauh d
  plotCircle (c,"c",v,d): Menggambar lingkaran c dan diberi label "c"
  plotLabel (label, P, V, d): menuliskan label pada posisi P
\end{eulerttcomment}
\begin{eulercomment}

Fungsi-fungsi Geometri Analitik (numerik maupun simbolik):

\end{eulercomment}
\begin{eulerttcomment}
  turn(v, phi): memutar vektor v sejauh phi
  turnLeft(v):   memutar vektor v ke kiri
  turnRight(v):  memutar vektor v ke kanan
  normalize(v): normal vektor v
  crossProduct(v, w): hasil kali silang vektorv dan w.
  lineThrough(A, B): garis melalui A dan B, hasilnya [a,b,c] sdh.
\end{eulerttcomment}
\begin{eulercomment}
ax+by=c.\\
\end{eulercomment}
\begin{eulerttcomment}
  lineWithDirection(A,v): garis melalui A searah vektor v
  getLineDirection(g): vektor arah (gradien) garis g
  getNormal(g): vektor normal (tegak lurus) garis g
  getPointOnLine(g):  titik pada garis g
  perpendicular(A, g):  garis melalui A tegak lurus garis g
  parallel (A, g):  garis melalui A sejajar garis g
  lineIntersection(g, h):  titik potong garis g dan h
  projectToLine(A, g):   proyeksi titik A pada garis g
  distance(A, B):  jarak titik A dan B
  distanceSquared(A, B):  kuadrat jarak A dan B
  quadrance(A, B): kuadrat jarak A dan B
  areaTriangle(A, B, C):  luas segitiga ABC
  computeAngle(A, B, C):   besar sudut <ABC
  angleBisector(A, B, C): garis bagi sudut <ABC
  circleWithCenter (A, r): lingkaran dengan pusat A dan jari-jari r
  getCircleCenter(c):  pusat lingkaran c
  getCircleRadius(c):  jari-jari lingkaran c
  circleThrough(A,B,C):  lingkaran melalui A, B, C
  middlePerpendicular(A, B): titik tengah AB
  lineCircleIntersections(g, c): titik potong garis g dan lingkran c
  circleCircleIntersections (c1, c2):  titik potong lingkaran c1 dan
\end{eulerttcomment}
\begin{eulercomment}
c2\\
\end{eulercomment}
\begin{eulerttcomment}
  planeThrough(A, B, C):  bidang melalui titik A, B, C
\end{eulerttcomment}
\begin{eulercomment}

Fungsi-fungsi Khusus Untuk Geometri Simbolik:

\end{eulercomment}
\begin{eulerttcomment}
  getLineEquation (g,x,y): persamaan garis g dinyatakan dalam x dan y
  getHesseForm (g,x,y,A): bentuk Hesse garis g dinyatakan dalam x dan
\end{eulerttcomment}
\begin{eulercomment}
y dengan titik A pada sisi positif (kanan/atas) garis\\
\end{eulercomment}
\begin{eulerttcomment}
  quad(A,B): kuadrat jarak AB
  spread(a,b,c): Spread segitiga dengan panjang sisi-sisi a,b,c, yakni
\end{eulerttcomment}
\begin{eulercomment}
sin(alpha)\textasciicircum{}2 dengan alpha sudut yang menghadap sisi a.\\
\end{eulercomment}
\begin{eulerttcomment}
  crosslaw(a,b,c,sa): persamaan 3 quads dan 1 spread pada segitiga
\end{eulerttcomment}
\begin{eulercomment}
dengan panjang sisi a, b, c.\\
\end{eulercomment}
\begin{eulerttcomment}
  triplespread(sa,sb,sc): persamaan 3 spread sa,sb,sc yang memebntuk
\end{eulerttcomment}
\begin{eulercomment}
suatu segitiga\\
\end{eulercomment}
\begin{eulerttcomment}
  doublespread(sa): Spread sudut rangkap Spread 2*phi, dengan
\end{eulerttcomment}
\begin{eulercomment}
sa=sin(phi)\textasciicircum{}2 spread a.

\end{eulercomment}
\eulersubheading{Luas, Lingkaran Luar, Lingkaran Dalam Segitiga}
\begin{eulercomment}
Untuk menggambar objek-objek geometri, langkah pertama adalah
menentukan rentang sumbu-sumbu koordinat. Semua objek geometri akan
digambar pada satu bidang koordinat, sampai didefinisikan bidang
koordinat yang baru.
\end{eulercomment}
\begin{eulerprompt}
>setPlotRange(-0.5,2.5,-0.5,2.5); // mendefinisikan bidang koordinat baru 
\end{eulerprompt}
\begin{eulercomment}
Sekarang tetapkan tiga poin dan plot mereka.
\end{eulercomment}
\begin{eulerprompt}
>A=[1,0]; plotPoint(A,"A"); // definisi dan gambar tiga titik
>B=[0,1]; plotPoint(B,"B");
>C=[2,2]; plotPoint(C,"C");
\end{eulerprompt}
\begin{eulercomment}
Kemudian tiga segmen.
\end{eulercomment}
\begin{eulerprompt}
>plotSegment(A,B,"c"); // c=AB
>plotSegment(B,C,"a"); // a=BC
>plotSegment(A,C,"b"); // b=AC
\end{eulerprompt}
\begin{eulercomment}
Fungsi geometri meliputi fungsi untuk membuat garis dan lingkaran.
Format garis adalah [a,b,c], yang mewakili garis dengan persamaan
ax+by=c.
\end{eulercomment}
\begin{eulerprompt}
>lineThrough(B,C) // garis yang melalui B dan C
\end{eulerprompt}
\begin{euleroutput}
  [-1,  2,  2]
\end{euleroutput}
\begin{eulercomment}
Hitunglah garis tegak lurus yang melalui A pada BC.
\end{eulercomment}
\begin{eulerprompt}
>h=perpendicular(A,lineThrough(B,C)); // garis h tegak lurus BC melalui A
\end{eulerprompt}
\begin{eulercomment}
Dan persimpangannya dengan BC.
\end{eulercomment}
\begin{eulerprompt}
>D=lineIntersection(h,lineThrough(B,C)); // D adalah titik potong h dan BC
\end{eulerprompt}
\begin{eulercomment}
Plot itu.
\end{eulercomment}
\begin{eulerprompt}
>plotPoint(D,value=1); // koordinat D ditampilkan
>aspect(1); plotSegment(A,D): // tampilkan semua gambar hasil plot...()
\end{eulerprompt}
\eulerimg{27}{images/Nur Alya Fadilah_Aplikom-186.png}
\begin{eulercomment}
Hitung luas ABC:

\end{eulercomment}
\begin{eulerformula}
\[
L_{\triangle ABC}= \frac{1}{2}AD.BC.
\]
\end{eulerformula}
\begin{eulerprompt}
>norm(A-D)*norm(B-C)/2 // AD=norm(A-D), BC=norm(B-C)
\end{eulerprompt}
\begin{euleroutput}
  1.5
\end{euleroutput}
\begin{eulercomment}
Bandingkan dengan rumus determinan.
\end{eulercomment}
\begin{eulerprompt}
>areaTriangle(A,B,C) // hitung luas segitiga langusng dengan fungsi
\end{eulerprompt}
\begin{euleroutput}
  1.5
\end{euleroutput}
\begin{eulercomment}
Cara lain menghitung luas segitigas ABC:
\end{eulercomment}
\begin{eulerprompt}
>distance(A,D)*distance(B,C)/2
\end{eulerprompt}
\begin{euleroutput}
  1.5
\end{euleroutput}
\begin{eulercomment}
Sudut di C
\end{eulercomment}
\begin{eulerprompt}
>degprint(computeAngle(B,C,A))
\end{eulerprompt}
\begin{euleroutput}
  36°52'11.63''
\end{euleroutput}
\begin{eulercomment}
Sekarang lingkaran luar segitiga.
\end{eulercomment}
\begin{eulerprompt}
>c=circleThrough(A,B,C); // lingkaran luar segitiga ABC
>R=getCircleRadius(c); // jari2 lingkaran luar 
>O=getCircleCenter(c); // titik pusat lingkaran c 
>plotPoint(O,"O"); // gambar titik "O"
>plotCircle(c,"Lingkaran luar segitiga ABC"):
\end{eulerprompt}
\eulerimg{27}{images/Nur Alya Fadilah_Aplikom-188.png}
\begin{eulercomment}
Tampilkan koordinat titik pusat dan jari-jari lingkaran luar.
\end{eulercomment}
\begin{eulerprompt}
>O, R
\end{eulerprompt}
\begin{euleroutput}
  [1.16667,  1.16667]
  1.17851130198
\end{euleroutput}
\begin{eulercomment}
Sekarang akan digambar lingkaran dalam segitiga ABC. Titik pusat lingkaran dalam adalah
titik potong garis-garis bagi sudut.
\end{eulercomment}
\begin{eulerprompt}
>l=angleBisector(A,C,B); // garis bagi <ACB
>g=angleBisector(C,A,B); // garis bagi <CAB
>P=lineIntersection(l,g) // titik potong kedua garis bagi sudut
\end{eulerprompt}
\begin{euleroutput}
  [0.86038,  0.86038]
\end{euleroutput}
\begin{eulercomment}
Tambahkan semuanya ke plot.
\end{eulercomment}
\begin{eulerprompt}
>color(5); plotLine(l); plotLine(g); color(1); // gambar kedua garis bagi sudut
>plotPoint(P,"P"); // gambar titik potongnya
>r=norm(P-projectToLine(P,lineThrough(A,B))) // jari-jari lingkaran dalam
\end{eulerprompt}
\begin{euleroutput}
  0.509653732104
\end{euleroutput}
\begin{eulerprompt}
>plotCircle(circleWithCenter(P,r),"Lingkaran dalam segitiga ABC"): // gambar lingkaran dalam
\end{eulerprompt}
\eulerimg{27}{images/Nur Alya Fadilah_Aplikom-189.png}
\eulersubheading{Geometri Simbolik}
\begin{eulercomment}
Kita dapat menghitung geometri eksak dan simbolik menggunakan Maxima.

File geometri.e menyediakan fungsi yang sama (dan lebih banyak lagi)
di Maxima. Namun, kita dapat menggunakan perhitungan simbolis
sekarang.
\end{eulercomment}
\begin{eulerprompt}
>A &= [1,0]; B &= [0,1]; C &= [2,2]; // menentukan tiga titik A, B, C
\end{eulerprompt}
\begin{eulercomment}
Fungsi untuk garis dan lingkaran bekerja seperti fungsi Euler, tetapi
memberikan perhitungan simbolis.
\end{eulercomment}
\begin{eulerprompt}
>c &= lineThrough(B,C) // c=BC
\end{eulerprompt}
\begin{euleroutput}
  
                               [- 1, 2, 2]
  
\end{euleroutput}
\begin{eulercomment}
Kita bisa mendapatkan persamaan garis dengan mudah.
\end{eulercomment}
\begin{eulerprompt}
>$getLineEquation(c,x,y), $solve(%,y) | expand // persamaan garis c
\end{eulerprompt}
\begin{eulerformula}
\[
2\,y-x=2
\]
\end{eulerformula}
\begin{eulerformula}
\[
\left[ y=\frac{x}{2}+1 \right] 
\]
\end{eulerformula}
\begin{eulerprompt}
>$getLineEquation(lineThrough(A,[x1,y1]),x,y) // persamaan garis melalui A dan (x1, y1)
\end{eulerprompt}
\begin{eulerformula}
\[
\left({\it x_1}-1\right)\,y-x\,{\it y_1}=-{\it y_1}
\]
\end{eulerformula}
\begin{eulerprompt}
>h &= perpendicular(A,lineThrough(B,C)) // h melalui A tegak lurus BC
\end{eulerprompt}
\begin{euleroutput}
  
                                [2, 1, 2]
  
\end{euleroutput}
\begin{eulerprompt}
>Q &= lineIntersection(c,h) // Q titik potong garis c=BC dan h
\end{eulerprompt}
\begin{euleroutput}
  
                                   2  6
                                  [-, -]
                                   5  5
  
\end{euleroutput}
\begin{eulerprompt}
>$projectToLine(A,lineThrough(B,C)) // proyeksi A pada BC
\end{eulerprompt}
\begin{eulerformula}
\[
\left[ \frac{2}{5} , \frac{6}{5} \right] 
\]
\end{eulerformula}
\begin{eulerprompt}
>$distance(A,Q) // jarak AQ
\end{eulerprompt}
\begin{eulerformula}
\[
\frac{3}{\sqrt{5}}
\]
\end{eulerformula}
\begin{eulerprompt}
>cc &= circleThrough(A,B,C); $cc // (titik pusat dan jari-jari) lingkaran melalui A, B, C
\end{eulerprompt}
\begin{eulerformula}
\[
\left[ \frac{7}{6} , \frac{7}{6} , \frac{5}{3\,\sqrt{2}} \right] 
\]
\end{eulerformula}
\begin{eulerprompt}
>r&=getCircleRadius(cc); $r , $float(r) // tampilkan nilai jari-jari
\end{eulerprompt}
\begin{eulerformula}
\[
\frac{5}{3\,\sqrt{2}}
\]
\end{eulerformula}
\begin{eulerformula}
\[
1.178511301977579
\]
\end{eulerformula}
\begin{eulerprompt}
>$computeAngle(A,C,B) // nilai <ACB
\end{eulerprompt}
\begin{eulerformula}
\[
\arccos \left(\frac{4}{5}\right)
\]
\end{eulerformula}
\begin{eulerprompt}
>$solve(getLineEquation(angleBisector(A,C,B),x,y),y)[1] // persamaan garis bagi <ACB
\end{eulerprompt}
\begin{eulerformula}
\[
y=x
\]
\end{eulerformula}
\begin{eulerprompt}
>P &= lineIntersection(angleBisector(A,C,B),angleBisector(C,B,A)); $P // titik potong 2 garis bagi sudut
\end{eulerprompt}
\begin{eulerformula}
\[
\left[ \frac{\sqrt{2}\,\sqrt{5}+2}{6} , \frac{\sqrt{2}\,\sqrt{5}+2
 }{6} \right] 
\]
\end{eulerformula}
\begin{eulerprompt}
>P() // hasilnya sama dengan perhitungan sebelumnya
\end{eulerprompt}
\begin{euleroutput}
  [0.86038,  0.86038]
\end{euleroutput}
\eulersubheading{Garis dan Lingkaran yang Berpotongan}
\begin{eulercomment}
Tentu saja, kita juga dapat memotong garis dengan lingkaran, dan
lingkaran dengan lingkaran.
\end{eulercomment}
\begin{eulerprompt}
>A &:= [1,0]; c=circleWithCenter(A,4);
>B &:= [1,2]; C &:= [2,1]; l=lineThrough(B,C);
>setPlotRange(5); plotCircle(c); plotLine(l);
\end{eulerprompt}
\begin{eulercomment}
Perpotongan garis dengan lingkaran menghasilkan dua titik dan jumlah
titik potong.
\end{eulercomment}
\begin{eulerprompt}
>\{P1,P2,f\}=lineCircleIntersections(l,c);
>P1, P2,
\end{eulerprompt}
\begin{euleroutput}
  [4.64575,  -1.64575]
  [-0.645751,  3.64575]
\end{euleroutput}
\begin{eulerprompt}
>plotPoint(P1); plotPoint(P2):
\end{eulerprompt}
\eulerimg{27}{images/Nur Alya Fadilah_Aplikom-201.png}
\begin{eulercomment}
Begitu pula di Maxima.
\end{eulercomment}
\begin{eulerprompt}
>c &= circleWithCenter(A,4) // lingkaran dengan pusat A jari-jari 4
\end{eulerprompt}
\begin{euleroutput}
  
                                [1, 0, 4]
  
\end{euleroutput}
\begin{eulerprompt}
>l &= lineThrough(B,C) // garis l melalui B dan C
\end{eulerprompt}
\begin{euleroutput}
  
                                [1, 1, 3]
  
\end{euleroutput}
\begin{eulerprompt}
>$lineCircleIntersections(l,c) | radcan, // titik potong lingkaran c dan garis l
\end{eulerprompt}
\begin{eulerformula}
\[
\left[ \left[ \sqrt{7}+2 , 1-\sqrt{7} \right]  , \left[ 2-\sqrt{7}
  , \sqrt{7}+1 \right]  \right] 
\]
\end{eulerformula}
\begin{eulercomment}
Akan ditunjukkan bahwa sudut-sudut yang menghadap bsuusr yang sama adalah sama besar.
\end{eulercomment}
\begin{eulerprompt}
>C=A+normalize([-2,-3])*4; plotPoint(C); plotSegment(P1,C); plotSegment(P2,C);
>degprint(computeAngle(P1,C,P2))
\end{eulerprompt}
\begin{euleroutput}
  69°17'42.68''
\end{euleroutput}
\begin{eulerprompt}
>C=A+normalize([-4,-3])*4; plotPoint(C); plotSegment(P1,C); plotSegment(P2,C);
>degprint(computeAngle(P1,C,P2))
\end{eulerprompt}
\begin{euleroutput}
  69°17'42.68''
\end{euleroutput}
\begin{eulerprompt}
>insimg;
\end{eulerprompt}
\eulerimg{27}{images/Nur Alya Fadilah_Aplikom-203.png}
\eulersubheading{Garis Sumbu}
\begin{eulercomment}
Berikut adalah langkah-langkah menggambar garis sumbu ruas garis AB:

1. Gambar lingkaran dengan pusat A melalui B.\\
2. Gambar lingkaran dengan pusat B melalui A.\\
3. Tarik garis melallui kedua titik potong kedua lingkaran tersebut. Garis ini merupakan
garis sumbu (melalui titik tengah dan tegak lurus) AB.
\end{eulercomment}
\begin{eulerprompt}
>A=[2,2]; B=[-1,-2];
>c1=circleWithCenter(A,distance(A,B));
>c2=circleWithCenter(B,distance(A,B));
>\{P1,P2,f\}=circleCircleIntersections(c1,c2);
>l=lineThrough(P1,P2);
>setPlotRange(5); plotCircle(c1); plotCircle(c2);
>plotPoint(A); plotPoint(B); plotSegment(A,B); plotLine(l):
\end{eulerprompt}
\eulerimg{27}{images/Nur Alya Fadilah_Aplikom-204.png}
\begin{eulercomment}
Selanjutnya, kami melakukan hal yang sama di Maxima dengan koordinat
umum.
\end{eulercomment}
\begin{eulerprompt}
>A &= [a1,a2]; B &= [b1,b2];
>c1 &= circleWithCenter(A,distance(A,B));
>c2 &= circleWithCenter(B,distance(A,B));
>P &= circleCircleIntersections(c1,c2); P1 &= P[1]; P2 &= P[2];
\end{eulerprompt}
\begin{eulercomment}
Persamaan untuk persimpangan cukup terlibat. Tetapi kita dapat
menyederhanakannya, jika kita memecahkan y.
\end{eulercomment}
\begin{eulerprompt}
>g &= getLineEquation(lineThrough(P1,P2),x,y);
>$solve(g,y)
\end{eulerprompt}
\begin{eulerformula}
\[
\left[ y=\frac{-\left(2\,{\it b_1}-2\,{\it a_1}\right)\,x+{\it b_2}
 ^2+{\it b_1}^2-{\it a_2}^2-{\it a_1}^2}{2\,{\it b_2}-2\,{\it a_2}}
  \right] 
\]
\end{eulerformula}
\begin{eulercomment}
Ini memang sama dengan tegak lurus tengah, yang dihitung dengan cara
yang sama sekali berbeda.
\end{eulercomment}
\begin{eulerprompt}
>$solve(getLineEquation(middlePerpendicular(A,B),x,y),y)
\end{eulerprompt}
\begin{eulerformula}
\[
\left[ y=\frac{-\left(2\,{\it b_1}-2\,{\it a_1}\right)\,x+{\it b_2}
 ^2+{\it b_1}^2-{\it a_2}^2-{\it a_1}^2}{2\,{\it b_2}-2\,{\it a_2}}
  \right] 
\]
\end{eulerformula}
\begin{eulerprompt}
>h &=getLineEquation(lineThrough(A,B),x,y);
>$solve(h,y)
\end{eulerprompt}
\begin{eulerformula}
\[
\left[ y=\frac{\left({\it b_2}-{\it a_2}\right)\,x-{\it a_1}\,
 {\it b_2}+{\it a_2}\,{\it b_1}}{{\it b_1}-{\it a_1}} \right] 
\]
\end{eulerformula}
\begin{eulercomment}
Perhatikan hasil kali gradien garis g dan h adalah:

\end{eulercomment}
\begin{eulerformula}
\[
\frac{-(b_1-a_1)}{(b_2-a_2)}\times \frac{(b_2-a_2)}{(b_1-a_1)} = -1.
\]
\end{eulerformula}
\begin{eulercomment}
Artinya kedua garis tegak lurus.
\end{eulercomment}
\eulersubheading{Jarak Minimal pada Bidang}
\begin{eulercomment}
Fungsi yang, ke titik M di bidang, menetapkan jarak AM antara titik
tetap A dan M, memiliki garis level yang agak sederhana: lingkaran
berpusat di A.
\end{eulercomment}
\begin{eulerprompt}
>&remvalue();
>A=[-1,-1];
>function d1(x,y):=sqrt((x-A[1])^2+(y-A[2])^2)
>fcontour("d1",xmin=-2,xmax=0,ymin=-2,ymax=0,hue=1, ...
>title="If you see ellipses, please set your window square"):
\end{eulerprompt}
\eulerimg{27}{images/Nur Alya Fadilah_Aplikom-209.png}
\begin{eulercomment}
dan grafiknya juga agak sederhana: bagian atas kerucut:
\end{eulercomment}
\begin{eulerprompt}
>plot3d("d1",xmin=-2,xmax=0,ymin=-2,ymax=0):
\end{eulerprompt}
\eulerimg{27}{images/Nur Alya Fadilah_Aplikom-210.png}
\begin{eulercomment}
b. Hal hal yang dilakukan dalam mempelajari materi\\
- Mencari informasi mengenai materi geometri.\\
- mencarai latihan soal di buku dan internet.\\
- Mempelajari perintah perintah yang ada di EMT berkaitan dengan
geometri.

c. Kendala kendala dan usaha untuk mengatasi kendala tersebut\\
- Kesulitan dalam memahami perintah yang berkaitan dengan geometri,
solusinya dengan menonton youtube dan mempelajari rumus rumus
geometri.\\
\end{eulercomment}
\eulersubheading{}
\eulerheading{7. Penggunaan software EMT untuk aplikasi Kalkulus}
\begin{eulercomment}
a. Hal hal yang dipelajari beserta contohnya\\
- Fungsi (fungsi aljabar, trigonometri, eksponensial, logaritma,
komposisi fungsi)\\
- Limit Fungsi,\\
- Turunan Fungsi,\\
- Integral Tak Tentu,\\
- Integral Tentu dan Aplikasinya,\\
- Barisan dan Deret (kekonvergenan barisan dan deret).


\end{eulercomment}
\eulersubheading{Mendefinisikan Fungsi}
\begin{eulercomment}
Terdapat beberapa cara mendefinisikan fungsi pada EMT, yakni:

- Menggunakan format nama\_fungsi := rumus fungsi (untuk fungsi
numerik),\\
- Menggunakan format nama\_fungsi \&= rumus fungsi (untuk fungsi
simbolik, namun dapat dihitung secara numerik),\\
- Menggunakan format nama\_fungsi \&\&= rumus fungsi (untuk fungsi
simbolik murni, tidak dapat dihitung langsung),\\
- Fungsi sebagai program EMT.

Setiap format harus diawali dengan perintah function (bukan sebagai
ekspresi).

Berikut adalah adalah beberapa contoh cara mendefinisikan fungsi.
\end{eulercomment}
\begin{eulerprompt}
>function f(x) &= 2*E^x // fungsi simbolik
\end{eulerprompt}
\begin{euleroutput}
  
                                      x
                                   2 E
  
\end{euleroutput}
\begin{eulerprompt}
>function g(x) &= 3*x+1
\end{eulerprompt}
\begin{euleroutput}
  
                                 3 x + 1
  
\end{euleroutput}
\begin{eulerprompt}
>function h(x) &= f(g(x)) // komposisi fungsi
\end{eulerprompt}
\begin{euleroutput}
  
                                   3 x + 1
                                2 E
  
\end{euleroutput}
\begin{eulercomment}
1. Untuk fungsi\\
\end{eulercomment}
\begin{eulerformula}
\[
k(x) = x^2-4
\]
\end{eulerformula}
\begin{eulercomment}
tentukan nilai\\
a. k(-4)\\
b. k(4)
\end{eulercomment}
\begin{eulerprompt}
>function k(x) := x^2 -4
>k(-4), k(4)
\end{eulerprompt}
\begin{euleroutput}
  12
  12
\end{euleroutput}
\begin{eulerprompt}
>plot2d("k",-4,4):
\end{eulerprompt}
\eulerimg{27}{images/Nur Alya Fadilah_Aplikom-212.png}
\begin{eulercomment}
2. Untuk fungsi\\
\end{eulercomment}
\begin{eulerformula}
\[
z(x) = \frac{x^2-16}{x-4}
\]
\end{eulerformula}
\begin{eulercomment}
hitunglah masing-masing nilai.\\
a. z(6)\\
b. z(2)
\end{eulercomment}
\begin{eulerprompt}
>function z(x) := (x^2-16)/(x-4)
>z(6), z(2)
\end{eulerprompt}
\begin{euleroutput}
  10
  6
\end{euleroutput}
\begin{eulerprompt}
>plot2d("z",-4,6):
\end{eulerprompt}
\eulerimg{27}{images/Nur Alya Fadilah_Aplikom-214.png}
\begin{eulercomment}
3. Untuk fungsi\\
\end{eulercomment}
\begin{eulerformula}
\[
r(x) = x^3 - 3x^2 + 2x - 4
\]
\end{eulerformula}
\begin{eulercomment}
tentukan nilai r(4), r(-6), r(8)
\end{eulercomment}
\begin{eulerprompt}
>function f(x) := x^3-3*x^2+2*x-4
>f(4), f(-6), f(8)
\end{eulerprompt}
\begin{euleroutput}
  20
  -340
  332
\end{euleroutput}
\begin{eulerprompt}
>plot2d("x^3-3*x^2+2*x-4",-2,9):
\end{eulerprompt}
\eulerimg{27}{images/Nur Alya Fadilah_Aplikom-216.png}
\begin{eulercomment}
4. Tentukan nilai f(200) dari fungsi berikut

\end{eulercomment}
\begin{eulerformula}
\[
f(x) = \sqrt{x-64}
\]
\end{eulerformula}
\begin{eulerprompt}
>function f(x) := sqrt(x-64)
>f(200)
\end{eulerprompt}
\begin{euleroutput}
  11.6619037897
\end{euleroutput}
\begin{eulerprompt}
>plot2d("sqrt(x-64)",0,200):
\end{eulerprompt}
\eulerimg{27}{images/Nur Alya Fadilah_Aplikom-218.png}
\begin{eulercomment}
5. Untuk fungsi\\
\end{eulercomment}
\begin{eulerformula}
\[
f(x) = x^2-3x+2
\]
\end{eulerformula}
\begin{eulerformula}
\[
dan
\]
\end{eulerformula}
\begin{eulerformula}
\[
g(x) = x+3
\]
\end{eulerformula}
\begin{eulercomment}
cari nilai fog(-4), gof(0)
\end{eulercomment}
\begin{eulerprompt}
>function f(x) := x^2-3*x+2; $f(x)
>function g(x) := x+3; $g(x)
>f(g(-4)), g(f(0))
\end{eulerprompt}
\begin{euleroutput}
  6
  5
\end{euleroutput}
\begin{eulerprompt}
>plot2d("(x+3)^2-3*(x+3)+2",-2,2):
\end{eulerprompt}
\eulerimg{27}{images/Nur Alya Fadilah_Aplikom-222.png}
\begin{eulercomment}
6. Tentukan nilai dari\\
\end{eulercomment}
\begin{eulerformula}
\[
f(x,y):=x^2+y^2+2x-2y+1
\]
\end{eulerformula}
\begin{eulercomment}
dengan x=1 dan y=3
\end{eulercomment}
\begin{eulerprompt}
>function f(x,y):= x^2+y^2+2*x-2*y+1
>f(1,3)
\end{eulerprompt}
\begin{euleroutput}
  7
\end{euleroutput}
\begin{eulerprompt}
>plot3d("x^2+y^2+2*x-2*y+1"):
\end{eulerprompt}
\eulerimg{27}{images/Nur Alya Fadilah_Aplikom-224.png}
\eulerheading{Menghitung Limit}
\begin{eulercomment}
Perhitungan limit pada EMT dapat dilakukan dengan menggunakan fungsi
Maxima, yakni "limit". Fungsi "limit" dapat digunakan untuk menghitung
limit fungsi dalam bentuk ekspresi maupun fungsi yang sudah
didefinisikan sebelumnya. Nilai limit dapat dihitung pada sebarang
nilai atau pada tak hingga (-inf, minf, dan inf). Limit kiri dan limit
kanan juga dapat dihitung, dengan cara memberi opsi "plus" atau
"minus". Hasil limit dapat berupa nilai, "und' (tak definisi), "ind"
(tak tentu namun terbatas), "infinity" (kompleks tak hingga).
\end{eulercomment}
\begin{eulerprompt}
>$showev('limit(1/(2*x-1),x,0))
\end{eulerprompt}
\begin{eulerformula}
\[
\lim_{x\rightarrow 0}{\frac{1}{2\,x-1}}=-1
\]
\end{eulerformula}
\begin{eulerprompt}
>$showev('limit((x^2-3*x-10)/(x-5),x,5))
\end{eulerprompt}
\begin{eulerformula}
\[
\lim_{x\rightarrow 5}{\frac{x^2-3\,x-10}{x-5}}=7
\]
\end{eulerformula}
\begin{eulerprompt}
>$showev('limit(sin(x)/x,x,0))
\end{eulerprompt}
\begin{eulerformula}
\[
\lim_{x\rightarrow 0}{\frac{\sin x}{x}}=1
\]
\end{eulerformula}
\begin{eulerprompt}
>plot2d("sin(x)/x",-pi,pi):
\end{eulerprompt}
\eulerimg{27}{images/Nur Alya Fadilah_Aplikom-228.png}
\begin{eulerprompt}
>$showev('limit(sin(x^3)/x,x,0))
\end{eulerprompt}
\begin{eulerformula}
\[
\lim_{x\rightarrow 0}{\frac{\sin x^3}{x}}=0
\]
\end{eulerformula}
\begin{eulerprompt}
>$showev('limit(log(x), x, minf))
\end{eulerprompt}
\begin{eulerformula}
\[
\lim_{x\rightarrow  -\infty }{\log x}={\it infinity}
\]
\end{eulerformula}
\begin{eulerprompt}
>$showev('limit((-2)^x,x, inf))
\end{eulerprompt}
\begin{eulerformula}
\[
\lim_{x\rightarrow \infty }{\left(-2\right)^{x}}={\it infinity}
\]
\end{eulerformula}
\begin{eulerprompt}
>$showev('limit(t-sqrt(2-t),t,2,minus))
\end{eulerprompt}
\begin{eulerformula}
\[
\lim_{t\uparrow 2}{t-\sqrt{2-t}}=2
\]
\end{eulerformula}
\begin{eulerprompt}
>$showev('limit(t-sqrt(2-t),t,5,plus)) // Perhatikan hasilnya
\end{eulerprompt}
\begin{eulerformula}
\[
\lim_{t\downarrow 5}{t-\sqrt{2-t}}=5-\sqrt{3}\,i
\]
\end{eulerformula}
\begin{eulerprompt}
>plot2d("x-sqrt(2-x)",-2,5):
\end{eulerprompt}
\eulerimg{27}{images/Nur Alya Fadilah_Aplikom-234.png}
\begin{eulerprompt}
>$showev('limit((x^2-9)/(2*x^2-5*x-3),x,3))
\end{eulerprompt}
\begin{eulerformula}
\[
\lim_{x\rightarrow 3}{\frac{x^2-9}{2\,x^2-5\,x-3}}=\frac{6}{7}
\]
\end{eulerformula}
\begin{eulerprompt}
>$showev('limit((1-cos(x))/x,x,0))
\end{eulerprompt}
\begin{eulerformula}
\[
\lim_{x\rightarrow 0}{\frac{1-\cos x}{x}}=0
\]
\end{eulerformula}
\begin{eulerprompt}
>$showev('limit((x^2+abs(x))/(x^2-abs(x)),x,0))
\end{eulerprompt}
\begin{eulerformula}
\[
\lim_{x\rightarrow 0}{\frac{\left| x\right| +x^2}{x^2-\left| x
 \right| }}=-1
\]
\end{eulerformula}
\begin{eulerprompt}
>$showev('limit((1+1/x)^x,x,inf))
\end{eulerprompt}
\begin{eulerformula}
\[
\lim_{x\rightarrow \infty }{\left(\frac{1}{x}+1\right)^{x}}=e
\]
\end{eulerformula}
\begin{eulerprompt}
>$showev('limit((1+k/x)^x,x,inf))
\end{eulerprompt}
\begin{eulerformula}
\[
\lim_{x\rightarrow \infty }{\left(\frac{k}{x}+1\right)^{x}}=e^{k}
\]
\end{eulerformula}
\begin{eulerprompt}
>$showev('limit((1+x)^(1/x),x,0))
\end{eulerprompt}
\begin{eulerformula}
\[
\lim_{x\rightarrow 0}{\left(x+1\right)^{\frac{1}{x}}}=e
\]
\end{eulerformula}
\begin{eulerprompt}
>$showev('limit((x/(x+k))^x,x,inf))
\end{eulerprompt}
\begin{eulerformula}
\[
\lim_{x\rightarrow \infty }{\left(\frac{x}{x+k}\right)^{x}}=e^ {- k
  }
\]
\end{eulerformula}
\begin{eulerprompt}
>$showev('limit(sin(1/x),x,0))
\end{eulerprompt}
\begin{eulerformula}
\[
\lim_{x\rightarrow 0}{\sin \left(\frac{1}{x}\right)}={\it ind}
\]
\end{eulerformula}
\begin{eulerprompt}
>$showev('limit(sin(1/x),x,inf))
\end{eulerprompt}
\begin{eulerformula}
\[
\lim_{x\rightarrow \infty }{\sin \left(\frac{1}{x}\right)}=0
\]
\end{eulerformula}
\begin{eulerprompt}
>plot2d("sin(1/x)",-5,5):
\end{eulerprompt}
\eulerimg{27}{images/Nur Alya Fadilah_Aplikom-244.png}
\begin{eulerprompt}
>function f(x) &= x^x
\end{eulerprompt}
\begin{euleroutput}
  
                                     x
                                    x
  
\end{euleroutput}
\begin{eulerprompt}
>&forget(x>0) // jangan lupa, lupakan asumsi untuk kembali ke semula
\end{eulerprompt}
\begin{euleroutput}
  
                                 [x > 0]
  
\end{euleroutput}
\begin{eulerprompt}
>&forget(x<0)
\end{eulerprompt}
\begin{euleroutput}
  
                                 [x < 0]
  
\end{euleroutput}
\begin{eulerprompt}
>function f(x) &= sinh(x) // definisikan f(x)=sinh(x)
\end{eulerprompt}
\begin{euleroutput}
  
                                 sinh(x)
  
\end{euleroutput}
\eulerheading{Integral}
\begin{eulercomment}
EMT dapat digunakan untuk menghitung integral, baik integral tak tentu
maupun integral tentu. Untuk integral tak tentu (simbolik) sudah tentu
EMT menggunakan Maxima, sedangkan untuk perhitungan integral tentu EMT
sudah menyediakan beberapa fungsi yang mengimplementasikan algoritma
kuadratur (perhitungan integral tentu menggunakan metode numerik).

\end{eulercomment}
\begin{eulerformula}
\[
\int_a^b f(x)\ dx = F(b)-F(a), \quad \text{ dengan  } F'(x) = f(x).
\]
\end{eulerformula}
\begin{eulercomment}
Fungsi untuk menentukan integral adalah integrate. Fungsi ini dapat
digunakan untuk menentukan, baik integral tentu maupun tak tentu (jika
fungsinya memiliki antiderivatif). Untuk perhitungan integral tentu
fungsi integrate menggunakan metode numerik (kecuali fungsinya tidak
integrabel, kita tidak akan menggunakan metode ini).
\end{eulercomment}
\begin{eulerprompt}
>$showev('integrate(x^n,x))
\end{eulerprompt}
\begin{euleroutput}
  Answering "Is n equal to -1?" with "no"
\end{euleroutput}
\begin{eulerformula}
\[
\int {x^{n}}{\;dx}=\frac{x^{n+1}}{n+1}
\]
\end{eulerformula}
\begin{eulerprompt}
>$showev('integrate(1/(1+x),x))
\end{eulerprompt}
\begin{eulerformula}
\[
\int {\frac{1}{x+1}}{\;dx}=\log \left(x+1\right)
\]
\end{eulerformula}
\begin{eulerprompt}
>$showev('integrate(1/(1+x^2),x))
\end{eulerprompt}
\begin{eulerformula}
\[
\int {\frac{1}{x^2+1}}{\;dx}=\arctan x
\]
\end{eulerformula}
\begin{eulerprompt}
>$showev('integrate(1/sqrt(1-x^2),x))
\end{eulerprompt}
\begin{eulerformula}
\[
\int {\frac{1}{\sqrt{1-x^2}}}{\;dx}=\arcsin x
\]
\end{eulerformula}
\begin{eulerprompt}
>$showev('integrate(sin(x),x,0,pi))
\end{eulerprompt}
\begin{eulerformula}
\[
\int_{0}^{\pi}{\sin x\;dx}=2
\]
\end{eulerformula}
\begin{eulerprompt}
>$showev('integrate(sin(x),x,a,b))
\end{eulerprompt}
\begin{eulerformula}
\[
\int_{a}^{b}{\sin x\;dx}=\cos a-\cos b
\]
\end{eulerformula}
\begin{eulerprompt}
>$showev('integrate(x^n,x,a,b))
\end{eulerprompt}
\begin{euleroutput}
  Answering "Is n positive, negative or zero?" with "positive"
\end{euleroutput}
\begin{eulerformula}
\[
\int_{a}^{b}{x^{n}\;dx}=\frac{b^{n+1}}{n+1}-\frac{a^{n+1}}{n+1}
\]
\end{eulerformula}
\begin{eulerprompt}
>$showev('integrate(x^2*sqrt(2*x+1),x))
\end{eulerprompt}
\begin{eulerformula}
\[
\int {x^2\,\sqrt{2\,x+1}}{\;dx}=\frac{\left(2\,x+1\right)^{\frac{7
 }{2}}}{28}-\frac{\left(2\,x+1\right)^{\frac{5}{2}}}{10}+\frac{\left(
 2\,x+1\right)^{\frac{3}{2}}}{12}
\]
\end{eulerformula}
\begin{eulerprompt}
>$showev('integrate(x^2*sqrt(2*x+1),x,0,2))
\end{eulerprompt}
\begin{eulerformula}
\[
\int_{0}^{2}{x^2\,\sqrt{2\,x+1}\;dx}=\frac{2\,5^{\frac{5}{2}}}{21}-
 \frac{2}{105}
\]
\end{eulerformula}
\begin{eulerprompt}
>$ratsimp(%)
\end{eulerprompt}
\begin{eulerformula}
\[
\int_{0}^{2}{x^2\,\sqrt{2\,x+1}\;dx}=\frac{2\,5^{\frac{7}{2}}-2}{
 105}
\]
\end{eulerformula}
\begin{eulerprompt}
>$showev('integrate((sin(sqrt(x)+a)*E^sqrt(x))/sqrt(x),x,0,pi^2))
\end{eulerprompt}
\begin{eulerformula}
\[
\int_{0}^{\pi^2}{\frac{\sin \left(\sqrt{x}+a\right)\,e^{\sqrt{x}}}{
 \sqrt{x}}\;dx}=\left(-e^{\pi}-1\right)\,\sin a+\left(e^{\pi}+1
 \right)\,\cos a
\]
\end{eulerformula}
\begin{eulerprompt}
>$factor(%)
\end{eulerprompt}
\begin{eulerformula}
\[
\int_{0}^{\pi^2}{\frac{\sin \left(\sqrt{x}+a\right)\,e^{\sqrt{x}}}{
 \sqrt{x}}\;dx}=\left(-e^{\pi}-1\right)\,\left(\sin a-\cos a\right)
\]
\end{eulerformula}
\begin{eulerprompt}
>function map f(x) &= E^(-x^2)
\end{eulerprompt}
\begin{euleroutput}
  
                                      2
                                   - x
                                  E
  
\end{euleroutput}
\begin{eulerprompt}
>$showev('integrate(f(x),x))
\end{eulerprompt}
\begin{eulerformula}
\[
\int {e^ {- x^2 }}{\;dx}=\frac{\sqrt{\pi}\,\mathrm{erf}\left(x
 \right)}{2}
\]
\end{eulerformula}
\begin{eulercomment}
Fungsi f tidak memiliki antiturunan, integralnya masih memuat integral
lain.

\end{eulercomment}
\begin{eulerformula}
\[
erf(x) = \int \frac{e^{-x^2}}{\sqrt{\pi}} \ dx.
\]
\end{eulerformula}
\begin{eulercomment}
Kita tidak dapat menggunakan teorema Dasar kalkulus untuk menghitung
integral tentu fungsi tersebut jika semua batasnya berhingga. Dalam
hal ini dapat digunakan metode numerik (rumus kuadratur).

Misalkan kita akan menghitung:

\end{eulercomment}
\begin{eulerformula}
\[
\int_{0}^{\pi}{x^{x}\;dx}
\]
\end{eulerformula}
\begin{eulerprompt}
>x=0:0.1:pi-0.1; plot2d(x,f(x+0.1),>bar); plot2d("f(x)",0,pi,>add):
\end{eulerprompt}
\eulerimg{27}{images/Nur Alya Fadilah_Aplikom-261.png}
\begin{eulercomment}
Integral tentu

maxima: 'integrate(f(x),x,0,pi)

dapat dihampiri dengan jumlah luas persegi-persegi panjang di bawah
kurva y=f(x) tersebut. Langkah-langkahnya adalah sebagai berikut.
\end{eulercomment}
\begin{eulerprompt}
>t &= makelist(a,a,0,pi-0.1,0.1); // t sebagai list untuk menyimpan nilai-nilai x
>fx &= makelist(f(t[i]+0.1),i,1,length(t)); // simpan nilai-nilai f(x)
>// jangan menggunakan x sebagai list, kecuali Anda pakar Maxima!
\end{eulerprompt}
\begin{eulercomment}
Hasilnya adalah:

maxima: 'integrate(f(x),x,0,pi) = 0.1*sum(fx[i],i,1,length(fx))

Jumlah tersebut diperoleh dari hasil kali lebar sub-subinterval (=0.1)
dan jumlah nilai-nilai f(x) untuk x = 0.1, 0.2, 0.3, ..., 3.2.
\end{eulercomment}
\begin{eulerprompt}
>0.1*sum(f(x+0.1)) // cek langsung dengan perhitungan numerik EMT
\end{eulerprompt}
\begin{euleroutput}
  0.836219610253
\end{euleroutput}
\begin{eulercomment}
Untuk mendapatkan nilai integral tentu yang mendekati nilai sebenarnya, lebar
sub-intervalnya dapat diperkecil lagi, sehingga daerah di bawah kurva tertutup
semuanya, misalnya dapat digunakan lebar subinterval 0.001. (Silakan dicoba!)

Meskipun Maxima tidak dapat menghitung integral tentu fungsi tersebut untuk
batas-batas yang berhingga, namun integral tersebut dapat dihitung secara eksak jika
batas-batasnya tak hingga. Ini adalah salah satu keajaiban di dalam matematika, yang
terbatas tidak dapat dihitung secara eksak, namun yang tak hingga malah dapat
dihitung secara eksak.
\end{eulercomment}
\begin{eulerprompt}
>$showev('integrate(f(x),x,0,inf))
\end{eulerprompt}
\begin{eulerformula}
\[
\int_{0}^{\infty }{e^ {- x^2 }\;dx}=\frac{\sqrt{\pi}}{2}
\]
\end{eulerformula}
\begin{eulercomment}
Berikut adalah contoh lain fungsi yang tidak memiliki antiderivatif, sehingga
integral tentunya hanya dapat dihitung dengan metode numerik.
\end{eulercomment}
\begin{eulerprompt}
>function f(x) &= x^x
\end{eulerprompt}
\begin{euleroutput}
  
                                     x
                                    x
  
\end{euleroutput}
\begin{eulerprompt}
>$showev('integrate(f(x),x,0,1))
\end{eulerprompt}
\begin{eulerformula}
\[
\int_{0}^{1}{x^{x}\;dx}=\int_{0}^{1}{x^{x}\;dx}
\]
\end{eulerformula}
\begin{eulerprompt}
>x=0:0.1:1-0.01; plot2d(x,f(x+0.01),>bar); plot2d("f(x)",0,1,>add):
\end{eulerprompt}
\eulerimg{27}{images/Nur Alya Fadilah_Aplikom-264.png}
\begin{eulercomment}
Maxima gagal menghitung integral tentu tersebut secara langsung menggunakan perintah
integrate. Berikut kita lakukan seperti contoh sebelumnya untuk mendapat hasil atau
pendekatan nilai integral tentu tersebut.
\end{eulercomment}
\begin{eulerprompt}
>t &= makelist(a,a,0,1-0.01,0.01);
>fx &= makelist(f(t[i]+0.01),i,1,length(t));
\end{eulerprompt}
\eulersubheading{Barisan dan Deret}
\begin{eulercomment}
Barisan dapat didefinisikan dengan beberapa cara di dalam EMT, di
antaranya:

- dengan cara yang sama seperti mendefinisikan vektor dengan
elemen-elemen beraturan (menggunakan titik dua ":");\\
- menggunakan perintah "sequence" dan rumus barisan (suku ke -n);\\
- menggunakan perintah "iterate" atau "niterate";\\
- menggunakan fungsi Maxima "create\_list" atau "makelist" untuk
menghasilkan barisan simbolik;\\
- menggunakan fungsi biasa yang inputnya vektor atau barisan;\\
- menggunakan fungsi rekursif.

EMT menyediakan beberapa perintah (fungsi) terkait barisan, yakni:

- sum: menghitung jumlah semua elemen suatu barisan\\
- cumsum: jumlah kumulatif suatu barisan\\
- differences: selisih antar elemen-elemen berturutan

EMT juga dapat digunakan untuk menghitung jumlah deret berhingga
maupun deret tak hingga, dengan menggunakan perintah (fungsi) "sum".
Perhitungan dapat dilakukan secara numerik maupun simbolik dan eksak.

Berikut adalah beberapa contoh perhitungan barisan dan deret
menggunakan EMT.
\end{eulercomment}
\begin{eulerprompt}
>1:10 // barisan sederhana
\end{eulerprompt}
\begin{euleroutput}
  [1,  2,  3,  4,  5,  6,  7,  8,  9,  10]
\end{euleroutput}
\begin{eulerprompt}
>1:2:30
\end{eulerprompt}
\begin{euleroutput}
  [1,  3,  5,  7,  9,  11,  13,  15,  17,  19,  21,  23,  25,  27,  29]
\end{euleroutput}
\begin{eulerprompt}
>sum(1:2:30), sum(1/(1:2:30))
\end{eulerprompt}
\begin{euleroutput}
  225
  2.33587263431
\end{euleroutput}
\begin{eulerprompt}
>$'sum(k, k, 1, n) = factor(ev(sum(k, k, 1, n),simpsum=true)) // simpsum:menghitung deret secara simbolik
\end{eulerprompt}
\begin{eulerformula}
\[
\sum_{k=1}^{n}{k}=\frac{n\,\left(n+1\right)}{2}
\]
\end{eulerformula}
\begin{eulerprompt}
>$'sum(1/(3^k+k), k, 0, inf) = factor(ev(sum(1/(3^k+k), k, 0, inf),simpsum=true))
\end{eulerprompt}
\begin{eulerformula}
\[
\sum_{k=0}^{\infty }{\frac{1}{3^{k}+k}}=\sum_{k=0}^{\infty }{\frac{
 1}{3^{k}+k}}
\]
\end{eulerformula}
\begin{eulercomment}
Di sini masih gagal, hasilnya tidak dihitung.
\end{eulercomment}
\begin{eulerprompt}
>$'sum(1/x^2, x, 1, inf)= ev(sum(1/x^2, x, 1, inf),simpsum=true) // ev: menghitung nilai ekspresi
\end{eulerprompt}
\begin{eulerformula}
\[
\sum_{x=1}^{\infty }{\frac{1}{x^2}}=\frac{\pi^2}{6}
\]
\end{eulerformula}
\begin{eulerprompt}
>$'sum((-1)^(k-1)/k, k, 1, inf) = factor(ev(sum((-1)^(x-1)/x, x, 1, inf),simpsum=true))
\end{eulerprompt}
\begin{eulerformula}
\[
\sum_{k=1}^{\infty }{\frac{\left(-1\right)^{k-1}}{k}}=-\sum_{x=1}^{
 \infty }{\frac{\left(-1\right)^{x}}{x}}
\]
\end{eulerformula}
\begin{eulercomment}
Di sini masih gagal, hasilnya tidak dihitung.
\end{eulercomment}
\begin{eulerprompt}
>$'sum((-1)^k/(2*k-1), k, 1, inf) = factor(ev(sum((-1)^k/(2*k-1), k, 1, inf),simpsum=true))
\end{eulerprompt}
\begin{eulerformula}
\[
\sum_{k=1}^{\infty }{\frac{\left(-1\right)^{k}}{2\,k-1}}=\sum_{k=1
 }^{\infty }{\frac{\left(-1\right)^{k}}{2\,k-1}}
\]
\end{eulerformula}
\begin{eulerprompt}
>$ev(sum(1/n!, n, 0, inf),simpsum=true)
\end{eulerprompt}
\begin{eulerformula}
\[
\sum_{n=0}^{\infty }{\frac{1}{n!}}
\]
\end{eulerformula}
\begin{eulercomment}
Di sini masih gagal, hasilnya tidak dihitung, harusnya hasilnya e.
\end{eulercomment}
\begin{eulerprompt}
>&assume(abs(x)<1); $'sum(a*x^k, k, 0, inf)=ev(sum(a*x^k, k, 0, inf),simpsum=true), &forget(abs(x)<1);
\end{eulerprompt}
\begin{eulerformula}
\[
a\,\sum_{k=0}^{\infty }{x^{k}}=\frac{a}{1-x}
\]
\end{eulerformula}
\begin{eulercomment}
Deret geometri tak hingga, dengan asumsi rasional antara -1 dan 1.

b. Hal hal yang dilakukan dalam mempelajari materi\\
- Mencari informasi mengenai materi kalkulus.\\
- mencarai latihan soal di buku dan internet.\\
- Mempelajari perintah perintah yang ada di EMT berkaitan dengan
kalkulus.\\
c. Kendala kendala dan usaha untuk mengatasi kendala tersebut\\
- Kesulitan dalam memahami perintah yang berkaitan dengan kalkulus,
solusinya dengan menonton youtube dan mempelajari rumus rumus
kalkulus.\\
\end{eulercomment}
\eulersubheading{}
\begin{eulerprompt}
> 
\end{eulerprompt}
\begin{eulercomment}
\begin{eulercomment}
\eulerheading{8. Penggunaan software EMT untuk aplikasi Statistika}
\begin{eulercomment}
a. Hal hal yang dipelajari beserta contohnya\\
- Menyimpan data\\
- Menghasilkan data acak\\
- Membaca data yang tersimpan\\
- Perhitungan analisis data statistika\\
- Membaca grafik statistika\\
- Menyimpan data hasil analisis

\end{eulercomment}
\eulersubheading{Menyimpan Data Dalam Bentuk Matriks}
\begin{eulercomment}
Array\\
Array adalah kumpulan-kumpulan variabel yang menyimpan data dengan
tipe yang sama atau data-data yang tersusun secara linear dimana di
dalamnya terdapat elemen dengan tipe yang sama.

Vektor digunakan untuk menggambarkan array angka satu dimensi. Vektor
memiliki panjang, yang merupakan jumlah elemen dalam array.

Sedangkan matriks digunakan dalam mendeskripsikan susunan bilangan dua\\
dimensi yang disusun dalam baris dan kolom. matriks memiliki ukuran,
yaitu jumlah baris dan kolom.

Hubungan antara array dan matriks adalah bahwa matriks adalah bentuk
khusus dari array. Array dapat memiliki lebih dari dua dimensi, tetapi
matriks selalu memiliki dua dimensi. Dalam pemrograman, array dan
matriks sering digunakan untuk menyimpan data dalam jumlah besar dan
memudahkan pengaksesan data tersebut.

Mari kita bahas beberapa hal terkait vektor terlebih dahulu

\end{eulercomment}
\begin{eulerprompt}
>v=shuffle(1:10)
\end{eulerprompt}
\begin{euleroutput}
  [5,  7,  9,  2,  6,  1,  8,  4,  3,  10]
\end{euleroutput}
\begin{eulerprompt}
>w=intrandom(10,12)
\end{eulerprompt}
\begin{euleroutput}
  [1,  12,  12,  1,  8,  11,  11,  8,  1,  10]
\end{euleroutput}
\begin{eulercomment}
Untuk mengurutkan angka acak 
\end{eulercomment}
\begin{eulerprompt}
>sort(v)
\end{eulerprompt}
\begin{euleroutput}
  [1,  2,  3,  4,  5,  6,  7,  8,  9,  10]
\end{euleroutput}
\begin{eulercomment}
Selanjutnya mengurutkan angka acak dengan menyederhanakan angka yang
sama
\end{eulercomment}
\begin{eulerprompt}
>unique(v)
\end{eulerprompt}
\begin{euleroutput}
  [1,  2,  3,  4,  5,  6,  7,  8,  9,  10]
\end{euleroutput}
\begin{eulercomment}
Menemukan banyaknya setiap elemen dengan bantuan interval
\end{eulercomment}
\begin{eulerprompt}
>s=intrandom(10,20)
\end{eulerprompt}
\begin{euleroutput}
  [19,  17,  17,  11,  5,  7,  14,  6,  18,  4]
\end{euleroutput}
\begin{eulerprompt}
>x=[5,10,15,20]
\end{eulerprompt}
\begin{euleroutput}
  [5,  10,  15,  20]
\end{euleroutput}
\begin{eulerprompt}
>find(x,s)
\end{eulerprompt}
\begin{euleroutput}
  [3,  3,  3,  2,  1,  1,  2,  1,  3,  0]
\end{euleroutput}
\begin{eulercomment}
Berikutnya adalah cara mencari indeks dari sebuah vektor dengan contoh
vekUntuk indeks pada EMT berbeda dengan indeks pada Phyton yang kita
pelajari sebelumnya di Algoritma dan pemrograman. Perbedaannya juka
sebelumnya untk menentukan indeks akan dimulalai dari nol namun di
mmenentukan indeks di EMT akan dimulai dari angka satu, berikut
penjelasannya
\end{eulercomment}
\begin{eulerprompt}
>indexof(w,1:10)
\end{eulerprompt}
\begin{euleroutput}
  [1,  0,  0,  0,  0,  0,  0,  5,  0,  10]
\end{euleroutput}
\begin{eulerprompt}
>x= sort(intrandom(10,12))
\end{eulerprompt}
\begin{euleroutput}
  [3,  4,  6,  7,  8,  8,  9,  10,  11,  11]
\end{euleroutput}
\begin{eulerprompt}
>indexofsorted(x,1:15)
\end{eulerprompt}
\begin{euleroutput}
  [0,  0,  1,  2,  0,  3,  4,  6,  7,  8,  10,  0,  0,  0,  0]
\end{euleroutput}
\begin{eulerprompt}
>z=intrandom(1000,10); multofsorted(sort(z),1:10), sum(%)
\end{eulerprompt}
\begin{euleroutput}
  [79,  94,  101,  111,  87,  98,  107,  111,  95,  117]
  1000
\end{euleroutput}
\begin{eulercomment}
Sampai disini pembahasan terkait dnegan vektor\\
Selanjutnya kita akan membahas  beberapa hal terkait matriks terkait

Untuk Menyimpan Data dalam bentuk Matrik

Pertama, buat sebuah variabel yang akan menampung data matrik, misal
X. Variabel ini bebas dengan syarat tidak sama dengan nama fungsi atau
konstanta yang sudah ada dalam software.

Selanjutnya,kita akan membuat matrik berordo mxn yang berisi angka
\end{eulercomment}
\begin{eulerprompt}
>X=[1,2,3,4;4,5,6,7;8,4,4,6]
\end{eulerprompt}
\begin{euleroutput}
              1             2             3             4 
              4             5             6             7 
              8             4             4             6 
\end{euleroutput}
\begin{eulerprompt}
>shortformat; A=random(3,4)
\end{eulerprompt}
\begin{euleroutput}
    0.31232   0.92519  0.014936   0.30395 
    0.41988  0.042611   0.93098   0.98727 
     0.2075   0.42556   0.42418   0.49088 
\end{euleroutput}
\begin{eulerprompt}
>shortformat; A=intrandom(5,4,20)
\end{eulerprompt}
\begin{euleroutput}
          6         2        20         2 
         20        12        17         3 
         16         8        14        10 
          3         5        16        14 
         20        14        17        11 
\end{euleroutput}
\begin{eulerprompt}
>shortformat; A=redim(1:15,4,4)
\end{eulerprompt}
\begin{euleroutput}
          1         2         3         4 
          5         6         7         8 
          9        10        11        12 
         13        14        15         0 
\end{euleroutput}
\begin{eulerprompt}
>(1:5)_2
\end{eulerprompt}
\begin{euleroutput}
          1         2         3         4         5 
          2         2         2         2         2 
\end{euleroutput}
\begin{eulerprompt}
>random(3,3)_random(2,2)
\end{eulerprompt}
\begin{euleroutput}
    0.45954   0.55028   0.96497 
    0.74057   0.95865   0.27366 
     0.4379   0.64177   0.93735 
    0.65758   0.20468         0 
    0.70238   0.47586         0 
\end{euleroutput}
\begin{eulerprompt}
>for k=1 to prod(size(A)); A\{k\}=k; end; short A
\end{eulerprompt}
\begin{euleroutput}
          1         2         3         4 
          5         6         7         8 
          9        10        11        12 
         13        14        15        16 
\end{euleroutput}
\begin{eulerprompt}
>B=zeros(size(A))
\end{eulerprompt}
\begin{euleroutput}
          0         0         0         0 
          0         0         0         0 
          0         0         0         0 
          0         0         0         0 
\end{euleroutput}
\begin{eulerprompt}
>B=ones(size(A))
\end{eulerprompt}
\begin{euleroutput}
          1         1         1         1 
          1         1         1         1 
          1         1         1         1 
          1         1         1         1 
\end{euleroutput}
\begin{eulercomment}
Berikutnya operasi penjumlahan dam pengurangan matriks
\end{eulercomment}
\begin{eulerprompt}
>shortformat; I=intrandom(3,4,10) 
\end{eulerprompt}
\begin{euleroutput}
          3         6         4        10 
          5        10         5         2 
          1         2         9         7 
\end{euleroutput}
\begin{eulerprompt}
>shortformat; J=intrandom(3,4,8)
\end{eulerprompt}
\begin{euleroutput}
          2         6         5         7 
          2         7         6         2 
          4         1         6         6 
\end{euleroutput}
\begin{eulerprompt}
>C= I-J
\end{eulerprompt}
\begin{euleroutput}
          1         0        -1         3 
          3         3        -1         0 
         -3         1         3         1 
\end{euleroutput}
\begin{eulerprompt}
>C= I+J
\end{eulerprompt}
\begin{euleroutput}
          5        12         9        17 
          7        17        11         4 
          5         3        15        13 
\end{euleroutput}
\begin{eulercomment}
Dalam materi matriks yang pernah kita pelajari ada sebutan transpose,
Invers dan juga determinan, jika menggunakan EMt sebagai berikut
secera berurutan:
\end{eulercomment}
\begin{eulerprompt}
>T = transpose(I)
\end{eulerprompt}
\begin{euleroutput}
          3         5         1 
          6        10         2 
          4         5         9 
         10         2         7 
\end{euleroutput}
\begin{eulerprompt}
>T = I'
\end{eulerprompt}
\begin{euleroutput}
          3         5         1 
          6        10         2 
          4         5         9 
         10         2         7 
\end{euleroutput}
\begin{eulerprompt}
>K = J^(-1)
\end{eulerprompt}
\begin{euleroutput}
        0.5   0.16667       0.2   0.14286 
        0.5   0.14286   0.16667       0.5 
       0.25         1   0.16667   0.16667 
\end{euleroutput}
\begin{eulerprompt}
>shortformat; L=intrandom(3,3,7)
\end{eulerprompt}
\begin{euleroutput}
          2         3         3 
          2         4         1 
          6         6         3 
\end{euleroutput}
\begin{eulerprompt}
>det(L)
\end{eulerprompt}
\begin{euleroutput}
  -24
\end{euleroutput}
\begin{eulercomment}
Selanjutnya adalah cara ekstraksi baris dan kolom, atau
sub-matriks,yang  mirip dengan R sebagai berikut:

\end{eulercomment}
\begin{eulerprompt}
>L[,2:3]
\end{eulerprompt}
\begin{euleroutput}
          3         3 
          4         1 
          6         3 
\end{euleroutput}
\begin{eulerprompt}
>shortformat; X=redim(1:20,4,5)
\end{eulerprompt}
\begin{euleroutput}
          1         2         3         4         5 
          6         7         8         9        10 
         11        12        13        14        15 
         16        17        18        19        20 
\end{euleroutput}
\begin{eulerprompt}
>function setmatrixvalue (M, i, j, v) ...
\end{eulerprompt}
\begin{eulerudf}
  loop 1 to max(length(i),length(j),length(v))
     M[i\{#\},j\{#\}] = v\{#\};
  end;
  endfunction
\end{eulerudf}
\begin{eulerprompt}
>setmatrixvalue(X,1:4,4:-1:1,0); X,
\end{eulerprompt}
\begin{euleroutput}
          1         2         3         0         5 
          6         7         0         9        10 
         11         0        13        14        15 
          0        17        18        19        20 
\end{euleroutput}
\begin{eulerprompt}
>(1:4)*(1:4)'
\end{eulerprompt}
\begin{euleroutput}
          1         2         3         4 
          2         4         6         8 
          3         6         9        12 
          4         8        12        16 
\end{euleroutput}
\begin{eulerprompt}
>a=0:10; b=a'; p=flatten(a*b); q=flatten(p-p'); ...
>u=sort(unique(q)); f=getmultiplicities(u,q); ...
>statplot(u,f,"h"):
\end{eulerprompt}
\eulerimg{27}{images/Nur Alya Fadilah_Aplikom-272.png}
\begin{eulerprompt}
>getfrequencies(q,-50:10:50)
\end{eulerprompt}
\begin{euleroutput}
  [613,  814,  1088,  1404,  1904,  2389,  1431,  1109,  841,  680]
\end{euleroutput}
\begin{eulerprompt}
>plot2d(q,distribution=11):
\end{eulerprompt}
\eulerimg{27}{images/Nur Alya Fadilah_Aplikom-273.png}
\begin{eulerprompt}
>\{x,y\}=histo(q,v=-55:10:55); y=y/sum(y)/differences(x);
>plot2d(x,y,>bar,style="/"):
\end{eulerprompt}
\eulerimg{27}{images/Nur Alya Fadilah_Aplikom-274.png}
\begin{eulercomment}
\end{eulercomment}
\eulersubheading{Menghasilkan Data Acak Menggunakan Fungsi Distribusi CAKUPAN MATERI}
\begin{eulercomment}
1. Definisi Bilangan Acak dan Data Acak

\end{eulercomment}
\begin{eulerttcomment}
      Bilangan Acak adalah bilangan yang tidak dapat diprediksi
\end{eulerttcomment}
\begin{eulercomment}
kemunculannya. Sehingga, tidak ada komputasi yang benar-benar
menghasilkan deret bilangan acak secara sempurna.\\
\end{eulercomment}
\begin{eulerttcomment}
      Bilangan acak sendiri dapat dibangkitkan dengan pola tertentu
\end{eulerttcomment}
\begin{eulercomment}
yang dinamakan dengan distribusi, dengan catatan mengikuti fungsi
distribusi yang ditentukan.\\
\end{eulercomment}
\begin{eulerttcomment}
      Data acak merupakan hasil dari suatu percobaan acak. Sedangkan
\end{eulerttcomment}
\begin{eulercomment}
percobaan acak adalah suatu proses yang dilakukan sedemikian rupa
sehingga hasilnya tidak dapat ditentukan dengan pasti sebelum
percobaan tersebut selesai dilakukan

contoh :
\end{eulercomment}
\begin{eulerprompt}
>intrandom(1,10,10)
\end{eulerprompt}
\begin{euleroutput}
  [3,  8,  3,  2,  3,  1,  5,  1,  8,  7]
\end{euleroutput}
\begin{eulercomment}
\end{eulercomment}
\eulersubheading{}
\begin{eulercomment}
2. Pengertian Distribusi Diskrit dan Konsep yang Terkait

\end{eulercomment}
\begin{eulerttcomment}
       Distribusi diskrit dalam statistika adalah distribusi data yang
\end{eulerttcomment}
\begin{eulercomment}
memiliki nilai-nilai yang terpisah dan dapat dihitung. Contohnya\\
adalah jumlah anak dalam sebuah keluarga, jumlah mata dadu yang\\
muncul, atau jumlah pelanggan yang datang ke sebuah toko.\\
\end{eulercomment}
\begin{eulerttcomment}
       Distribusi diskrit merujuk pada distribusi
\end{eulerttcomment}
\begin{eulercomment}
probabilitas yang melibatkan variabel acak diskrit. Variabel acak\\
diskrit adalah variabel acak yang hanya dapat mengambil nilai-nilai\\
terpisah, bukan nilai-nilai kontinu seperti pada variabel acak\\
kontinu. Distribusi diskrit memberikan probabilitas masing-masing\\
nilai yang mungkin dari variabel acak tersebut. Berikut adalah
beberapa konsep kunci yang terkait dengan distribusi diskrit dalam\\
statistika: \\
1. Fungsi Probabilitas Diskrit (Probability Mass Function - PMF):\\
-Fungsi probabilitas diskrit, atau PMF, memberikan probabilitas bahwa\\
variabel acak diskrit akan mengambil nilai tertentu.\\
PMF umumnya dilambangkan dengan P(X=x), di mana X adalah variabel acak\\
dan x adalah nilai yang mungkin dari variabel tersebut.\\
\end{eulercomment}
\begin{eulerttcomment}
 
\end{eulerttcomment}
\begin{eulercomment}
2. Ruang Sampel (Sample Space):\\
-Ruang sampel adalah himpunan semua hasil mungkin dari suatu percobaan\\
acak yang dapat diukur.\\
-Setiap elemen dalam ruang sampel merupakan hasil yang mungkin dari\\
variabel acak.\\
\end{eulercomment}
\begin{eulerttcomment}
 
\end{eulerttcomment}
\begin{eulercomment}
3. Hukum Probabilitas untuk Distribusi Diskrit:\\
Probabilitas suatu kejadian adalah bilangan yang berada dalam rentang\\
0 hingga 1, atau 0 \textless{}= P(A)\textless{}=1 untuk setiap kejadian A.\\
Probabilitas total dari semua hasil dalam ruang sampel adalah 1, atau\\
P (S)= 1, di mana S adalah ruang sampel.\\
\end{eulercomment}
\begin{eulerttcomment}
 
\end{eulerttcomment}
\begin{eulercomment}
4. Fungsi Distribusi Kumulatif (Cumulative Distribution Function -\\
CDF):\\
-Fungsi distribusi kumulatif memberikan probabilitas bahwa variabel\\
acak diskrit kurang dari atau sama dengan nilai tertentu.\\
-Notasi matematisnya sering kali disimbolkan sebagai F(x)-P(X\textless{}=x)\\
\end{eulercomment}
\begin{eulerttcomment}
 
\end{eulerttcomment}
\begin{eulercomment}
5. Harapan (Expectation) dan Varians:\\
-Harapan atau nilai rata-rata (E(X)) dari distribusi diskrit adalah\\
jumlah tertimbang dari nilai-nilai mungkin berdasarkan probabilitas\\
masing-masing nilai.\\
-Varians Var(X)) mengukur sejauh mana nilai-nilai distribusi tersebar\\
dari nilai rata-ratanya.

\end{eulercomment}
\begin{eulerprompt}
> 
\end{eulerprompt}
\eulersubheading{}
\begin{eulercomment}
3. Metode Menentukan Distribusi Diskrit

\end{eulercomment}
\begin{eulerttcomment}
      Untuk menentukan distribusi diskrit sendiri, dapat menggunakan
\end{eulerttcomment}
\begin{eulercomment}
metode berikut. Pertama kita mengatur fungsi distribusi, fungsi
distribusi adalah fungsi yang menggambarkan kemungkinan suatu variabel
acak untuk memiliki nilai tertentu atau dalam rentang waktu tertentu.\\
Langkah mengatur fungsi distribusi:\\
-Menentukan jenis var acak yg akan diteliti, apakah diskrit atau\\
kontinu\\
-Menentukan parameter-parameter yang berkaitan dengan fungsi\\
distribusi, spt probabilitas\\
-Menentukan bentuk fungsi distribusi yg sesuai dg variabel acak dan\\
parameter yg sudah ditentukan
\end{eulercomment}
\begin{eulerprompt}
>wd = 0|((1:6)+[-0.01,0.01,0,0,0,0])/5
\end{eulerprompt}
\begin{euleroutput}
  [0,  0.198,  0.402,  0.6,  0.8,  1,  1.2]
\end{euleroutput}
\begin{eulercomment}
Artinya dengan probabilitas wd[i+1]-wd[i] kita menghasilkan nilai acak
i.

Ini hampir merupakan distribusi yang seragam. Mari kita tentukan
generator angka acak untuk ini. Fungsi find(v,x) menemukan nilai x
dalam vektor v. Fungsi ini juga berlaku untuk vektor x.
\end{eulercomment}
\begin{eulerprompt}
>function wrongdice (n,m) := find(wd,random(n,m))
\end{eulerprompt}
\begin{eulercomment}
Kesalahannya sangat halus sehingga melihatnya hanya dengan iterasi
yang sangat banyak.
\end{eulercomment}
\begin{eulerprompt}
>columnsplot(getmultiplicities(1:6,wrongdice(1,1000000))):
\end{eulerprompt}
\eulerimg{27}{images/Nur Alya Fadilah_Aplikom-275.png}
\begin{eulercomment}
Berikut adalah fungsi sederhana untuk memeriksa distribusi seragam
dari nilai 1...K dalam v. menerima hasilnya, jika untuk semua
frekuensi

\end{eulercomment}
\begin{eulerformula}
\[
\left|f_i-\frac{1}{K}\right| < \frac{\delta}{\sqrt{n}}.
\]
\end{eulerformula}
\begin{eulerprompt}
>function checkrandom (v, delta=1) 
\end{eulerprompt}
\begin{eulerudf}
  K=max(v); n=cols(v);
  fr=getfrequencies(v,1:K);
  return max(fr/n-1/K)<delta/sqrt(n);
  endfunction 
\end{eulerudf}
\begin{eulercomment}
Memang fungsi menolak distribusi seragam.
\end{eulercomment}
\begin{eulerprompt}
>checkrandom(wrongdice(1,1000000)) 
\end{eulerprompt}
\begin{euleroutput}
  0
\end{euleroutput}
\begin{eulercomment}
Dan itu menerima generator acak bawaan.
\end{eulercomment}
\begin{eulerprompt}
>checkrandom(intrandom(1,1000000,6))
\end{eulerprompt}
\begin{euleroutput}
  1
\end{euleroutput}
\begin{eulercomment}
Kita dapat menghitung distribusi binomial. Pertama ada binomialsum(),
yang mengembalikan probabilitas i atau kurang hit dari n percobaan.
\end{eulercomment}
\begin{eulerprompt}
>bindis(410,1000,0.4)
\end{eulerprompt}
\begin{euleroutput}
  0.7514
\end{euleroutput}
\begin{eulercomment}
Perintah berikut adalah cara langsung untuk mendapatkan hasil di atas.\\
Tapi untuk n besar, penjumlahan langsungnya tidak akurat dan lambat.
\end{eulercomment}
\begin{eulerprompt}
>p=0.4; i=0:410; n=1000; sum(bin(n,i)*p^i*(1-p)^(n-i))
\end{eulerprompt}
\begin{euleroutput}
  0.7514
\end{euleroutput}
\eulerheading{membaca data dari CSV}
\begin{eulercomment}
pertama tama kita download file csv yang telah di sediakan di besmart,
setelah itu kita jadi satukan dalam 1 folder dengan file emt kita.
lalu masukan file tersebut dengan definisi file="nama file csv"
\end{eulercomment}
\begin{eulerprompt}
>file="test.csv";  ...
>M=random(3,3); writematrix(M,file)
\end{eulerprompt}
\begin{eulercomment}
M mendefinisikan sebagai matrix\\
random(n,m) mendefinisikan matrix dengan variabel acak yang akan di
keluarakan\\
writematrix digunakan untuk menuliskan matriks yang ada

lalu kita print datanya dengan
\end{eulercomment}
\begin{eulerprompt}
>printfile(file)
\end{eulerprompt}
\begin{euleroutput}
  0.2715906836489253,0.7715320729558478,0.6110298176936501
  0.4607076408155958,0.7475173601596502,0.5399615927118507
  0.8808809313340481,0.1926857542145481,0.3004269774706536
  
\end{euleroutput}
\begin{eulercomment}
titik desimal pada data tersebut dapat di jadikan pada format EMT
dengan cara menggunakan readmatrix()
\end{eulercomment}
\begin{eulerprompt}
>readmatrix(file)
\end{eulerprompt}
\begin{euleroutput}
    0.27159   0.77153   0.61103 
    0.46071   0.74752   0.53996 
    0.88088   0.19269   0.30043 
\end{euleroutput}
\begin{eulerprompt}
> 
\end{eulerprompt}
\begin{eulercomment}
Di Excel atau spreadsheet serupa, Anda dapat mengekspor matriks
sebagai CSV (nilai dipisahkan koma). Di Excel 2007, gunakan "simpan
sebagai" dan "format lain", lalu pilih "CSV". Pastikan, tabel saat ini
hanya berisi data yang ingin Anda ekspor.

Berikut ini contohnya.
\end{eulercomment}
\begin{eulerprompt}
>printfile("excel-data.csv")
\end{eulerprompt}
\begin{euleroutput}
  Could not open the file
  excel-data.csv
  for reading!
  Try "trace errors" to inspect local variables after errors.
  printfile:
      open(filename,"r");
\end{euleroutput}
\begin{eulercomment}
Seperti yang Anda lihat, sistem Jerman saya menggunakan titik koma
sebagai pemisah dan koma desimal. Anda dapat mengubahnya di pengaturan
sistem atau di Excel, tetapi tidak perlu membaca matriks ke EMT.

Cara termudah untuk membaca ini ke dalam Euler adalah readmatrix ().
Semua koma diganti dengan titik dengan parameter\textgreater{} koma. Untuk CSV
bahasa Inggris, cukup abaikan parameter ini.
\end{eulercomment}
\begin{eulerprompt}
>M=readmatrix("excel-data.csv",>comma)
\end{eulerprompt}
\begin{euleroutput}
  Could not open the file
  excel-data.csv
  for reading!
  Try "trace errors" to inspect local variables after errors.
  readmatrix:
      if filename<>"" then open(filename,"r"); endif;
\end{euleroutput}
\begin{eulercomment}
data siap di analisis lebih lanjut
\end{eulercomment}
\begin{eulerprompt}
>reset;
\end{eulerprompt}
\eulerheading{Sub Topik 5: Perhitungan terkait analisis data statistika deskriptif}
\begin{eulercomment}
Rata-rata, simpangan baku, jangkauan, modus, ukuran data,varians dan
median.\\
Analisis data statistika deskriptif

Statistika deskriptif adalah bidang ilmu statistika yang mempelajari
cara-cara untuk pengumpulan, penyusunan, dan penyajian data sehingga
memberikan informasi yang berguna. Perlu diketahui juga bahwa
statistika deskriptif memberikan informasi hanya mengenai data yang
dipunyai dan sama sekali tidak menarik inferensia atau kesimpulan
apapun tentang gugus data induknya yang lebih besar.

Dalam praktiknya,analisis data statistika deskriptif bisa dilakukan
dengan menerapkan sejumlah metode statistik, seperti :

\end{eulercomment}
\eulersubheading{1. Mencari rata rata/mean}
\begin{eulercomment}
Metode pertama yang digunakan untuk melakukan analisis statistika
adalah mean atau sering disebut rata-rata. Saat akan menghitung
rata-rata, kita bisa melakukan dengan cara menambahkan daftar angka
kemudian membagi angka tersebut dengan jumlah item dalam daftar.
Metode ini memungkinkan penentuan tren keseluruhan dari kumpulan data
dan mampu mendapatkan tampilan data yang cepat dan ringkas. Manfaat
dari metode ini juga termasuk perhitungan yang sederhana dan cepat.

\end{eulercomment}
\eulersubheading{a. Rata-rata hitung data tunggal}
\begin{eulercomment}
Misalkan\\
\end{eulercomment}
\begin{eulerformula}
\[
x_1 , x_2 , x_3 ,..., x_n
\]
\end{eulerformula}
\begin{eulercomment}
adalah data yang dikumpulkan dari suatu sampel atau populasi maka
rata-rata hitung untuk sampel disimbolkan dengan\\
\end{eulercomment}
\begin{eulerformula}
\[
\bar{x}
\]
\end{eulerformula}
\begin{eulercomment}
dan rata-rata hitung untuk populasi disimbolkan dengan\\
\end{eulercomment}
\begin{eulerformula}
\[
\mu
\]
\end{eulerformula}
\begin{eulercomment}
Sehingga, untuk mencari rata-rata hitung data tunggal terdapat 2 jenis
rumus sebagai berikut :\\
1. Rata-rata hitung sampel\\
\end{eulercomment}
\begin{eulerformula}
\[
\bar{x}=\frac{\sum_{i=1}^{n} x_i}{n}
\]
\end{eulerformula}
\begin{eulercomment}
2. Rata-rata hitung populasi,\\
\end{eulercomment}
\begin{eulerformula}
\[
\mu=\frac{\sum_{i=1}^{n} X_i}{n}
\]
\end{eulerformula}
\begin{eulercomment}
Untuk menghitung rata-rata data tunggal dengan EMT, kita dapat
menggunakan sintaks

\textgreater{} mean ([data])

Contoh Soal:\\
1. Diketahui data usia(dalam tahun) penduduk suatu daerah adalah
sebagai berikut:\\
60,70,66,75,77,68,45,30,15,71,69,84,13\\
hitunglah rata-rata usia penduduk tersebut.\\
Jawab :
\end{eulercomment}
\begin{eulerprompt}
>mean([60,70,66,75,77,68,45,30,15,71,69,84,13])
\end{eulerprompt}
\begin{euleroutput}
  57.1538461538
\end{euleroutput}
\begin{eulercomment}
Jadi, rata rata data tersebut adalah 57.1538461538

2. Nilai ulangan matematika dari 10 siswa adalah 80, 88, 70, 60, 90,
75, 92, 78, 67, 90. Tentukan rata-rata dari data tersebut!\\
Jawab :
\end{eulercomment}
\begin{eulerprompt}
>mean([80, 88, 70, 60, 90, 75, 92, 78, 67, 90])
\end{eulerprompt}
\begin{euleroutput}
  79
\end{euleroutput}
\begin{eulercomment}
Jadi, rata-rata dari data tersebut yaitu 79

\end{eulercomment}
\eulersubheading{b. Rata-rata data tabel distribusi}
\begin{eulercomment}
Jika diberikan data\\
\end{eulercomment}
\begin{eulerformula}
\[
x_1,x_2,...,x_n
\]
\end{eulerformula}
\begin{eulercomment}
yang memiliki frekuensi berturut- turut\\
\end{eulercomment}
\begin{eulerformula}
\[
f_1,f_2,...,f_n
\]
\end{eulerformula}
\begin{eulercomment}
maka, rataan hitung dari data yang disajikan dalam daftar distribusi
tersebut ditentukan dengan 2 jenis rumus sebagai berikut :

1. Rata-rata hitung sampel\\
Untuk rata-rata hitung sampel,\\
\end{eulercomment}
\begin{eulerformula}
\[
\bar{x}=\frac{\sum_{i=1}^{n} f_i x_i}{\sum_{i=1}^{n} f_i}
\]
\end{eulerformula}
\begin{eulercomment}
2.Rata-rata hitung populasi\\
Untuk rata-rata hitung populasi,\\
\end{eulercomment}
\begin{eulerformula}
\[
\mu=\frac{\sum_{i=1}^{n} f_i x_i}{\sum_{i=1}^{n} f_i}
\]
\end{eulerformula}
\begin{eulercomment}
Cara diatas adalah beberapa perhitungan untuk mencari rata-rata data
tabel distribusi menggunakan metode yang ada dalam statistika. Dengan
menggunakan EMT kita juga bisa menghitung rata-rata data tabel
distribusi dengan mudah, yaitu dengan cara berikut:\\
1. Mendeskripsikan data dan frekuensi\\
2. Menghitung rata-rata menggunakan perintah berikut :

\textgreater{} mean(data,frekuensi)

Contoh soal:\\
Diberikan data berat badan siswa kelas V SD yang memiliki jumlah siswa
sebanyak 35 orang anak. anak dengan berat 30kg terdapat 5 orang, anak
dengan berat 35kg terdapat 11 orang, anak dengan berat 40kg terdapat 4
orang, anak dengan berat 38kg terdapat 7 orang, anak dengan berat 44kg
terdapat 7 orang, dan anak dengan berat 50kg terdapat 1 orang.
Tentukan rata-rata berat siswa kelas V SD tersebut!\\
Jawab :
\end{eulercomment}
\begin{eulerprompt}
>printfile("tabel berat badan kelas V SD.dat",7); //meringkas informasi pada soal dengan membuat tabel
\end{eulerprompt}
\begin{euleroutput}
  Could not open the file
  tabel berat badan kelas V SD.dat
  for reading!
  Try "trace errors" to inspect local variables after errors.
  printfile:
      open(filename,"r");
\end{euleroutput}
\begin{eulerprompt}
>data=[30,35,38,40,44,50]//mendefinisikan data sebagai berat siswa dalam satuan kilogram
\end{eulerprompt}
\begin{euleroutput}
  [30,  35,  38,  40,  44,  50]
\end{euleroutput}
\begin{eulerprompt}
>frekuensi=[5,11,7,4,7,1]//mendefinisikan frekuensi sebagai banyak siswa
\end{eulerprompt}
\begin{euleroutput}
  [5,  11,  7,  4,  7,  1]
\end{euleroutput}
\begin{eulerprompt}
>mean(data,frekuensi) //menghitung rata-rata
\end{eulerprompt}
\begin{euleroutput}
  37.6857142857
\end{euleroutput}
\begin{eulercomment}
Jadi, rata-rata berat badan siswa SD kelas V adalah 37.6857142857
\end{eulercomment}
\eulersubheading{c. Rata-rata hitung data kelompok}
\begin{eulercomment}
Misalkan suatu data kelompok terdiri dari n kelas dengan nilai tengah
masing-masing kelas secara berturut-turut adalah\\
\end{eulercomment}
\begin{eulerformula}
\[
t_1, t_2,...,t_n
\]
\end{eulerformula}
\begin{eulercomment}
dan masing-masing frekuensinya adalah\\
\end{eulercomment}
\begin{eulerformula}
\[
f_1, f_2,..., f_n
\]
\end{eulerformula}
\begin{eulercomment}
Untuk mencari rata rata hitung data tersebut terdapat 2 jenis rumus
sebagai berikut :\\
1. Rata-rata hitung sampel\\
untuk rata-rata hitung sampel,\\
\end{eulercomment}
\begin{eulerformula}
\[
\bar{x}=\frac{\sum_{i=1}^{n} t_i f_i}{\sum_{i=1}^{n} f_i}
\]
\end{eulerformula}
\begin{eulercomment}
2. Rata-rata hitung populasi\\
untuk rata-rata hitung populasi,\\
\end{eulercomment}
\begin{eulerformula}
\[
\mu=\frac{\sum_{i=1}^{n} t_i f_i}{\sum_{i=1}^{n} f_i}
\]
\end{eulerformula}
\begin{eulercomment}
Untuk menghitung rata-rata data kelompok di EMT dapat dilakukan dengan
langkah berikut :\\
1. Menentukan tepi bawah kelas(Tb), panjang kelas(P), dan tepi atas
kelas(Ta) dengan rumus :\\
\end{eulercomment}
\begin{eulerformula}
\[
Tb=a-0,5
\]
\end{eulerformula}
\begin{eulerformula}
\[
P=(b-a)+1
\]
\end{eulerformula}
\begin{eulerformula}
\[
Ta=b+0,5
\]
\end{eulerformula}
\begin{eulercomment}
Keterangan :\\
a = batas bawah kelas\\
b = batas atas kelas


2. Membuat data menjadi bentuk tabel, dengan perintah

\textgreater{} r= tepi bawah terkecil : panjang kelas : tepi atas terbesar;\\
f=[frekuensi];\\
\textgreater{}T:r[1:jumlah kelas]' \textbar{} r[2:jumlah kelas + 1]' \textbar{}f';\\
writetable(T, labc=["tepi bawah", "tapi atas", "frekuensi"])

3. Menghitung nilai tengah kelas, dengan perintah

\textgreater{}T[,1]+T[,2]/2

4. Mengubah baris menjadi kolom

\textgreater{}t=fold(r,[0.5,0.5])

5. Menghitung rata-rata, dengan perintah

\textgreater{}mean(t,f)

Contoh soal :\\
1. Disajikan data kelompok seperti berikut :
\end{eulercomment}
\begin{eulerprompt}
>printfile("Tabel rata-rata data kelompok.dat",7)
\end{eulerprompt}
\begin{euleroutput}
  Could not open the file
  Tabel rata-rata data kelompok.dat
  for reading!
  Try "trace errors" to inspect local variables after errors.
  printfile:
      open(filename,"r");
\end{euleroutput}
\begin{eulerprompt}
>31-0.5  //Tepi bawah terkecil
\end{eulerprompt}
\begin{euleroutput}
  30.5
\end{euleroutput}
\begin{eulerprompt}
>(40-31)+1  //Panjang kelas
\end{eulerprompt}
\begin{euleroutput}
  10
\end{euleroutput}
\begin{eulerprompt}
>90+0.5 //Tepi atas kelas
\end{eulerprompt}
\begin{euleroutput}
  90.5
\end{euleroutput}
\begin{eulerprompt}
>r=30.5:10:90.5; f=[3, 5, 10, 11, 8, 3];
>T:=r[1:6]' | r[2:7]' | f' ; writetable(T,labc=["tepi bawah", "tepi atas", "frekuensi"])
\end{eulerprompt}
\begin{euleroutput}
   tepi bawah tepi atas frekuensi
         30.5      40.5         3
         40.5      50.5         5
         50.5      60.5        10
         60.5      70.5        11
         70.5      80.5         8
         80.5      90.5         3
\end{euleroutput}
\begin{eulerprompt}
>t=(T[,1]+T[,2])/2  //menghitung nilai tengah kelas
\end{eulerprompt}
\begin{euleroutput}
           35.5 
           45.5 
           55.5 
           65.5 
           75.5 
           85.5 
\end{euleroutput}
\begin{eulerprompt}
>t=fold(r,[0.5,0.5]) // mengubah tampilan data kolom menjadi baris dan sebaliknya
\end{eulerprompt}
\begin{euleroutput}
  [35.5,  45.5,  55.5,  65.5,  75.5,  85.5]
\end{euleroutput}
\begin{eulerprompt}
>mean(t,f)
\end{eulerprompt}
\begin{euleroutput}
  61.75
\end{euleroutput}
\begin{eulercomment}
Jadi, rata-rata data kelompok tersebut adalah 61,75

2. Diberikan data kelompok berikut yang mewakili jumlah jam belajar
per minggu dari sekelompok siswa :
\end{eulercomment}
\begin{eulerprompt}
>printfile("Tabel data kelompok conso 2.dat",5)
\end{eulerprompt}
\begin{euleroutput}
  Could not open the file
  Tabel data kelompok conso 2.dat
  for reading!
  Try "trace errors" to inspect local variables after errors.
  printfile:
      open(filename,"r");
\end{euleroutput}
\begin{eulercomment}
Hitunglah rata-rata jumlah jam belajar per minggu dari data kelompok
tersebut!\\
Jawab :
\end{eulercomment}
\begin{eulerprompt}
>10-0.5  //tepi bawah terkecil
\end{eulerprompt}
\begin{euleroutput}
  9.5
\end{euleroutput}
\begin{eulerprompt}
>(14-10)+1  //panjang kelas
\end{eulerprompt}
\begin{euleroutput}
  5
\end{euleroutput}
\begin{eulerprompt}
>29+0.5  //tepi atas terbesar
\end{eulerprompt}
\begin{euleroutput}
  29.5
\end{euleroutput}
\begin{eulerprompt}
>r=9.5:5:29.5; f=[5, 8, 12, 6];
>T:=r[1:4]' | r[2:5]' | f'; writetable(T,labc=["tepi bawah", "tepi atas", "frekuensi"])
\end{eulerprompt}
\begin{euleroutput}
   tepi bawah tepi atas frekuensi
          9.5      14.5         5
         14.5      19.5         8
         19.5      24.5        12
         24.5      29.5         6
\end{euleroutput}
\begin{eulerprompt}
>t=(T[,1]+T[,2])/2  // menghitung nilai tengah kelas
\end{eulerprompt}
\begin{euleroutput}
             12 
             17 
             22 
             27 
\end{euleroutput}
\begin{eulerprompt}
>t=fold(r,[0.5,0.5]) // mengubah tampilan data kolom menjadi baris dan sebaliknya
\end{eulerprompt}
\begin{euleroutput}
  [12,  17,  22,  27]
\end{euleroutput}
\begin{eulerprompt}
>mean(t,f)
\end{eulerprompt}
\begin{euleroutput}
  20.064516129
\end{euleroutput}
\begin{eulercomment}
Jadi, rata-rata data kelompok tersebut yaitu 20.064516129\\
\end{eulercomment}
\eulersubheading{2. Mencari median}
\begin{eulercomment}
Median (Me) adalah nilai tengah dari suatu data yang telah disusun
dari data terkecil sampai data terbesar atau sebaliknya. Selain
sebagai ukuran pemusatan data, median juga dijadikan sebagai ukuran
letak data dan dikenal sebagai kuartil 2 (Q2). Rumus perhitungan
median dibedakan untuk data tak berkelompok dan data berkelompok.

\end{eulercomment}
\eulersubheading{a. Median data tunggal}
\begin{eulercomment}
Median data tunggal adalah mengurutkan data berdasarkan nilainya,
misalkan data yang telah terurut dari data terkecil ke data terbesar
adalah\\
\end{eulercomment}
\begin{eulerformula}
\[
x_1, x_2,..., x_n
\]
\end{eulerformula}
\begin{eulerttcomment}
 untuk menentukan letak median dengan menggunakan rumus :
\end{eulerttcomment}
\begin{eulercomment}
1. Jika jumlah suatu data(n) berjumlah ganjil maka nilai mediannya
adalah sama dengan data yang memiliki nilai di urutan paling tengah
yang memiliki nomor urut k, dimana untuk menentukan nilai k dapat
dihitung menggunakan rumus:

\end{eulercomment}
\begin{eulerformula}
\[
k=\frac{n+1}2
\]
\end{eulerformula}
\begin{eulercomment}
2. Jika jumlah suatu data (n) berjumlah genap, maka untuk menghitung
mediannya dengan menggunakan rumus :\\
\end{eulercomment}
\begin{eulerformula}
\[
k=\frac{n}2
\]
\end{eulerformula}
\begin{eulerformula}
\[
Median = \frac{1}2(x_k+x_{k+1})
\]
\end{eulerformula}
\begin{eulercomment}
Diatas adalah rumus untuk mencari median secara statistika. Dengan
menggunakan EMT kita bisa menentukan median dengan menggunakan
perintah

\textgreater{} median([data])

perintah tersebut dapat berjalan dengan baik apabila data sudah
diurutkan terlebih dahulu dari data terkecil hingga terbesar.\\
Contoh soal :\\
Diketahui data hasil tes SKD calon PNS adalah sebagai berikut :\\
487, 300, 450, 500, 521, 440\\
Tentukan nilai median dari data tersebut!\\
Jawab :
\end{eulercomment}
\begin{eulerprompt}
>data=[487, 300, 450, 500, 521, 440]; //mendeskripsikan data
>urutan=sort(data)  //mengurutkan data
\end{eulerprompt}
\begin{euleroutput}
  [300,  440,  450,  487,  500,  521]
\end{euleroutput}
\begin{eulerprompt}
>median([urutan])
\end{eulerprompt}
\begin{euleroutput}
  468.5
\end{euleroutput}
\begin{eulercomment}
Jadi, nilai median dari data hasil tes SKD adalah 468.5\\
\end{eulercomment}
\eulersubheading{b. Median data kelompok}
\begin{eulercomment}
Menghitung median data kelompok dapat menggunakan rumus di bawah ini :

\end{eulercomment}
\begin{eulerformula}
\[
M_e = Tb + p  \frac{\frac{1}2 n - F}f
\]
\end{eulerformula}
\begin{eulercomment}
Keterangan:\\
Tb = tepi bawah kelas median, ialah kelas dimana median terletak\\
p = panjang kelas median\\
n = ukuran sampel / banyak data\\
F = jumlah semua frekuensi dengan tanda kelas lebih kecil dari tanda
kelas median.\\
f = frekuensi kelas median

Untuk menghitung median data berkelompok di EMT, dapat dilakukan
dengan cara berikut:\\
1. Menentukan tepi bawah kelas (Tb), panjang kelas (P), dan tepi atas
kelas (Ta) dengan rumus :

\end{eulercomment}
\begin{eulerformula}
\[
T_b=a-0,5
\]
\end{eulerformula}
\begin{eulerformula}
\[
P=(b-a)+1
\]
\end{eulerformula}
\begin{eulerformula}
\[
T_a=b+0.5
\]
\end{eulerformula}
\begin{eulercomment}
2. Mendeskripsikan data dalam bentuk tabel, dengan perintah

\textgreater{} r=tepi bawah terkecil:panjang kelas:tepi atas terbesar;
f=[frekuensi];\\
\textgreater{} T:=r[1:jumlah kelas]' \textbar{} r[2:jumlah kelas + 1]' \textbar{} f';
writetable(T,labc=["tepi bawah","tepi atas","frekuensi"]))

3. Mendeskripsikan batas bawah kelas median, panjang kelas median,
banyak data, jumlah frekuensi sebelum kelas median, frekuensi median

\textgreater{} Tb=(tepi bawah kelas median), p=(panjang kelas median), n=(banyak
data), F=(jumlah frekuensi sebelum kelas median), f=(frekuensi kelas
median)

4. Menghitung median data dengan perintah:

\textgreater{} Tb+p*(1/2*n-F)/f

Contoh soal :\\
Berikut adalah data hasil dari pengukuran berat badan 20 siswa SD
kelas V. Dari ke 20 siswa, siswa yang mempunyai berat badan dalam
rentang 21-26 kg sebanyak 5 orang, yang mempunyai berat badan dalam
rentang 27-32 kg sebanyak 4 orang, yang mempunyai berat badan dalam
rentang 33-38 kg sebanyak 3 orang, yang mempunyai berat badan dalam
rentang 39-44 kg sebanyak 2 orang, yang mempunyai berat badan dalam
rentang 45-50 kg sebanyak 3 orang, dan yang mempunyai berat badan
51-56 kg sebanyak 3 orang. Tentukan median dari\\
data hasil pengukuran berat badan 20 siswa di SD tersebut!\\
Penyelesaian:\\
Menentukan tepi bawah kelas yang terkecil
tukan tepi bawah kelas yang terkecil
\end{eulercomment}
\begin{eulerprompt}
>21-0.5 // menentukan tepi bawah kelas terkecil
\end{eulerprompt}
\begin{euleroutput}
  20.5
\end{euleroutput}
\begin{eulerprompt}
> (26-21)+1 // menentukan panjang kelas
\end{eulerprompt}
\begin{euleroutput}
  6
\end{euleroutput}
\begin{eulerprompt}
> 56+0.5 // tepi atas kelas terbesar
\end{eulerprompt}
\begin{euleroutput}
  56.5
\end{euleroutput}
\begin{eulerprompt}
>r=20.5:6:56.5; f=[5, 4, 3, 2, 3, 3];
>T :=r[1:6]' | r[2:7]' | f' ; writetable(T, labc=["Tb", "Ta", "frekuensi"])
\end{eulerprompt}
\begin{euleroutput}
          Tb        Ta frekuensi
        20.5      26.5         5
        26.5      32.5         4
        32.5      38.5         3
        38.5      44.5         2
        44.5      50.5         3
        50.5      56.5         3
\end{euleroutput}
\begin{eulerprompt}
>Tb=32.5, p=6, n=20, F=9, f=3
\end{eulerprompt}
\begin{euleroutput}
  32.5
  6
  20
  9
  3
\end{euleroutput}
\begin{eulerprompt}
>Tb+p*(1/2*n-F)/f
\end{eulerprompt}
\begin{euleroutput}
  34.5
\end{euleroutput}
\begin{eulercomment}
Jadi, median dari data hasil pengukuran berat badan 20 siswa SD kelas
V adalah 34.5

\end{eulercomment}
\eulersubheading{3. Mencari Modus}
\begin{eulercomment}
Modus adalah area fokus dalam analisis statistika deskriptif yang
termasuk dalam ukuran pusat data. Ini adalah nilai yang paling sering
muncul dalam kumpulan data atau nilai yang memiliki frekuensi
tertinggi dalam distribusi data. Modus dapat dibagi menjadi dua jenis,
yaitu modus untuk data tunggal dan modus untuk data kelompok.

\end{eulercomment}
\eulersubheading{a.Modus untuk data tunggal:}
\begin{eulercomment}
Menentukan modus untuk data tunggal cukup sederhana. Pertama, data
diurutkan dari nilai terkecil ke terbesar sehingga data dengan nilai
yang sama berdekatan satu sama lain. Selanjutnya, frekuensi
masing-masing data dihitung, dan data yang memiliki frekuensi
tertinggi dipilih sebagai modus.

\end{eulercomment}
\eulersubheading{b.Modus untuk data kelompok}
\begin{eulercomment}
Berikut rumus untuk mencari modus data kelompok :\\
\end{eulercomment}
\begin{eulerformula}
\[
M_o = Tb+ \frac{d_1}{d_1+d_2} c
\]
\end{eulerformula}
\begin{eulercomment}
Keterangan :\\
Tb = Tepi bawah\\
d1 = selisih f modus dengan f sebelumnya\\
d2 = selisih f modus dengan f sesudahnya\\
\end{eulercomment}
\begin{eulerttcomment}
 c = Panjang kelas
\end{eulerttcomment}
\begin{eulercomment}

Untuk menghitung modus data berkelompok di EMT, dapat dilakukan dengan
cara berikut:\\
1. Menentukan tepi bawah kelas (Tb), panjang kelas (P), dan tepi atas
kelas (Ta) dengan rumus :\\
\end{eulercomment}
\begin{eulerformula}
\[
T_b=a-0,5
\]
\end{eulerformula}
\begin{eulerformula}
\[
P=(b-a)+1
\]
\end{eulerformula}
\begin{eulerformula}
\[
T_a=b+0.5
\]
\end{eulerformula}
\begin{eulercomment}
dimana a = batas bawah kelas dan b = batas atas kelas

2. Mendeskripsikan data dalam bentuk tabel, dengan perintah\\
\textgreater{} r=tepi bawah terkecil:panjang kelas:tepi atas terbesar;
v=[frekuensi];\\
\textgreater{} T:=r[1:jumlah kelas]’ \textbar{} r[2:jumlah kelas + 1]’ \textbar{} f’;
writetable(T,labc=[”tepi bawah”,”tepi atas”,”frekuensi”]))

3. Mendeskripsikan tepi bawah kelas modus, panjang kelas modus,selisih
frekuensi modus dengan\\
frekuensi sebelumnya, selisih frekuensi modus dengan frekuensi
sesudahnya\\
\textgreater{} Tb=(tepi bawah kelas modus), p=(panjang kelas modus), d1=(selisih
frekuensi modus dengan frekuensi sebelumnya), d2=(selisih frekuensi
dengan frekuensi sesudahnya)

4. Menghitung modus dengan perintah:\\
\textgreater{} Tb+p*d1/(d1+d2)

Contoh soal\\
Diketahui sebuah data kelompok sebagai berikut :
\end{eulercomment}
\begin{eulerprompt}
>printfile("Tabel modus data kelompok.dat",8) 
\end{eulerprompt}
\begin{euleroutput}
  Could not open the file
  Tabel modus data kelompok.dat
  for reading!
  Try "trace errors" to inspect local variables after errors.
  printfile:
      open(filename,"r");
\end{euleroutput}
\begin{eulercomment}
Berapakah modus dari data tersebut?
\end{eulercomment}
\begin{eulerprompt}
>20-0.5  //menentukan tepi bawah kelas
\end{eulerprompt}
\begin{euleroutput}
  19.5
\end{euleroutput}
\begin{eulerprompt}
>(29-20)+1 //menentukan panjang kelas
\end{eulerprompt}
\begin{euleroutput}
  10
\end{euleroutput}
\begin{eulerprompt}
>89+0.5 //menentukan tepi atas
\end{eulerprompt}
\begin{euleroutput}
  89.5
\end{euleroutput}
\begin{eulerprompt}
>r=19.5:10:89.5; f=[3, 7, 8, 12, 9, 6, 5];
>T:=r[1:7]' | r[2:8]' | f'; writetable(T,labc=["Tb", "Ta", "frekuensi"])
\end{eulerprompt}
\begin{euleroutput}
          Tb        Ta frekuensi
        19.5      29.5         3
        29.5      39.5         7
        39.5      49.5         8
        49.5      59.5        12
        59.5      69.5         9
        69.5      79.5         6
        79.5      89.5         5
\end{euleroutput}
\begin{eulercomment}
Berdasarkan tabel di atas, modus berada pada kelas 49.5-59.5
\end{eulercomment}
\begin{eulerprompt}
>Tb=49.5, p=10, d1=12-8, d2=12-9
\end{eulerprompt}
\begin{euleroutput}
  49.5
  10
  4
  3
\end{euleroutput}
\begin{eulerprompt}
>Tb+p*d1/(d1+d2)
\end{eulerprompt}
\begin{euleroutput}
  55.2142857143
\end{euleroutput}
\begin{eulercomment}
Jadi, modus dari data kelompok di atas adalah 55.2142857143
\end{eulercomment}
\eulersubheading{4. Mencari varians/ragam}
\begin{eulercomment}
Varians digunakan untuk mengetahui bagaimana sebaran data terhadap
mean atau nilai rata-rata. Sederhananya, varians adalah ukuran
statistik jauh dekatnya penyebaran data dari nilai rata-ratanya. Dalam
mencari ragam dapat dikelompokkan menjadi 2 jenis yaitu sebagai
berikut :

\end{eulercomment}
\eulersubheading{a.Varians data tunggal}
\begin{eulercomment}
Rumus untuk varians data tunggal berikut :\\
1) Untuk populasi

\end{eulercomment}
\begin{eulerformula}
\[
{\sigma}^2=\frac{\sum_{i=1}^{n} (x-\mu)^2}n
\]
\end{eulerformula}
\begin{eulercomment}
2) Untuk sampel

\end{eulercomment}
\begin{eulerformula}
\[
S^2=\frac{\sum_{i=1}^{n} (x-\bar{x})^2}{n-1}
\]
\end{eulerformula}
\begin{eulercomment}
Pada EMT, untuk menemukan suatu Ragam data tunggal dapat menggunakan
perintah berikut:

\textgreater{} mean(dev\textasciicircum{}2)

Contoh soal\\
Hitunglah nilai varians dari data sampel nilai siswa: 9, 10, 6, 7!
\end{eulercomment}
\begin{eulerprompt}
>data=[9, 10, 6, 7]; //mendefinisikan data
>urut=sort(data) //mengurutkan data
\end{eulerprompt}
\begin{euleroutput}
  [6,  7,  9,  10]
\end{euleroutput}
\begin{eulerprompt}
>xbar=mean(urut) //menghitung rata rata dari data
\end{eulerprompt}
\begin{euleroutput}
  8
\end{euleroutput}
\begin{eulerprompt}
>dev= urut-xbar 
\end{eulerprompt}
\begin{euleroutput}
  [-2,  -1,  1,  2]
\end{euleroutput}
\begin{eulerprompt}
>varians=mean(dev^2) //menghitung varians
\end{eulerprompt}
\begin{euleroutput}
  2.5
\end{euleroutput}
\begin{eulercomment}
Jadi, varians dari data sampel tersebut adalah 2.5

\end{eulercomment}
\eulersubheading{b. Varians data kelompok}
\begin{eulercomment}
Rumus untuk varians data kelompok sebagai berikut :\\
1) Untuk populasi

\end{eulercomment}
\begin{eulerformula}
\[
{\sigma}^2=\frac{\sum_{i=1}^{n} f_i (x_i-\mu)^2}{\sum_{i=1}^{n}f_i}
\]
\end{eulerformula}
\begin{eulercomment}
2) Untuk sampel

\end{eulercomment}
\begin{eulerformula}
\[
S^2=\frac{\sum_{i=1}^{n} f_i (x_i-\bar{x})^2}{\sum_{i=1}^{n}f_i-1}
\]
\end{eulerformula}
\begin{eulercomment}
Pada EMT, untuk menemukan Ragam data berkelompk dapat menggunakan
perintah berikut:

1. Menentukan tepi bawah kelas (Tb), panjang kelas (P), dan tepi atas
kelas (Ta) dengan rumus :

\end{eulercomment}
\begin{eulerformula}
\[
T_b=a-0,5
\]
\end{eulerformula}
\begin{eulerformula}
\[
P=(b-a)+1
\]
\end{eulerformula}
\begin{eulerformula}
\[
T_a=b+0.5
\]
\end{eulerformula}
\begin{eulercomment}
dengan a = batas bawah kelas dan b = batas atas kelas

2. Mendeskripsikan data dalam bentuk tabel, dengan perintah

\textgreater{} r=tepi bawah terkecil:panjang kelas:tepi atas terbesar;
f=[frekuensi];\\
\textgreater{} T:=r[1:jumlah kelas]' \textbar{} r[2:jumlah kelas + 1]' \textbar{} f';
writetable(T,labc=["tepi bawah","tepi atas","frekuensi"])

3. Menghitung Ragam dengan perintah

\textgreater{} (T[,1]+T[,2])/2; t=fold(r,[0.5,0.5]);m=mean(t,f);\\
\textgreater{} sum(f*(t-m)\textasciicircum{}2)/sum(f)  //untuk populasi\\
\textgreater{} sum(f*(t-m)\textasciicircum{}2)/(sum(f)-1)  //untuk sampel

Contoh soal\\
Tentukan varians data sampel dari tabel berikut :
\end{eulercomment}
\begin{eulerprompt}
>printfile("Tabel data kelompok varians.dat",7) 
\end{eulerprompt}
\begin{euleroutput}
  Could not open the file
  Tabel data kelompok varians.dat
  for reading!
  Try "trace errors" to inspect local variables after errors.
  printfile:
      open(filename,"r");
\end{euleroutput}
\begin{eulerprompt}
>63-0.5  //tapi bawah terkecil
\end{eulerprompt}
\begin{euleroutput}
  62.5
\end{euleroutput}
\begin{eulerprompt}
>(67-63)+1  //panjang kelas
\end{eulerprompt}
\begin{euleroutput}
  5
\end{euleroutput}
\begin{eulerprompt}
>92+0.5  //tepi atas terbesar
\end{eulerprompt}
\begin{euleroutput}
  92.5
\end{euleroutput}
\begin{eulerprompt}
>r=62.5:5:92.5; f=[3,2,7,3,4,1];
>T:=r[1:6]' | r[2:7]' | f'; writetable(T, labc=["tepi bawah", "tepi atas", "frekuensi"])
\end{eulerprompt}
\begin{euleroutput}
   tepi bawah tepi atas frekuensi
         62.5      67.5         3
         67.5      72.5         2
         72.5      77.5         7
         77.5      82.5         3
         82.5      87.5         4
         87.5      92.5         1
\end{euleroutput}
\begin{eulerprompt}
>(T[,1]+T[,2])/2; t=fold(r,[0.5,0.5]) 
\end{eulerprompt}
\begin{euleroutput}
  [65,  70,  75,  80,  85,  90]
\end{euleroutput}
\begin{eulerprompt}
>m=mean(t,f)
\end{eulerprompt}
\begin{euleroutput}
  76.5
\end{euleroutput}
\begin{eulerprompt}
>sum(f*(t-m)^2)/(sum(f)-1)
\end{eulerprompt}
\begin{euleroutput}
  52.8947368421
\end{euleroutput}
\begin{eulercomment}
Jadi, varians dari data kelompok dari tabel di atas adalah
52.8947368421
\end{eulercomment}
\begin{eulerprompt}
> 
\end{eulerprompt}
\eulersubheading{5. Mencari Simpangan Baku}
\begin{eulercomment}
Standar Deviasi atau simpangan baku adalah akar dari ragam/varians.
Untuk nenetukan nilai standar deviasi, caranya:

\end{eulercomment}
\begin{eulerformula}
\[
\sigma=\sqrt{\sigma^2}
\]
\end{eulerformula}
\begin{eulerformula}
\[
atau
\]
\end{eulerformula}
\begin{eulerformula}
\[
S=\sqrt{S^2}
\]
\end{eulerformula}
\begin{eulercomment}
\end{eulercomment}
\eulersubheading{a. Simpangan baku data tunggal}
\begin{eulercomment}
Untuk data tunggal, simpangan baku populasi atau sampel dapat
dirumuskan sebagai berikut:

1) Untuk populasi

\end{eulercomment}
\begin{eulerformula}
\[
{\sigma}=\sqrt{\frac{\sum_{i=1}^{n} (x-\mu)^2}n}
\]
\end{eulerformula}
\begin{eulercomment}
2) Untuk sampel

\end{eulercomment}
\begin{eulerformula}
\[
S=\sqrt{\frac{\sum_{i=1}^{n} (x-\bar{x})^2}{n-1}}
\]
\end{eulerformula}
\begin{eulercomment}
Pada EMT, untuk menemukan suatu Ragam data tunggal dapat menggunakan
perintah berikut:

\textgreater{} sqrt(mean(dev\textasciicircum{}2))

Contoh soal :\\
1. Simpangan baku untuk data 70,80,40,25,65,87,97,59,24,77,45 adalah\\
Jawab :
\end{eulercomment}
\begin{eulerprompt}
>data=[70,80,40,25,65,87,97,59,24,77,45];
>urut=sort(data)
\end{eulerprompt}
\begin{euleroutput}
  [24,  25,  40,  45,  59,  65,  70,  77,  80,  87,  97]
\end{euleroutput}
\begin{eulerprompt}
>x=mean(urut)
\end{eulerprompt}
\begin{euleroutput}
  60.8181818182
\end{euleroutput}
\begin{eulerprompt}
>dev=urut-x
\end{eulerprompt}
\begin{euleroutput}
  [-36.8182,  -35.8182,  -20.8182,  -15.8182,  -1.81818,  4.18182,
  9.18182,  16.1818,  19.1818,  26.1818,  36.1818]
\end{euleroutput}
\begin{eulerprompt}
>varians=mean(dev^2)
\end{eulerprompt}
\begin{euleroutput}
  550.148760331
\end{euleroutput}
\begin{eulerprompt}
>simpanganbaku= sqrt(varians)
\end{eulerprompt}
\begin{euleroutput}
  23.4552501656
\end{euleroutput}
\begin{eulercomment}
Jadi, simpang baku data tersebut adalah 23.4552501656


\end{eulercomment}
\eulersubheading{b. Simpangan baku data kelompok}
\begin{eulerttcomment}
 Untuk data berkelompok dapat dirumuskan seperti berikut:
\end{eulerttcomment}
\begin{eulercomment}
1) Untuk populasi

\end{eulercomment}
\begin{eulerformula}
\[
{\sigma}=\sqrt{\frac{\sum_{i=1}^{n} f_i (x_i-\mu)^2}{\sum_{i=1}^{n}f_i}}
\]
\end{eulerformula}
\begin{eulercomment}
2) Untuk sampel

\end{eulercomment}
\begin{eulerformula}
\[
S=\sqrt{\frac{\sum_{i=1}^{n} f_i (x_i-\bar{x})^2}{\sum_{i=1}^{n}f_i-1}}
\]
\end{eulerformula}
\begin{eulercomment}
Pada EMT, untuk menemukan Ragam data berkelompk dapat menggunakan
perintah berikut:

1. Menentukan tepi bawah kelas (Tb), panjang kelas (P), dan tepi atas
kelas (Ta) dengan rumus :

\end{eulercomment}
\begin{eulerformula}
\[
T_b=a-0,5
\]
\end{eulerformula}
\begin{eulerformula}
\[
P=(b-a)+1
\]
\end{eulerformula}
\begin{eulerformula}
\[
T_a=b+0.5
\]
\end{eulerformula}
\begin{eulercomment}
dengan a = batas bawah kelas dan b = batas atas kelas

2. Mendeskripsikan data dalam bentuk tabel, dengan perintah

\textgreater{} r=tepi bawah terkecil:panjang kelas:tepi atas terbesar;
f=[frekuensi];\\
\textgreater{} T:=r[1:jumlah kelas]' \textbar{} r[2:jumlah kelas + 1]' \textbar{} f';
writetable(T,labc=["tepi bawah","tepi atas","frekuensi"])

3. Menghitung Ragam dengan perintah

\textgreater{} (T[,1]+T[,2])/2; t=fold(r,[0.5,0.5]);m=mean(t,f);\\
\textgreater{} sqrt(sum(f*(t-m)\textasciicircum{}2)/sum(f))     // untuk populasi\\
\textgreater{} sqrt(sum(f*(t-m)\textasciicircum{}2)/(sum(f)-1))     // untuk sampel

Contoh soal :\\
Simpangan baku dari tabel dibawah ini adalah
\end{eulercomment}
\begin{eulerprompt}
>printfile("Tabel simpangan baku data kelompok.dat",7)
\end{eulerprompt}
\begin{euleroutput}
  Could not open the file
  Tabel simpangan baku data kelompok.dat
  for reading!
  Try "trace errors" to inspect local variables after errors.
  printfile:
      open(filename,"r");
\end{euleroutput}
\begin{eulerprompt}
>41-0.5 //tepi bawah terkecil
\end{eulerprompt}
\begin{euleroutput}
  40.5
\end{euleroutput}
\begin{eulerprompt}
>(45-41)+1 //panjang kelas
\end{eulerprompt}
\begin{euleroutput}
  5
\end{euleroutput}
\begin{eulerprompt}
>70+0.5 //tepi atas terbesar
\end{eulerprompt}
\begin{euleroutput}
  70.5
\end{euleroutput}
\begin{eulerprompt}
>r=40.5:5:70.5; f=[10,12,18,34,20,6];
>T:=r[1:6]' | r[2:7]' | f'; writetable(T,labc=["tepi bawah", "tepi atas", "frekuensi"])
\end{eulerprompt}
\begin{euleroutput}
   tepi bawah tepi atas frekuensi
         40.5      45.5        10
         45.5      50.5        12
         50.5      55.5        18
         55.5      60.5        34
         60.5      65.5        20
         65.5      70.5         6
\end{euleroutput}
\begin{eulerprompt}
>(T[,1]+T[,2])/2; t=fold(r,[0.5,0.5]); m=mean(t,f);
\end{eulerprompt}
\begin{eulercomment}
karena data tersebut merupakan data sampel, maka menggunakan rumus
berikut
\end{eulercomment}
\begin{eulerprompt}
>sqrt(sum(f*(t-m)^2)/(sum(f)-1))
\end{eulerprompt}
\begin{euleroutput}
  6.81649810861
\end{euleroutput}
\begin{eulercomment}
Jadi, simpangan baku data kelompok tersebut adalah 6.81649810861

\end{eulercomment}
\eulersubheading{6.Mencari Jangkauan/Range}
\begin{eulercomment}
Jangkauan, atau biasa disebut range, merupakan perbedaan antara nilai
data tertinggi dan nilai data terendah dalam suatu set data. Metode
pencarian jangkauan berbeda antara data tunggal dan data kelompok.

\end{eulercomment}
\eulersubheading{a. Jangkauan/Range Data Tunggal}
\begin{eulercomment}
Bila ada sekumpulan data tunggal terurut dari yang terkecil sampai
terbesar adalah

\end{eulercomment}
\begin{eulerformula}
\[
x_1, x_2,..., x_n
\]
\end{eulerformula}
\begin{eulercomment}
maka jangkauannya adalah:

\end{eulercomment}
\begin{eulerformula}
\[
Jangkauan = x_n-n_1
\]
\end{eulerformula}
\begin{eulercomment}
Untuk menemukan jangkauan data tunggal di EMT dapat menggunakan
perintah berikut:

\textgreater{} x=[data]; max(x)-min(x)

Contoh soal

Jangkauan dari data 30,60,87,55,87,98,22,75,81,70,69,84,75 adalah...
\end{eulercomment}
\begin{eulerprompt}
>x=[30,60,87,55,87,98,22,75,81,70,69,84,75]; max(x)- min(x)
\end{eulerprompt}
\begin{euleroutput}
  76
\end{euleroutput}
\begin{eulercomment}
Jadi, jangkauan dari data tersebut adalah 76

\end{eulercomment}
\eulersubheading{b. Jangkauan data kelompok}
\begin{eulercomment}
Jangkauan pada data berkelompok adalah selisih antara batas atas dari
kelas tertinggi dengan batas bawah dari kelas terendah.

Pada EMT,  untuk menemukan jangkauan dari data berkelompok dapat
menggunakan perintah berikut:

1. Menentukan tepi bawah kelas (Tb), panjang kelas (P), dan tepi atas
kelas (Ta) dengan rumus :

\end{eulercomment}
\begin{eulerformula}
\[
T_b=a-0,5
\]
\end{eulerformula}
\begin{eulerformula}
\[
P=(b-a)+1
\]
\end{eulerformula}
\begin{eulerformula}
\[
T_a=b+0.5
\]
\end{eulerformula}
\begin{eulercomment}
dengan a = batas bawah kelas dan b = batas atas kelas

2. Mendeskripsikan data dalam bentuk tabel, dengan perintah

\textgreater{} r=tepi bawah terkecil:panjang kelas:tepi atas terbesar;
f=[frekuensi];\\
\textgreater{} T:=r[1:jumlah kelas]' \textbar{} r[2:jumlah kelas + 1]' \textbar{} f';
writetable(T,labc=["tepi bawah","tepi atas","frekuensi"])

3. Menghitung jangkauan data berkelompok

\textgreater{} max(transpose(T[,2]))-min(transpose(T[,1]))\\
Contoh soal :\\
Berikut adalah data hasil dari pengukuran berat badan 20 siswa SD
kelas V. Dari ke 20 siswa,siswa yang mempunyai berat badan dalam
rentang 21-26 kg sebanyak 5 orang, yang mempunyai berat badan dalam
rentang 27-32 kg sebanyak 4 orang, yang mempunyai berat badan dalam
rentang 33-38 kg sebanyak 3 orang, yang mempunyai berat badan dalam
rentang 39-44 kg sebanyak 2 orang, yang mempunyai berat badan dalam
rentang 45-50 kg sebanyak 3 orang, dan yang mempunyai berat badan
51-56 kg sebanyak 3 orang. Tentukan jangkauan dari\\
data hasil pengukuran berat badan 20 siswa di SD tersebut!\\
Jawab :
\end{eulercomment}
\begin{eulerprompt}
>printfile("Tabel jangkauan data kelompok.dat",7) //menyederhanakan informasi
\end{eulerprompt}
\begin{euleroutput}
  Could not open the file
  Tabel jangkauan data kelompok.dat
  for reading!
  Try "trace errors" to inspect local variables after errors.
  printfile:
      open(filename,"r");
\end{euleroutput}
\begin{eulerprompt}
>21-0.5 //tepi bawah terkecil
\end{eulerprompt}
\begin{euleroutput}
  20.5
\end{euleroutput}
\begin{eulerprompt}
>(26-21)+1 //panjang kelas
\end{eulerprompt}
\begin{euleroutput}
  6
\end{euleroutput}
\begin{eulerprompt}
>56+0.5 //tepi atas terbesar
\end{eulerprompt}
\begin{euleroutput}
  56.5
\end{euleroutput}
\begin{eulerprompt}
>r=20.5:6:56.5; f=[5,4,3,2,3,3];
>T:=r[1:6]' | r[2:7]' | f'; writetable(T,labc=["tepi bawah","tepi atas","frekuensi"])
\end{eulerprompt}
\begin{euleroutput}
   tepi bawah tepi atas frekuensi
         20.5      26.5         5
         26.5      32.5         4
         32.5      38.5         3
         38.5      44.5         2
         44.5      50.5         3
         50.5      56.5         3
\end{euleroutput}
\begin{eulerprompt}
>max(transpose(T[,2]))-min(transpose(T[,1]))
\end{eulerprompt}
\begin{euleroutput}
  36
\end{euleroutput}
\begin{eulercomment}
Jadi, Jangkauan dari data kelompok tersebut adalah 36
\end{eulercomment}
\eulersubheading{7.Menentukan ukuran letak}
\begin{eulercomment}
Ukuran letak merupakan ukuran untuk melihat dimana letak salah satu
data dari sekumpulan banyak data yang ada. Yang termasuk ukuran ukuran
letak antara lain adalah kuartil(Q), desil(D) dan persentil(P). Dalam
menentukan ke-3 nya yang harus diingat adalah mengurutkan distribusi
data dari yang terkecil sampai terbesar

1. Kuartil\\
Dalam EMT untuk menghitung kuartil bisa dilakukan dengan perintah\\
\textgreater{}quartiles(data)\\
perintah tersebut akan menghasilkan nilai Q1, Q2, Q3, nilai minimum
dan nilai maksimum dari suatu data\\
2. Desil\\
Dalam EMT untuk menghitung desil bisa dilakukan dengan perintah\\
\textgreater{}quantile(data)\\
3. Persentil\\
Dalam EMT untuk menghitung persentil bisa dilakukan dengan perintah\\
\textgreater{}quantile(data)\\
perintah "\textgreater{}quantile(data)" dapat digunakan untuk menentukan desil dan
persentil perbedaannya tergantung pada nilai dari pembaginya

Contoh soal\\
1. Tentukan Q1,Q2 dan Q3 dari data :
7,3,8,5,9,4,8,3,10,2,7,6,8,7,2,6,9.
\end{eulercomment}
\begin{eulerprompt}
>data=[7,3,8,5,9,4,8,3,10,2,7,6,8,7,2,6,9];
>urut=sort(data)
\end{eulerprompt}
\begin{euleroutput}
  [2,  2,  3,  3,  4,  5,  6,  6,  7,  7,  7,  8,  8,  8,  9,  9,  10]
\end{euleroutput}
\begin{eulerprompt}
>quartiles(urut)
\end{eulerprompt}
\begin{euleroutput}
  [2,  3.5,  7,  8,  10]
\end{euleroutput}
\begin{eulercomment}
dari hasil di atad diperoleh nilai sebagai berikut :\\
Nilai minimal data = 2\\
Q1=3.5\\
Q2=7\\
Q3=8\\
Nilai maksimal data = 10

2. Tentukan D8 dari data : 6,3,8,9,5,9,9,7,5,7,4,5,8,3,7,6
\end{eulercomment}
\begin{eulerprompt}
>data=[6,3,8,9,5,9,9,7,5,7,4,5,8,3,7,6];
>urut=sort(data)
\end{eulerprompt}
\begin{euleroutput}
  [3,  3,  4,  5,  5,  5,  6,  6,  7,  7,  7,  8,  8,  9,  9,  9]
\end{euleroutput}
\begin{eulerprompt}
>quantile(urut,0.8)  //nilai 0.8 diapatkan karena kita akan mencari D8
\end{eulerprompt}
\begin{euleroutput}
  8
\end{euleroutput}
\begin{eulercomment}
Jadi, nilai dari D8 berdasarkan perhitungan di atas adalah 8

3.Tentukan persentil ke-65 dari data : 6,5,8,7,9,4,5,8,4,7,8,5,8,4,5
\end{eulercomment}
\begin{eulerprompt}
>data=[6,5,8,7,9,4,5,8,4,7,8,5,8,4,5];
>urut=sort(data)
\end{eulerprompt}
\begin{euleroutput}
  [4,  4,  4,  5,  5,  5,  5,  6,  7,  7,  8,  8,  8,  8,  9]
\end{euleroutput}
\begin{eulerprompt}
>quantile(urut,65%)
\end{eulerprompt}
\begin{euleroutput}
  7.1
\end{euleroutput}
\begin{eulercomment}
\begin{eulercomment}
\eulerheading{Sub Topik 6: Menggambar Grafik  Statistika}
\begin{eulercomment}
\end{eulercomment}
\eulersubheading{Diagram Kotak}
\begin{eulercomment}
Diagram kotak atau box plot merupakan ringkasan distribusi sampel yang
disajikan secara grafis yang bisa menggambarkan bentuk distribusi data
(skewness), ukuran tendensi sentral dan ukuran penyebaran (keragaman)
data pengamatan. Diagram kotak sering digunakan ketika  jumlah
distribusi data perlu dibandingkan. Diagram kotak menyajikan informasi
tentang nilai--nilai inti dalam distribusi data termasuk juga
pencilan. Pencilan adalah titik data yang terpaut jauh dari titik data
lainnya.

Contoh:\\
Diketahui data berat badan mahasiswa di Universitas A sebagai berikut.
\end{eulercomment}
\begin{eulerprompt}
>A=[55,50,33,42,44,37,63,74,56,34,51,43,45,39,64,77,60,35,53,43,48,41,65,87,61,36,54,44,49,41,66,89]
\end{eulerprompt}
\begin{euleroutput}
  [55,  50,  33,  42,  44,  37,  63,  74,  56,  34,  51,  43,  45,  39,
  64,  77,  60,  35,  53,  43,  48,  41,  65,  87,  61,  36,  54,  44,
  49,  41,  66,  89]
\end{euleroutput}
\begin{eulercomment}
Buatlah diagram kotak (box plot) kemudian tuliskan interpretasinya.
\end{eulercomment}
\begin{eulerprompt}
>boxplot(A):
\end{eulerprompt}
\eulerimg{27}{images/Nur Alya Fadilah_Aplikom-327.png}
\begin{eulercomment}
Dari gambar box plot berat  badan mahasiswa Universitas  A, sepintas
kita bisa menentukan beberapa ukuran statistik, meskipun tidak persis
sekali. Nilai statistik pada badan boxplot berkisar pada: Nilai
Minimum = 33 , Q1 = 41.5 , Median (Q2) = 49.5 , Q3 = 62 , Nilai
Maksimum  = 89 . Sebaran data tidak simetris, melainkan menjulur ke
arah kanan (postively skewness). Karena nilai jarak Q1 dengan Q2 lebih
pendek dari jarak Q2 dengan Q3, maka data lebih terpusat di kiri. Akan
tetapi data tersebut tergolong cenderung mesokurtik karena jarak IQR
dengan panjang hampir sama, dengan data berpusat di angka 49.5


Adapun contoh perbandingan 10 simulasi 500 nilai terdistribusi normal
menggunakan box plot dan terdapat pencilan sebagai berikut.

\end{eulercomment}
\begin{eulerprompt}
> p=normal(10,500); boxplot(p):
\end{eulerprompt}
\eulerimg{27}{images/Nur Alya Fadilah_Aplikom-328.png}
\begin{eulercomment}
pada diagram diatas, adalah membuat boxplot distribusi normal dengan
rata-rata 10 dan standar deviasi 500. Boxplot adalah representasi
grafis dari lokalitas, penyebaran, dan kecondongan sekelompok data
numerik melalui kuartil mereka\\
2

\end{eulercomment}
\eulersubheading{Diagram Batang}
\begin{eulercomment}
Diagram batang adalah representasi visual dari data yang menggunakan
balok atau kolom vertikal untuk mewakili kategori, nilai atau variabel
tertentu. Setiap kolom yang ada pada diagram  batang memiliki
frekuensi atau jumlah dalam kategori tersebut.

Contoh:

Kita akan membuat diagram batang secara random.
\end{eulercomment}
\begin{eulerprompt}
>columnsplot(cumsum(random(6)),style="/",color=red):
\end{eulerprompt}
\eulerimg{27}{images/Nur Alya Fadilah_Aplikom-329.png}
\begin{eulerprompt}
>columnsplot(cumsum(random(15)),style="-",color=black):
\end{eulerprompt}
\eulerimg{27}{images/Nur Alya Fadilah_Aplikom-330.png}
\begin{eulerprompt}
>columnsplot(cumsum(random(3)),style="|",color=orange):
\end{eulerprompt}
\eulerimg{27}{images/Nur Alya Fadilah_Aplikom-331.png}
\begin{eulercomment}
Selanjutnya kita akan mencoba  membuat diagram batang penjualan yang
menggunakan variabel.
\end{eulercomment}
\begin{eulerprompt}
>months=["Januari","Februari","Maret","April","Mei"];
>values=[20,50,40,70,30];
>columnsplot(values,lab=months,color=yellow);
>title("Data Penjualan Beras Toko Kuning pada tahun 2023"):
\end{eulerprompt}
\eulerimg{27}{images/Nur Alya Fadilah_Aplikom-332.png}
\begin{eulercomment}
Perintah "columnsplot(values,lab=months,color=yellow);" merupakan
sintaks untuk membuat diagram batang dengan menggunakan nilai dari
variabel "values", label bulan dari variabel "months", dan warna
kuning

Dari diagram batang tersebut kita bisa mengetahui data penjualan toko
kuning selama lima bulan pada tahun 2023  yaitu, pada bulan Januari,
Februari, Maret , April, Mei. Januari terjual 20 ton beras, Februari
terjual 50 ton beras, Maret terjual 40 ton beras, April terjual 70 ton
beras, dan Mei terjual 30 ton beras.

\end{eulercomment}
\eulersubheading{Diagram Lingkaran}
\begin{eulercomment}
Diagram lingkaran merupakan penyajian statistik data tunggal dalam\\
bentuk lingkaran yang dibagi menjadi beberapa juring atau sektor yang\\
menggambarkan banyak frekuensi untuk setiap data.Diagram lingkaran\\
tidak menampilkan informasi frekuensi dari masing-masing data secara\\
detail.
\end{eulercomment}
\begin{eulerprompt}
>CP:=[rgb(0.5,0.5,0.5),red,yellow,green,rgb(0.9,0,0)]
\end{eulerprompt}
\begin{euleroutput}
  [5.87532e+07,  2,  15,  3,  6.54049e+07]
\end{euleroutput}
\begin{eulerprompt}
>i=[1,2,3,4,5]; piechart(values[i],color=CP[i],lab=months[i]):
\end{eulerprompt}
\eulerimg{27}{images/Nur Alya Fadilah_Aplikom-333.png}
\begin{eulercomment}
RGB adalah singkatan dari Red, Green, and Blue, dan setiap parameter
mendefinisikan intensitas warna dengan nilai antara 0 dan 1. Warna
pertama dalam daftar adalah warna abu-abu dengan jumlah merah, hijau,
dan biru yang sama. Warna kedua merah, ketiga kuning, dan keempat
hijau. Warna terakhir adalah warna merah dengan lebih banyak merah
daripada hijau atau biru.

\end{eulercomment}
\eulersubheading{Diagram Bintang}
\begin{eulercomment}
Diagram bintang, terkadang disebut diagram radar atau diagram web,
adalah metode perangkat grafis yang digunakan untuk menampilkan data
multivariat. Multivariat dalam pengertian ini mengacu pada memiliki
banyak karakteristik untuk diamati. Variabelnya juga harus berupa
nilai yang berkisar.\\
Diagram bintang terdiri dari rangkaian jari-jari bersudut sama, yang
disebut jari-jari, dengan masing-masing jari mewakili salah satu
variabel. Panjang jari-jari data sebanding dengan besaran variabel
pada titik data relatif terhadap besaran maksimum variabel di
seluruh titik data.
\end{eulercomment}
\begin{eulerprompt}
>starplot(normal(1,15)+16,lab=1:15,>rays):
\end{eulerprompt}
\eulerimg{27}{images/Nur Alya Fadilah_Aplikom-334.png}
\begin{eulerprompt}
>starplot(values,lab=months,>rays):
\end{eulerprompt}
\eulerimg{27}{images/Nur Alya Fadilah_Aplikom-335.png}
\begin{eulercomment}
Syntax starplot(values,lab=months,rays) adalah perintah untuk membuat
grafik bintang (star plot) dengan menggunakan nilai-nilai yang
diberikan dalam vektor values, label sumbu yang diberikan dalam vektor
months, dan jumlah rays yang menentukan jumlah garis radial yang
digunakan dalam grafik


\end{eulercomment}
\eulersubheading{Diagram Impuls}
\begin{eulercomment}
Impuls (impulse) adalah perubahan momentum. Contohnya adalah sebuah
bola bermassa yang tengah ditendang, bola menggelinding yang
dihentikan, bola jatuh yang memantul, mobil yang menabrak tembok,
telur jatuh yang pecah.\\
Berikut adalah plot impuls dari data acak 1 sampai 20, terdistribusi
secara merata di [0,1].
\end{eulercomment}
\begin{eulerprompt}
>plot2d(makeimpulse(1:20,random(1,20)),>bar):
\end{eulerprompt}
\eulerimg{27}{images/Nur Alya Fadilah_Aplikom-336.png}
\begin{eulercomment}
Tetapi untuk data yang terdistribusi secara eksponensial, kita mungkin
memerlukan plot logaritmik.
\end{eulercomment}
\begin{eulerprompt}
> logimpulseplot(1:20,-log(random(1,20))*10):
\end{eulerprompt}
\eulerimg{27}{images/Nur Alya Fadilah_Aplikom-337.png}
\begin{eulercomment}
Jadi gambar grafiknya terlihat naik turun (mengalami perubahan).
\end{eulercomment}
\eulersubheading{Histogram}
\begin{eulercomment}
Histogram adalah representasi grafis (diagram) yang mengatur dan
menampilkan frekuensi data sampel pada rentang tertentu. Frekuensi
data yang ada pada masing-masing kelas direpresentasikan dengan bentuk
grafik diagram batang atau kolom.
\end{eulercomment}
\begin{eulerprompt}
>aspect(1); plot2d(random(100),>histogram):
\end{eulerprompt}
\eulerimg{27}{images/Nur Alya Fadilah_Aplikom-338.png}
\begin{eulerprompt}
>r=150:5:185; v=[22,71,136,150,139,71,32];
>plot2d(r,v,a=150,b=185,c=0,d=150,bar=1,style="/"):
\end{eulerprompt}
\eulerimg{27}{images/Nur Alya Fadilah_Aplikom-339.png}
\begin{eulercomment}
Pola "r=150:5:185" berarti bahwa nilai r dimulai dari 150, kemudian
bertambah 5 setiap kali, dan berakhir saat mencapai atau melebihi 185.
Dengan pola ini, kita dapat menentukan nilai-nilai r yang sesuai.

Dari data yang diperoleh dapat diketahui bahwa dari rentang kelas
150-155 memiliki frekuensi 22, rentang kelas 155-160 memiliki
frekuensi 71,  dan seterusnya.

\end{eulercomment}
\eulersubheading{Kurva Fungsi Kerapatan Probabilitas}
\begin{eulercomment}
Secara teoritis kurva probabilitas populasi diwakili oleh poligon
frekuensi relatif yang dimuluskan (variabel acak  kontiniu
diperlakukan seperti variabel acak diskrit yang rapat).Karena itu
fungsi dari variabel acak kontiniu merupakan fungsi kepadatan
probabilitas (probability density function – pdf). Pdf menggambarkan
besarnya probabilitas per unit interval nilai variabel acaknya.
\end{eulercomment}
\begin{eulerprompt}
>plot2d("qnormal(x,0,1)",-5,5);  ...
>plot2d("qnormal(x,0,1)",a=1,b=4,>add,>filled):
\end{eulerprompt}
\eulerimg{27}{images/Nur Alya Fadilah_Aplikom-340.png}
\begin{eulercomment}
Perintah "plot2d("qnormal(x,0,1)",-5,5)" digunakan untuk membuat plot
dari distribusi normal dengan mean 0 dan standard deviation 1 di
rentang -5 hingga 5

Probabilitas variabel acak x yang terletak antara 1 dan 4 memenuhi\\
P(1\textless{}X\textless{}4)= luas daerah hijau

\end{eulercomment}
\eulersubheading{Kurva Fungsi Distribusi Kumulatif}
\begin{eulercomment}
Cumulative Distribution Function (CDF) atau fungsi distribusi
kumulatif adalah fungsi matematika yang digunakan untuk menghitung
probabilitas variabel acak diskrit atau kontinu. CDF memberikan
probabilitas bahwa variabel acak akan menghasilkan nilai kurang dari
atau sama dengan nilai tertentu. Dalam hal ini, CDF dapat digunakan
untuk menghitung probabilitas kumulatif dari variabel acak.

Berikut merupakan contoh kurva fungsi distribusi kumulatif kontinu:
\end{eulercomment}
\begin{eulerprompt}
>splot2d("normaldis",-3,5):
\end{eulerprompt}
\begin{euleroutput}
  Function splot2d not found.
  Try list ... to find functions!
  Error in:
  splot2d("normaldis",-3,5): ...
                           ^
\end{euleroutput}
\begin{eulercomment}
Dapat kita lihat dalam kurva fungsi distribusi kumulatif kontinu
terdiri atas tiga bagian yaitu:\\
1. Bernilai 0 untuk x di  bawah minimal dari daerah rentang.\\
2. Merupakan fungsi monoton naik pada daerah rentang.\\
3. Mempunyai nilai konstan 1 di atas batas maksimum daerah rentangnya.

Adapun contoh kurva fungsi distribusi kumulatif diskrit sebagai
berikut.
\end{eulercomment}
\begin{eulerprompt}
>x=normal(1,6);
\end{eulerprompt}
\begin{eulercomment}
Baris kode tersebut akan menghasilkan suatu nilai acak dari distribusi
normal dengan mean 1 dan deviasi standar 6, dan nilai tersebut
disimpan dalam variabel x. Variabel x kemudian dapat digunakan dalam
perhitungan atau analisis selanjutnya

Fungsi empdist(x,vs) membutuhkan array nilai yang diurutkan. Jadi kita
harus mengurutkan x sebelum kita dapat menggunakannya.
\end{eulercomment}
\begin{eulerprompt}
>xs=sort(x);
>plot2d("empdist",-3,5;xs):
\end{eulerprompt}
\eulerimg{27}{images/Nur Alya Fadilah_Aplikom-341.png}
\begin{eulercomment}
Grafik fungsi distribusi kumulatif peubah acak diskrit merupakan
fungsi tangga naik dengan nilai terendah 0 dan nilai tertinggi 1.
\end{eulercomment}
\begin{eulercomment}

\end{eulercomment}
\eulersubheading{Menyimpan Data Hasil Analisis}
\begin{eulercomment}
Data-data yang kita gunakan dalam melakukan analisis statistika dapat
kita simpan ke dalam suatu file sehingga ketika kelak ingin digunakan
lagi, data tersebut masih ada di file penyimpanan kita. Tak hanya itu,
hasil dari analisis statistika yang sudah kita lakukan pun dapat kita
simpan ke dalam suatu file.

Berikut adalah cara menyimpan/menulis data ke suatu file.
\end{eulercomment}
\begin{eulerprompt}
>a=random(1,100); mean(a); dev(a);
>filename="Simpan";
\end{eulerprompt}
\begin{eulercomment}
Seletah memberi nama untuk file, kita akan menulis vektor a ke dalam
file dengan menggunakan fungsi writematrix() dan menggunakan fungsi
readmatrix() untuk membaca data.
\end{eulercomment}
\begin{eulerprompt}
>writematrix(a',filename);
>a=readmatrix(filename)';
\end{eulerprompt}
\begin{eulercomment}
Kita juga bisa menghapus file yang sudah tersimpan dengan menggunakan
fileremove
\end{eulercomment}
\begin{eulerprompt}
>fileremove(filename);
\end{eulerprompt}
\begin{eulercomment}
Kemudian kita akan mencoba untuk menggantikan data baru ke file lama
dengan menghapus semua data lama, dan menulis lagi data baru yang akan
disimpan.
\end{eulercomment}
\begin{eulerprompt}
>file="Simpan"; open(file,"w");
>writeln("A,B,C"); writematrix(random(3,3));
>close();
>printfile(file)
\end{eulerprompt}
\begin{euleroutput}
  A,B,C
  0.4740947109172355,0.6328068695909269,0.5833355628808601
  0.5532677420544243,0.08495847192661377,0.9098541742831491
  0.9660500335107803,0.4155435349425303,0.9654721548152776
  
\end{euleroutput}
\begin{eulercomment}
Selain itu kita juga bisa menyimpan dalam bentuk excel
\end{eulercomment}
\begin{eulerprompt}
>file="test.csv";
>M=random(3,3); writematrix(M,file);
\end{eulerprompt}
\begin{eulercomment}
Berikut adalah isi dari file ini.
\end{eulercomment}
\begin{eulerprompt}
>printfile(file)
\end{eulerprompt}
\begin{euleroutput}
  0.1409789519306323,0.6345761036216137,0.3852004398321907
  0.1785920128134643,0.6613914250481273,0.04869641928896105
  0.9168512648587849,0.04738687026681564,0.3943845525023932
  
\end{euleroutput}
\begin{eulercomment}
CVS ini dapat dibuka pada sistem bahasa Inggris ke dalam Excel dengan
klik dua kali. Jika Anda mendapatkan file seperti itu di sistem
Jerman, Anda perlu mengimpor data ke Excel dengan memperhatikan titik
desimal.

Tetapi titik desimal juga merupakan format default untuk EMT. Anda
dapat membaca matriks dari file dengan readmatrix().
\end{eulercomment}
\begin{eulerprompt}
>readmatrix(file)
\end{eulerprompt}
\begin{euleroutput}
       0.140979      0.634576        0.3852 
       0.178592      0.661391     0.0486964 
       0.916851     0.0473869      0.394385 
\end{euleroutput}
\begin{eulercomment}
b. Hal hal yang dilakukan dalam mempelajari materi\\
- Mencari informasi mengenai materi statistika.\\
- mencarai latihan soal di buku dan internet.\\
- Mempelajari perintah perintah yang ada di EMT berkaitan dengan
statistika.\\
- Mencari data untuk percobaan perintah EMT.

c. Kendala kendala dan usaha untuk mengatasi kendala tersebut\\
- Kesulitan dalam memahami perintah yang berkaitan dengan statistika,
solusinya dengan menonton youtube.\\
- Kesulitan dalam memahami hasil perintah EMT, solusinya dengan
mempelajari referensi EMT\\
\end{eulercomment}
\eulersubheading{}
\eulerheading{9. Pengolahan dokumen menggunakan LaTeX.}
\begin{eulercomment}
a. Hal hal yang dipelajari beserta contohnya\\
- Menggunakan software LaTeX\\
- Menghasilkan dokumen menggunakan LaTeX\\
- Menggunakan Overleaf\\
- Membuat akun Github

Contoh gambar Overleaf dan Github\\
image: G2.png

image: G1.png

b. Hal hal yang dilakukan dalam mempelajari materi\\
- Mencari informasi mengenai LaTeX\\
- Mencari informasi cara membuat akun Github\\
- Menonton youtube dalam menggunakan Overleaf

c. Kendala kendala dan usaha untuk mengatasi kendala tersebut\\
- Kesulitan dalam menggunakan overleaf, solusinya dengan menonton
youtube cara kerja overleaf.\\
\end{eulercomment}
\eulersubheading{}
\end{eulernotebook}
\end{document}
